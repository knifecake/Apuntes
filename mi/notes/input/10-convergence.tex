% !TeX root = ../mi-notes.tex

\chapter{Convergence theorems}

The purpose of this chapter is to generalise the results that were discussed in chapter 7, namely \autoref{thm:beppo-levi}, \autoref{thm:fatou-lemma} and the reverse Fatou lemma. By generalisation we mean that we will remove the restrictions about positive functions and/or increasing sequences.

\section{Convergence theorems}

First, we extend \autoref{thm:beppo-levi} to negative functions.
\begin{thm}[Monotone convergence]
	\label{thm:monotone-convergence}
	Let $(X, \sa, \mu)$ be a measure space and $(u_n)_{n\in\N} \subset \calL^1(\mu)$ be an increasing sequence of measurable functions such that
	\begin{align*}
		u_n \leq u_{n+1},\ \forall n \in \N\text{ and define } u = \sup_{n\in \N} u_n = \lim_{n \to \infty} u_n \in \calM_\Rb(\sa).
	\end{align*}
	Then,
	\begin{align}
		\label{eq:monotone-convergence-1}
		u \in \calL^1(\mu) \iff \sup_{n\in\N} \int u_n d\mu < \infty,
	\end{align}
	and in that case
	\begin{align}
		\label{eq:monotone-convergence-2}
		\int u d\mu = \int \lim_{n \to \infty} u_n d\mu = \int \sup_{n\in\N} u_n = \sup_{n\in\N} \int u_n d\mu = \lim_{n \to \infty} \int u_n d\mu.
	\end{align}
\end{thm}

Here we're writting both the $\lim$ and the $\sup$ versions of the equalities to be extra clear at the expense of being too verbose. This is not the case in \cite[p- 88]{schilling2017}.

\begin{proof}
	We want to use Beppo-Lévi but the functions may be negative, so we define a new sequence $w_n = u_n - u_1 \geq 0$ since $u_1 \leq u_n,\ \forall n \in \N$. Clearly, $w_n \in \calL^1(\mu) \subset \calM(\sa)$ and $0 \leq w_1 \leq w_2 \leq \dots$.
	
	Firstly, by \autoref{thm:beppo-levi},
	\begin{align*}
		0 \leq \int \sup_{n\in\N} w_n d\mu = \sup_{n\in\N} \int w_n d\mu. 
	\end{align*}
	Equivalently,
	\begin{align*}
		0 &\leq \int \sup(u_n - u_1)d\mu \\
		&= \sup_{n\in\N}\left(\int (u_n - u_1)d\mu \right)  \\
		&= \sup_{n\in\N}\left( \int u_n d\mu - \int u_1 d\mu \right) \\
		&= \sup_{n\in\N}\int u_n d\mu - \int u_1 d\mu.
	\end{align*}
	Note how we wait until $\sup$ is outside the integral to apply linearity, since we don't know if $\sup(u_n - u_1) \in \calL^1(\mu)$ yet.
	
	Assume $u = \sup_n u_n \in \calL^1(\mu)$. Then $\sup_n (u_n - u_1) = \sup_n u_n - u_1 \in \calL^1(\mu)$, and the above gives
	\begin{align*}
		\int \sup_n u_n d\mu - \int u_1 d\mu = \sup_n \int u_n d\mu - \int u_1 d\mu.
	\end{align*}
	Since $u_1 \in \calL^1(\mu)$, its integral is finite and we can subtract it from both sides to get
	\begin{align*}
		\sup_n \int u_n d\mu = \int \sup_n u_n d\mu < \infty.
	\end{align*}
	
	Conversely, assume $\sup_n \int u_n d\mu < \infty$. Then,
	\begin{align*}
		\int \sup_n (u_n - u_1)d\mu = \sup_n \int u_n d\mu - \int u_1 d\mu < \infty,
	\end{align*}
	which implies that $\sup(u_n - u_1) = \sup_n u_n - u_1 \in \calL^1(\mu)$. Since $u_1 \in \calL^1(\mu)$, we see that $\sup_n u_n \in \calL^1(\mu)$ as required. Also, if $\sup_n u_n \in \calL^1(\mu)$ the above shows \autoref{eq:monotone-convergence-2}.
\end{proof}

Of course, we can state the same for decreasing sequences and infima, as taking $u_n = - v_n$ for some decreasing sequence $(v_n)$ is enough to fulfill the assumptions of the \autoref{thm:monotone-convergence}. Anyway, we can state the result.

\begin{cor}
	Let $(u_n)_{n\in\N} \subset \calL^1(\mu)$ be a sequence of decreasing integrable functions such that
	\begin{align*}
	u_n \geq u_{n+1},\ \forall n \in \N\text{ and define } u = \inf_{n\in\N} u_n = \lim_{n \to \infty} u_n \in \calM_\Rb(\sa).
	\end{align*}
	Then,
	\begin{align}
	u \in \calL^1(\mu) \iff \inf_{n\in\N} \int u_n d\mu > -\infty,
	\end{align}
	in which case
	\begin{align}
	\int u d\mu 
	= \int \lim_{n \to \infty} u_n d\mu
	= \int \inf_{n\in\N} u_n d\mu
	= \inf_{n\in\N} \int u_n d\mu
	= \lim_{n \to \infty} \int u_n d\mu.
	\end{align}
\end{cor}

\begin{proof}
	Set $v_n = -u_n$ and apply the proof of \autoref{thm:monotone-convergence}.
\end{proof}

Now we move on to a very important result. This time, not only we drop non-negativity, but also, monotonicity. Of course, there is a price to pay for this.

\begin{thm}[Lebesgue Dominated Convergence]
	\label{thm:lebesgue-dominated-convergence}
	Let $(X, \sa, \mu)$ be a measure space and $(u_n)_{n\in\N} \subset \calL^1(\mu)$ be a sequence of functions such that
	\begin{enumerate}
		\item there exists $0 \leq w \in \calL^1(\mu)$ such that $\abs{u_n} \leq w \muae$ for all $n \in \N$, and
		\item $u = \lim_{n\to\infty} u_n$ exists in $\Rb\muae$.  
	\end{enumerate}
	
	Then $u \in \calL^1(\mu)$ and we have
	\begin{align}
		\lim_{n \to \infty} \int |u_n - u| d\mu = 0,
	\end{align}
	\begin{align}
		\lim_{n \to \infty} \int u_n d\mu = \int \lim_{n \to \infty} u_n d\mu = \int ud\mu.
	\end{align}
\end{thm}

\begin{proof}
	TODO
\end{proof}

\begin{remark}
	We have proved the theorem for the case in which $\abs{u_n} \leq w$ everywhere and $\lim_{n\to\infty} u_n \in \Rb$ everywhere, but they may both be relaxed to $\muae$.
\end{remark}

\section{Applications to parameter dependent-integrals}

\begin{thm}[Continuity lemma]
	\label{thm:continuity-lemma}
	
	Let $(X, \sa, \mu)$ be a measure space and $\emptyset \neq (a,b) \subset \R$ a non-degenerate open interval and $u: (a, b) \times X \to \R$ be a function satisfying
	\begin{enumerate}
		\item $x \mapsto u(t, x)$ is in $\calL^1(\mu)$ for every fixed $t \in (a, b)$;
		\item $t \mapsto u(t, x)$ is continuous for every fixed $x \in X$; and
		\item $|u(t, x)| \leq w(x)$ for all $(t, x) \in (a, b) \times X$ and some $w \in \calL^1_\plus(\mu)$.
	\end{enumerate}
	Then the function $v: (a, b) \to \R$ given by
	\begin{align}
		t \mapsto v(t) := \int u(t, x) \mu(dx)
	\end{align}
	is continuous.
\end{thm}

\begin{thm}[Differentiability lemma]
	\label{thm:differentiability-lemma}
	
	Let $(X, \sa, \mu)$ be a measure space, $\emptyset \neq (a, b) \subset \R$ a non-degenerate open interval and $u: (a, b) \times X \to \R$ be a function satisfying
	\begin{enumerate}
		\item $x \mapsto u(t, x)$ is in $\calL^1(\mu)$ for every fixed $t \in (a, b)$;
		\item $t \mapsto u(t, x)$ is differentiable for every fixed $x \in X$; and
		\item $|\partial_t u(t, x)| \leq w(x)$ for all $(t, x) \in (a, b) \times X$ and some $w \in \calL^1_\plus(\mu)$.
	\end{enumerate}
	Then the function $v: (a,b) \to \R$ given by
	\begin{align}
		t \mapsto v(t) := \int u(t, x) \mu(dx)
	\end{align}
	is differentiable and its derivative is
	\begin{align}
		\partial_t v(t) = \int \partial_t u(t, x) \mu(dx).
	\end{align}
\end{thm}

\section{Riemann integral vs. Lebesgue integral}

\begin{thm}
	Let $u: [a, b] \to \R$ be a measurable and Riemann-integrable function. Then $u \in \calL^1(\lambda)$ and the Riemann integral and Lebesgue integral agree.
	\begin{align}
		\int_a^b u dx = \int_{[a, b]} u d\lambda
	\end{align}
\end{thm}

\begin{remark}
	Note that the converse need not be true. For instance the Dirichlet function $\ind_{Q}$ (see \cite{dirichlet}) is Lebesgue integrable but not Riemann-integrable.
\end{remark}

\begin{proof}
	TODO
\end{proof}

\begin{thm}
	Let $u:[a, b] \to \R$ be a bounded and Riemann-integrable function, then
	\begin{align}
		\{x \in X \mid u \text{ is not continuous in } x\} \subseteq N \in \calN_\mu.
	\end{align}
\end{thm}

Note that the additional subset $N$ is not required if $u$ is measurable.

\subsection{Improper Riemann integrals}

\begin{cor}
	Let $u:[a, b] \to \R$ be a measurable function that is Riemann integrable on $[0, N]$ for all $N\geq 1$. Then $u \in \calL^1([0,\infty])$ if, and only if,
	\begin{align}
		\lim_{N\to\infty} \int_0^N \abs{u}dx < \infty.
	\end{align}
	In this case,
	\begin{align*}
		\int_0^\infty udx = \int_{[0, \infty]} u d\lambda.
	\end{align*}
\end{cor}