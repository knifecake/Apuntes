% !TeX root = ../mi-notes.tex


\chapter{Exercise sets}

\section{Exercise set 1}

Due September 20th, 2019.


\begin{ex}
	Let $X$ be a nonempty set and $\sa = \{A_1, A_2, \dots\}$ a collection of disjoint subsets of $X$ such that $X = \bigcup_{n = 1}^\infty A_n$. Show that each element $A \in \sigma(\sa)$ is a union of at most a countable subcollection of elements of $\sa$. (3 pts)
\end{ex}

The technique used here is similar to the good set principle (\autoref{rem:good-set-principle}), only that it is not necessary to show that the good set is inside the \siga.

\begin{proof}
	The idea is to prove that $\sigma(\sa)$ only contains countable unions of sets of $\sa$.
	
	Let us define
	\begin{align*}
		\calB = \left\{\bigcup_{i \in I} A_i \mid A_i \in \sa \land I \subset \N\right\}.
	\end{align*}
	
	We shall prove that $\calB$ is a \siga on $\sa$ and thus $\sigma (\sa) \subseteq \calB$, i.e. $\sigma(\sa)$ is made up of unions of at most a countable subcollection of $\sa$.
	
	\begin{enumerate}
		\item Firstly, $\emptyset \in \calB$ since $\emptyset = \bigcup_{i \in I} A_i$ by choosing $I = \emptyset \subset \N$.
		\item Secondly, for any set $B = \bigcup_{i \in I} A_i \in \calB$ we have that
		\begin{align*}
			B^\complement = \left(\bigcup_{i \in I} A_i\right)^\complement = \bigcap_{i \in I} A_i^\complement = \bigcap_{i \in I} \bigcup_{j \neq i} A_j = \bigcup_{j \not\in I} A_j = \bigcup_{j \in I^\complement} A_j \in \calB,
		\end{align*}
		since $I^\complement = \N \setminus I \subset \N$.
		
		\item Finally, for any countable collection $(B_n)_{n \in \N} \in \calB$ we need to show that the union is also in $\calB$.
		\begin{align*}
			\bigcup_{n\in\N} B_n = \bigcup_{n\in\N} \bigcup_{i \in I_n} A_i = \bigcup_{j \in J} A_j \in \calB,
		\end{align*}
		since $J = \bigcup_{n \in \N} I_n \subset \N$.
	\end{enumerate}
	Since we have shown that $\calB$ is a \siga on $\sa$ and thus $\sigma(\sa) \subset \calB$, we conclude that the elements of $\sigma(\sa)$ must be all unions of at most countable subcollections of elements in $\sa$, which are the only elements of $\calB$.
\end{proof}

\begin{ex}
	Let $(X, \calD, \mu)$ be a measure space, and let $\overline{\calD}^\mu$ be the completion of the \siga $\calD$ with respect to the measure $\mu$ (see exercise 4.15). We denote by $\overline{\mu}$ the extension of the measure $\mu$ to the \siga $\overline{\calD}^\mu$. Supose $f: X \to X$ is a function such that $\inv{f}(B) \in \calD$ and $\mu(\inf{f}(B)) = \mu(B)$ for each $B \in \calD$. Show that $\inf{f}(\overline{B}) \in \overline{\calD}^\mu$ and $\overline{\mu}(\inv{f}(\overline{B})) = \overline{\mu}(\overline{B})$ for all $\overline{B} \in \overline{\calD}^\mu$. (3 pts)
\end{ex}

\begin{proof}
	First we show that $\inv{f}(\overline{B}) \in \overline{\calD}^\mu$, for all $\overline{B} \in \overline{\calD}^\mu$. Recall from the definition of the completion $\overline{\calD}^\mu$ of the \siga $\calD$ that any set $\overline{B} \in \overline{\calD}^\mu$ can be written as $\overline{B} = B \cap M$ for some subset $M$ of a $\mu$-measurable null set $N$ in $\calD$. Therefore
	\begin{align*}
		\inv{f}(\overline{B}) = \inv{f}(B \cup M) = \inv{f}(B) \cup \inv{f}(M)
	\end{align*}
	Because $N \supset M$ we also have that $\inv{f}(N) \subset \inv{f}(M)$ and $\mu(\inv{f}(N)) = \mu(N) = 0$ by the definition of $f$. This means that $\inv{f}(M)$ is also a subset of a $\mu$-measurable null set in $\calD$ and since $\inv{f}(B) \in \calD$ by definition of $f$ we have
	\begin{align*}
		\inv{f}(\overline{B}) = \underbrace{\inv{f}(B)}_{\in \calD} \cup \underbrace{\inv{f}(M)}_{\subset \inv{f}(N),\ \mu(\inv{f}(N)) = 0} \in \overline{\calD}^\mu
	\end{align*}
	
	Now we need to verify that $\overline{\mu}(\inv{f}(\overline{B})) = \overline{\mu}(\overline{B})$ for all $\overline{B} \in \overline{\calD}^\mu$. Recall that the extension $\overline{\mu}$ is well-defined in $\overline{\calD}^\mu$ with $\overline{\mu}(\overline{B}) := \mu(B)$ for any $\overline{B} = B \cup M \in \overline{\calD}^\mu$. Hence,
	\begin{align*}
		\overline{\mu}(\inv{f}(\overline{B}))
		&= \overline{\mu}(\inv{f}(B \cup M)) \\
		&= \overline{\mu}(\inv{f}(B) \cup \inv{f}(M)) \\
		&= \mu(\inv{f}(B)) \\
		&= \mu(B) =: \overline{\mu}(\overline{B})
	\end{align*}
\end{proof}

\begin{ex}
	Let $(X, \sa)$ be a measurable space and $\mu : \sa \to [0, \infty)$ a function satisfying
	\begin{enumerate}
		\item $\mu$ is finitely additive
		\item $\mu$ is $\sigma$-subadditive
	\end{enumerate}
	Show that $\mu$ is $\sigma$-additive. (4 pts)
\end{ex}

\begin{proof}
	The plan for the proof is to sandwich $\mu(\bigcup_{n\in\N} A_n)$ between two sums that are the same when taking the limit.
		
	First of all, because of $\sigma$-subadditivity we have that, for any countable collection $(A_n)_{n\in\N} \subset \sa$,
	\begin{align}
		\label{eq:es1.3a}
		\mu(\bigcup_{n\in\N} A_n) \leq \sum_{n\in \N} \mu(A_n),
	\end{align}
	which in particular holds for pairwise disjoint unions, which we will assume from here on. We can rewrite the union as
	\begin{align*}
		\bigcupdot_{n\in \N} A_n = \bigcupdot_{n = 1}^N \cupdot \bigcupdot_{n = N+1}^\infty A_n.
	\end{align*}
	Because of finite additivity we can introduce the measure as
	\begin{align*}
		\mu\left(\bigcupdot_{n\in \N} A_n\right) = \mu\left(\bigcupdot_{n = 1}^N\right) + \mu\left(\bigcupdot_{n = N+1}^\infty A_n\right).
	\end{align*}
	And applying finite additivity again we get
	\begin{align*}
		\mu\left(\bigcupdot_{n\in \N} A_n\right) = \sum_{n = 1}^N \mu(A_n) + \mu\left(\bigcupdot_{n = N+1}^\infty A_n\right).
	\end{align*}
	Since $\mu \geq 0$ we can rearrange the previous expression to obtain
	\begin{align}
		\label{eq:es1.3b}
		\sum_{n = 1}^N \mu(A_n) \leq \sum_{n = 1}^N \mu(A_n) + \mu\left(\bigcupdot_{n = N+1}^\infty A_n\right) = 	\mu\left(\bigcupdot_{n\in \N} A_n\right)
	\end{align}
	By combining \ref{eq:es1.3a} and \ref{eq:es1.3b} we get
	\begin{align*}
		\sum_{i = 1}^N \mu(A_n) \leq \mu\left(\bigcupdot_{n\in \N} A_n\right) \leq \sum_{n \in \N} \mu(A_n)
	\end{align*}
	Taking the limit as $N \to \infty$, which we can do since $N$ is not inside the arguments to $\mu$ (in that case it would require that $\mu$ was already a measure, which it isn't, yet), we have
	\begin{align*}
		\sum_{n \in \N} \mu(A_n) \leq \mu\left(\bigcupdot_{n\in \N} A_n\right) \leq \sum_{n \in \N} \mu(A_n) \implies \mu\left(\bigcupdot_{n\in \N} A_n\right) = \sum_{n \in \N} \mu(A_n),
	\end{align*}
	or that $\mu$ is $\sigma$-additive.
\end{proof}


\section{Exercise set 2}

Due September 27th, 2019.

\begin{ex}
	Let $\Q$ be the set of all real rational numbers and let $\calI_\Q = \{[a, b)_\Q \mid a,b \in \Q\}$ where $[a,b)_\Q = \{q \in \Q \mid a \leq q < b\}$.
	\begin{enumerate}
		\item Prove that $\sigma(\calI_\Q) = \powerset(\Q)$ where $\powerset(\Q)$ is the collection of all subsets of $\Q$. (1.5 pts.)
		\item Let $\mu$ be the counting measure on $\powerset(\Q)$ and let $\nu = 2\mu$. Show that $\nu(A) = \mu(A)$ for all $A \in \calI_\Q$, but $\nu \neq \mu$ on $\sigma(\calI_\Q) = \powerset(\Q)$. Why doesn't this contradict Theorem 5.7 in your book? (1.5 pts.)
	\end{enumerate}
\end{ex}

\begin{proof}$ $\newline
	\begin{enumerate}
		\item We shall prove the double containment. First, recall that $\powerset(\Q)$ is a \siga on $\Q$. Also, by Remark 3.5 we have that $\calI_\Q \subseteq \powerset(\Q) \implies \sigma(\calI_\Q) \subseteq \sigma(\powerset(\Q)) \subseteq \powerset(\Q)$. The last inclusion comes from the fact that $\powerset(\Q)$ is also a \siga on $\Q$ so it must contain the smallest \siga on $\Q$ that contains information about $\powerset(\Q)$. For the reverse containment, we shall prove that any subset $A \in \powerset(\Q)$ is also in $\sigma(\calI_\Q)$. For any $A \subset \Q$ define $(q_n)_{n \in \N}$ to be an enumeration of the rationals in $A$. This is posible since $\# \Q = \# \N$. Therefore, we can write
		\begin{align*}
			A = \bigcup_{n \in \N} \{q_n\},\text{ where } \{q_n\} \in \sigma(\calI_\Q)
		\end{align*}
		Therefore, $A \in \sigma(\calI_\Q)$ because \sigas are closed under countable union.
		
		\item It is clear that any interval $A \in \calI_\Q$ contains infinitely many rationals, except if the interval is empty, i.e. $a = b \implies [a,b) = \emptyset$. Therefore,
		\begin{align*}
			\mu(A) = \nu(A) = \begin{cases}
			0       \tif A = \emptyset \\
			\infty &\text{ otherwise }
			\end{cases}
		\end{align*}
		But, if we consider $A \in \sigma(\calI_\Q) = \powerset(\Q)$ then we have some finite sets where $\mu(A) = \# A$, and clearly $\nu(A) = 2 \# A$. The equality between $\mu$ and $\nu$ only holds when the set is either empty or infinite, but not for finite sets such as $A = \{1, 2\}$ where $\mu(A) = 2$ but $\nu(A) = 4$.
		
		Why doesn't this contradict Theorem 5.7? Even though there is an exhausting sequence in the generator, namely $(A_n)_{n\in\N}$ where $A_n = [-n, n)_\Q$, the measure is not finite for any $A_n$. Moreover, there cannot be any exhausting sequence in $\calI_\Q$ with a finite measure because we already saw that $\mu(A) = \infty,\ \forall A \in \calI_\Q,\ A \neq \emptyset$.
	\end{enumerate}
\end{proof}

\begin{ex}
	Let $X$ be a set and $\mu, \nu:\powerset(X) \to [0, \infty)$ two outer measures on $X$. Define $\rho: \powerset(X) \to [0, \infty)$ by $\rho(A) = \max(\mu(A), \nu(A))$. Show that $\rho$ is another outer measure on $X$.
\end{ex}

\begin{proof}
	Firstly, from the definition of $\rho$ we can see that the domain and codomain are compatible with the definition of an outer measure. Next, we prove each of the properties of an outer measure.
	\begin{enumerate}
		\item $\rho(\emptyset) = \max(\mu(\emptyset), \nu(\emptyset)) = \max (0, 0) = 0$
		\item For any $A, B \in \powerset(X),\ A \subseteq B$ we have
		\begin{align*}
			\rho(A) =\max(\mu(A), \nu(A)) \leq \max (\mu(B), \nu(B)) = \rho(B)
		\end{align*}
		\item For any sequence $(A_n)_{n\in \N} \subset \powerset(X)$ we must verify that $\rho(\bigcup_{n \in \N} A_n) \leq \sum_{n\in \N} \rho(A_n)$.
		
		Let us prove the following first. For any function $f: X \times Y \to \R$ we have
		\begin{align*}
			\max_{x \in X} \sum_{y \in Y} f(x, y) \leq \sum_{y \in Y} \max_{x \in X} f(x, y)
		\end{align*}
		Choose any $x_0 \in X$ and any $y_0 \in Y$ and we have that $f(x_0, y_0) \leq \max f(x_0, y_0)$. Hence $\sum_{y \in Y} f(x_0, y) \leq \sum_{y \in Y} f(x, y)$. Because this is true for all $x_0 \in X$ we have $\max_{x \in X} \sum_{y \in Y} f(x, y) \leq \sum_{y \in Y} \max_{x \in X} f(x,y)$.
		
		Using this, i.e. choosing $Y = \N,\ X = \{\mu, \nu\}$ and defining $f(\mu, n) = \mu(A_n)$ and $f(\nu, n) = \nu(A_n)$ we have
		\begin{multline*}
			\rho(\bigcup_{n \in \N} A_n) = \max\left\{\mu\left(\bigcup_{n \in \N} A_n\right), \nu\left(\bigcup_{n \in \N} A_n\right) \right\} = \max \left\{\sum_{n\in \N} \mu(A_n), \sum_{n\in \N} \nu(A_n) \right\} \\
			\leq \sum_{n\in \N} \max\{\mu(A_n), \nu(A_n)\} = \sum_{n\in \N} \rho(A_n)
		\end{multline*}
	\end{enumerate}
\end{proof}


\begin{ex}
	Let $(X, \sa, \mu)$ be a measure space. For $A \in \sa$ let
	\begin{align*}
		S(A) := \{B \in \sa \mid B \subset A, \mu(B) < \infty\}.
	\end{align*}
	
	Define $\nu: \sa \to [0, \infty]$ by $\nu(A) := \sup\{ \mu(B) \mid B \in S(A)\}$.
	
	\begin{enumerate}
		\item Show that $\nu$ is monotone, i.e. if $A_1, A_2 \in \sa$ such that $A_1 \subseteq A_2$, then $\nu(A_1) \leq \nu(A_2)$. (0.5 pts).
		\item Show that if $A \in \sa$ with $\mu(A) < \infty$, then $\nu(A) = \mu(A)$. (1 pt.)
		\item Show that $\nu$ is a measure on $\sa$. (2.5 pts.)
		\item Show that if $\mu$ is $\sigma$-finite, the $\mu = \nu$. (1 pt.)
	\end{enumerate}
\end{ex}

\begin{proof}
	$ $\newline
	\begin{enumerate}
		\item We first show that if $A_1 \subseteq A_2$, then $S(A_1) \subseteq A_2$. Let $C \in S(A_1)$, then $C \in \sa \land C \subset A_1 \land \mu(C) < \infty$. Clearly $C \subseteq A_2$ since $A_1 \subseteq A_2$ so $C \in S(A_2)$. Finally,
		
		\begin{align*}
			\nu(A_1) = \sup \{\mu(B) \mid B \in S(A_1)\} \leq \sup \{\mu(B) \mid B \in S(A_2)\} = \nu(A_2).
		\end{align*}
		
		\item Clearly the domain and range of $\nu$ are compatible with the definition of a measure, i.e. $\nu: \sa \to [0, \infty]$. We need to show that the two properties from \autoref{dfn:measure} hold.
		\begin{enumerate}
			\item $\nu(\emptyset) = \sup \{\nu(B) \mid B \in S(\emptyset)\} = \mu(\emptyset) = 0$ since $S(\emptyset) = \{ \emptyset\}$.
			\item Let $(A_n)_{n\in\N} \subset \sa$ be a pairwise disjoint sequence of sets in $\sa$. Then,
			\begin{align*}
				\nu\left(\bigcupdot_{n \in \N} A_n\right) = 
			\end{align*}
		\end{enumerate}
	\end{enumerate}
\end{proof}


\section{Exercise set 3}

Due October 4th, 2019.

\begin{ex}
	Let $(X, \sa, \mu)$ be a measure space and $\calG=\{A_1, A_2, \dots\}$ a countable partition of $X$ with $A_k \in \sa,\ \forall k \in \N$. Define a function $u : X \to \R$ by $u(x) = \sum_{k = 1}^{\infty} k \cdot \ind_{A_k}$.
	
	\begin{enumerate}
		\item Show that $u$ is $\sa/\borel(\R)$-measurable. (1.5 pts)
		\item Show that $\sigma(\mu) = \sigma(\calG)$, where $\sigma(\mu)$ is the smallest \siga making $u$ measurable. (2.5 pts)
		\item Suppose that $0 < \mu(A_n) < \infty$ for all $n \in \N$. Define $\nu$ on $\sa$ by
		\begin{align*}
		\nu(B) = \sum_{n = 1}^{\infty} 3^{-n} \cdot \frac{\mu(B \cap A_n)}{\mu (A_n)}.
		\end{align*}
		Show that $\nu$ is a \textbf{finite} measure on $(X, \sa)$. (1.5 pts)
		
		\item Under the assumptions of part 3, prove that if $B \in \sa$, then $\mu(B) = 0$ if and only if $\nu(B) = 0$. (1 pt)
	\end{enumerate}
\end{ex}

\begin{proof}$ $\newline
	\begin{enumerate}
		\item Observe that the function $u(x)$ always evaluates to a natural number. In particular, $u(x) = k$ for the $k \in \N$ that satisfies $x \in A_k$. $u$ is well-defined since $\calG$ is a countable partition of $X$.
		
		We must check whether for any $B \in \borel(\R), \inv{u}(B) \in \sa$. Given a $B \in \borel(\R)$ we rewrite it as $B = B' \cup B_{nat}$ where $B_{nat}$ contains all the naturals in $B$ and $B' = B \setminus B_{nat}$. Then,
		\begin{align*}
		\inv{u}(B) = \inv{u}(B' \cup B_{nat}) = \inv{u}(B') \cup \inv{u}(B_{nat}) = \emptyset \cup \bigcup_{n \in B_{nat}} A_n \in \sa.
		\end{align*}
		
		\item By definition of $\sigma(u)$ we have
		\begin{align*}
		\sigma(u) := \sigma (\inv{u}(\borel(\R))) = \inv{u}(\borel(\R)),
		\end{align*}
		since the preimage of any \siga is a \siga.
		
		Also, recall from exercise set 1 that
		\begin{align*}
		\sigma(\calG) = \left\{ \bigcup_{i \in I} A_i \mid A_i \in \calG,\ I \subset \N \right\}.
		\end{align*}
		
		Now we prove the double containment. Take any $A \in \sigma(u)$. Then there is a $B \in \borel(\R)$ such that $A = \inv{u}(B)$. By part one we already now that
		\begin{align*}
		A = \emptyset \cup \bigcupdot_{n \in B_{nat}} A_n \in \sigma(\calG).
		\end{align*}
		
		For the reverse containment, let $G \in \sigma(\calG)$, therefore there is a set $I \subset \N$ such that $G = \bigcup_{i \in I} A_i$. Furthermore, $I \in \borel(\R)$ and thus, by part one,
		\begin{align*}
		\bigcup_{i \in I} A_i = \inv{u}(I) \in \sigma(u).
		\end{align*}
		
		\item First let us check that $\nu$ is well defined. $\nu(B)$ is well defined for any $B \in \sa$ since $B \cap A_n \in \sa$ as $\sa$ is \istable. Also, $\nu$ is non-negative since it is computed as a sum of products of non-negative numbers.
		
		Now we check the two properties in the definition of measure:
		\begin{enumerate}
			\item
			\begin{align*}
			\nu(\emptyset) = \sum_{n=1}^{\infty} 3^{-n} \cdot \frac{\mu(\emptyset \cap A_n)}{\mu(A_n)} = \sum_{n=1}^{\infty} 3^{-n} \cdot \frac{\mu(\emptyset)}{\mu(A_n)} = \sum_{n=1}^{\infty} 3^{-n} \cdot \frac{0}{\mu(A_n)} = 0
			\end{align*}
			\item For any pairwise disjoint collection $(B_n)_{n\in \N} \subset \sa$ we have
			\begin{align*}
			\nu\left(\bigcupdot_{j\in \N} B_j\right) &= \sum_{n = 1}^\infty 3^{-n} \cdot \frac{\mu\left(\left(\bigcupdot_{j\in \N} B_j\right) \cap A_n\right)}{\mu(A_n)}\\
			&= \sum_{n = 1}^\infty 3^{-n} \cdot \frac{\mu\left(\bigcupdot_{j\in \N} (B_j \cap A_n)\right)}{\mu(A_n)} \\
			&= \sum_{n = 1}^\infty 3^{-n} \cdot \frac{\sum_{j \in \N}\mu\left(B_j \cap A_n\right)}{\mu(A_n)} \\
			&= \sum_{j\in\N} \sum_{n=1}^{\infty} 3^{-n}\cdot \frac{\mu\left(B_j \cap A_n\right)}{\mu(A_n)} = \sum_{j \in \N} \nu(B_j)
			\end{align*}
		\end{enumerate}
		
		Additionally, we must check that $\nu$ is finite. Since $\mu$ is a measure, it is monotone so $\mu(B \cap A_n) \leq \mu(A_n)$ and thus
		\begin{align*}
		\nu(B) = \sum_{n = 1}^{\infty} 3^{-n} \cdot \frac{\mu(B \cap A_n)}{\mu (A_n)} \leq \sum_{n = 1}^{\infty} 3^{-n} < \infty
		\end{align*}
		
		
		\item It is clear that if $\mu(B) = 0$ then, by monotonicity, since $B \cap A_n \subset B$, we have that $0 \leq \mu(B \cap A_n) \leq \mu(B) = 0$ and therefore $\nu(B) = \sum_{n = 1}^\infty 3^{-n} \cdot 0 = 0$.
		
		For the reverse, we have that if $\nu(B) = 0$, then it must be because $\mu(B \cap A_n) = 0,\ \forall n \in \N$ since $3^{-n} > 0,\ \forall n \in \N$. We can rewrite $\mu(B)$ as
		\begin{align*}
		\mu(B) = \mu(B \cap X) = \mu\left(B \cap \bigcupdot_{j\in \N} A_j\right) = \mu\left(\bigcupdot_{j\in \N} B \cap A_j\right) \\
		= \sum_{j \in \N} \mu(B \cap A_j) = \sum_{j \in \N} 0 = 0.
		\end{align*}
		Therefore $\mu(B) = 0 \iff \nu(B) = 0,\ \forall B \in \sa$.
	\end{enumerate}
\end{proof}

\begin{ex}
	Consider the measure space $([0, 1], \borel, \lambda)$ where $\borel = \borel(\R) \cap [0, 1)$, i.e. the restriction of the Borel \siga to the interval $[0, 1)$, and $\lambda$ denotes the Lebesgue measure restricted to $\borel$. Define a map $T:[0, 1) \to [0, 1)$ by
	\begin{align*}
	T(x) = \begin{cases}
	4x \tif 0 \leq x \leq \frac{1}{4},\\
	\frac{4}{3}( x - \frac{1}{4}) \tif \frac{1}{4} \leq x < 1.
	\end{cases}
	\end{align*}
	
	\begin{enumerate}
		\item Show that $T$ is $\borel/\borel$-measurable (in short, Borel measurable). (1.5 pts)
		\item Consider the image measure $T(\lambda)$ defined by $T(\lambda)(B) = \lambda(\inv{T}(B))$, for all $B \in \borel$. Show that $T(\lambda) = \lambda$. (2 pts)
	\end{enumerate}
\end{ex}

\begin{proof}$ $\newline
	\begin{enumerate}
		\item It is enough to check if $\inv{T}([a,b)) \in \borel$, since $\calJ \cap [0, 1) = \{[a, b) \mid 0 \leq a \leq b \leq 1\}$ is a generator of $\borel$.
		
		\begin{align*}
		\inv{T}([a,b)) &= \{x \in [0, 1) \mid T(x) \in [a, b)\} \\
		&= \left\{x \in [0, \frac{1}{4}) \mid T(x) \in [a, b)\right\} \cup \left\{x \in [\frac{1}{4}, 1) \mid T(x) \in [a, b)\right\} \\
		&= \left\{x \mid 0 \leq a \leq x < b \leq \frac{1}{4}\right\} \cup \left\{x \mid \frac{1}{4} \leq a \leq \frac{4}{3}( x - \frac{1}{4}) < b \leq 1\right\} \\
		&= \left[\frac{a}{4}, \frac{b}{4}\right) \cup \left[\frac{3a}{4} + \frac{1}{4}, \frac{3b}{4} + \frac{1}{4}\right) \in \borel,
		\end{align*}
		since each of the intervals is itself in $\calJ \cap [0,1)$. (This is because $0 \leq \frac{a}{4}, \frac{b}{4}, \frac{3a}{4} + \frac{1}{4}, \frac{3b}{4} + \frac{1}{4} < 1$ since $a, b \in [0, 1)$.)
		
		\item First we will prove that $T(\lambda)([a, b)) = \lambda([a, b)), \forall [a, b) \in \calJ \cap [0,1)$. Using the same as in part one we have
		\begin{align*}
		T(\lambda)([a, b)) &= \lambda(\inv{T}[a,b))\\
		&= \lambda\left(\left[\frac{a}{4}, \frac{b}{4}\right) \cupdot \left[\frac{3a}{4} + \frac{1}{4}, \frac{3b}{4} + \frac{1}{4}\right)\right) \\
		&= \frac{b}{4} - \frac{a}{4} + \frac{3b}{4} + \frac{1}{4} - \frac{3a}{4} - \frac{1}{4} \\
		&= b - a = \lambda([a, b)).
		\end{align*} 
		Recall that $\borel = \sigma(\calJ \cap [0,1))$ and that there exists an exhausting sequence $B_n \uparrow [0,1)$ where $B_n = [0,1),\ \forall n \in \N$ and that $\lambda([0, 1)) < \infty$ and $T(\lambda)([0,1)) = \lambda(\inv{T}([0,1))) = \lambda([0,1)) < \infty$. Then, by uniqueness, we have that $T(\lambda) = \lambda$ on all $\borel$.
	\end{enumerate}
\end{proof}

\section{Exercise set 4}

Due October 25th, 2019.

\begin{ex}
	Let $(X, \sa, \mu)$ be a measure space and $u \in \calL^1_\Rb(\mu)$. Define $B_n = \{ x \in X \mid 2^{-n} \leq \abs{u(x)} < 2^n\}$, for $n \geq 1$. Set $B = \bigcup_{n = 1}^\infty B_n$.
	
	\begin{enumerate}
		\item Show that $\int_X u d\mu = \int_B u d\mu$. (1.5 pts)
		\item Prove that $\lim_{n \to \infty} \int_{B_n} u d\mu = \int_X u d\mu$. (1.5 pts)
		\item Show that for every $\varepsilon > 0$, there exists a positive integer $N$, such that $\mu(B_N) < \infty$ and $\abs{\int_{B_N^\complement} ud\mu} < \varepsilon$. (1.5 pts)
	\end{enumerate}
\end{ex}

\begin{proof}$ $\newline
	\begin{enumerate}
		\item We have that $u: X \to \Rb$ so for every $n \in \N,\ B_n \subset X$. Therefore, $B = \bigcup_{n=1}^\infty B_n \subset X$ and we may write
		\begin{align*}
		\int_X u d\mu
		= \int_B u d\mu + \int_{B^\complement} u d\mu.
		\end{align*}
		
		We will show that $\int_{B^\complement} u d\mu = 0$. We have that
		\begin{align*}
		B^\complement = \left( \bigcup_{n=1}^\infty B_n \right)^\complement = \bigcap_{n=1}^\infty B_n^\complement.
		\end{align*}
		For each $n \in \N$, $B_n^\complement = \{\abs{u} < 2^{-n}\} \cupdot \{\abs{u} \geq 2^n\}$ and $B_1^\complement \supset B_2^\complement \supset \dots$, so $(B_n^\complement)$ is a decreasing sequence with $B_n^\complement \downarrow B^\complement$. We can further split $\int_{B^\complement} u d\mu$ as
		\begin{align*}
		\int_{B^\complement} ud\mu
		&= \lim_{n\to\infty} \left(\int_{\{\abs{u} < 2^{-n}\}} ud\mu + \int_{\{\abs{u} \geq 2^n\}} u d\mu\right) \\
		&= \lim_{n\to\infty} \int_{\{\abs{u} < 2^{-n}\}} ud\mu + \lim_{n\to\infty} \int_{\{\abs{u} \geq 2^n\}} u d\mu.
		\end{align*}
		
		For the first integral we have that
		\begin{align*}
		\lim_{n\to\infty} \int_{\{\abs{u} < 2^{-n}\}} ud\mu
		\leq \lim_{n\to\infty} \int_{\{\abs{u} < 2^{-n}\}} 2^{-n} d\mu
		\leq \lim_{n\to\infty} \int 2^{-n} d\mu.
		\end{align*}
		
		Since $2^{-n} \geq 2^{-(n+1)} \geq \dots \geq 0$ is a decreasing sequence of $\mu$-integrable functions, the Monotone Convergence Theorem applies and
		\begin{align*}
		\lim_{n\to\infty} \int_{\{\abs{u} < 2^{-n}\}} ud\mu
		\leq \lim_{n\to\infty} \int 2^{-n} d\mu 
		= \int \inf_{n\in\N} 2^{-n} d\mu
		= \int 0 d\mu = 0.
		\end{align*}
		
		For the second integral, we use Markov's inequality to get
		\begin{align*}
		\mu(\{\abs{u} \geq 2^n\}) \leq \frac{1}{2^n} \int u d\mu.
		\end{align*}
		Since $u \in \calL^1_\Rb(\mu)$, we have that $\int{u} d\mu$ is a finite number so we can take the limit to get
		\begin{align*}
		\lim_{n\to\infty} \mu(\{\abs{u} \geq 2^n\}) \leq \lim_{n\to\infty} \frac{1}{2^n} \int u d\mu = \lim_{n\to\infty} \frac{1}{2^n} = 0,
		\end{align*}
		so $\{\abs{u} \geq 2^n\}$ is a $\mu$-null set and $\int_{\{\abs{u} \geq 2^n\}} u d\mu = 0$. We conclude that $\int_{B^\complement} u d\mu = 0$ and therefore $\int_X u d\mu = \int_B u d\mu$.
		
		
		\item Define $(u_n)_{n\in\N}$ by $u_n = u \ind_{B_n}$. Clearly, $u_n \uparrow u \ind_B$ and $u_n \leq u_{n+1}$. So the Monotone Convergence Theorem applies and
		\begin{align*}
		\lim_{n \to \infty} \int_{B_n} u d\mu
		&= \lim_{n \to \infty} \int u \ind_{B_n} d \mu \\
		&= \lim_{n \to \infty} \int u_n d\mu\\
		&= \int \lim_{n \to \infty} u_n d\mu\\
		&= \int u \ind_B d\mu\\
		&= \int_B ud\mu = \int_X u d\mu.
		\end{align*}
		
		\item As before, we write,
		\begin{align*}
		\int_{B_N^\complement} u d\mu
		= \int_X u d\mu - \int_{B_N} u d\mu.
		\end{align*}	
		By the definition of limit and part 2 we know that for every $\varepsilon > 0$, there exists an $N \in \N$ such that
		\begin{align*}
		\abs{\int_{B_N} u d\mu - \int_X u d\mu} = \abs{ \int_X u d\mu - \int_{B_N} u d\mu} < \varepsilon.
		\end{align*}
		Therefore,
		\begin{align*}
		\abs{\int_{B_N^\complement} u d\mu} = \abs{\int_X u d\mu - \int_{B_N} u d\mu} < \varepsilon.
		\end{align*}
		We just need to check that $\mu(B_N) < \infty$. For that we use the fact that $B_N \subset \{\abs{u} \leq 2^N\}$ and a version of Markov's inequality:
		\begin{align*}
		\mu(B_N)
		\leq \mu(\{\abs{u} \leq 2^N \})
		\leq \frac{1}{2^N} \int \abs{u} d\mu < \infty,
		\end{align*}
		since $u \in \calL_\Rb^1(\mu) \implies \abs{u} \in \calL_\Rb^1(\mu)$.
	\end{enumerate}
\end{proof}

\begin{ex}
	Consider the measure space $([0,1], \borel([0, 1]), \lambda)$ where $\borel([0,1])$ is the restriction of the Borel \siga to $[0,1]$, and $\lambda$ is the restriction of the one-dimensional Lebesgue measure to $[0,1]$.
	
	\begin{enumerate}
		\item Show that $\lim_{n \to \infty} \int_{[0,1]} \frac{x^n}{(1+ x)^2}d\lambda(x) = 0$. (1.5 pts)
		\item Show that $\lim_{n \to \infty} \int_{[0,1]} \frac{n x^{n-1}}{1 + x} d\lambda(x) = \frac{1}{2}$. (1 pt)
	\end{enumerate}
\end{ex}

\begin{proof}$ $\newline
	\begin{enumerate}
		\item Let $u_n = \frac{x^n}{(1+ x)^2}$. We have that, for every $n \in \N$ and every $x \in [0,1]$,
		\begin{align*}
		x^n \leq 1 \leq (1 + x)^2 \implies \abs{u_n} \leq 1, \forall x \in [0,1],\ n \in \N.
		\end{align*}
		We can rewrite $u_n$ as
		\begin{align*}
		u_n(x) = \frac{x^n}{(1+ x)^2}\ind_{[0,1)} + \frac{x^n}{(1+ x)^2}\ind_{\{1\}} = \frac{x^n}{(1+ x)^2}\ind_{[0,1)} + \frac{1}{4}\ind_{\{1\}},
		\end{align*}
		and therefore
		\begin{align*}
		\lim_{n\to\infty} u_n(x)
		&= \lim_{n\to\infty}\left( \frac{x^n}{(1+ x)^2}\ind_{[0,1)}(x) + \frac{1}{4}\ind_{\{1\}}(x) \right) \\
		&= 0\cdot\ind_{[0,1)}(x) + \frac{1}{4}\cdot\ind_{\{1\}}(x) \\
		&= \frac{1}{4}\ind_{\{1\}}(x),\quad \forall x \in [0,1].
		\end{align*}
		
		Therefore $(u_n)_{n\in\N}$ is a sequence of $\mu$-integrable functions bounded by $1$ and we can apply the Lebesgue Dominated Convergence theorem to obtain
		\begin{align*}
		\lim_{n\to\infty} \int_{[0,1]} u_n d\lambda(x)
		&= \int_{[0,1]} \lim_{n \to \infty} u_n d\lambda(x)\\
		&= \int \frac{1}{4}\ind_{\{1\}}(x) d\lambda(x)\\
		&= \int_{\{1\}} \frac{1}{4}d\lambda(x)
		= 0,
		\end{align*}
		since $\lambda(\{1\}) = 0$.
		
		\item
	\end{enumerate}
\end{proof}

\begin{ex}
	Let $(X, \sa, \mu)$ be a measure space, and $u \in \calL^p(\mu)$ for some $p \in [1, \infty)$. For $n \geq 1$, define $u_n = \min\{ \max(u, -n), n \}$.
	
	\begin{enumerate}
		\item Prove that $\lim_{n\to\infty} \norm{u_n}_p = \norm{u}_p$. (2 pts)
		\item Prove that for any $\varepsilon$, there exists an integer $n \geq 1$ such that $\int \abs{u - u_n}^p \leq \varepsilon$. (1 pt)
	\end{enumerate}
\end{ex}

\begin{proof}$ $\newline
	\begin{enumerate}
		\item Observe that we may rewrite $u_n$ as
		\begin{align*}
			u_n(x) = \begin{cases}
			-n \tif u(x) < n\\
			u(x) \tif -n \leq u(x) \leq n \\
			n \tif n < u(x)
			\end{cases}
		\end{align*}
		and therefore
		\begin{align*}
			\abs{u_n(x)} = \begin{cases}
			u(x) \tif 0 \leq \abs{u(x)} \leq n\\
			n \tif n < \abs{u(x)}
			\end{cases}.
		\end{align*}
		From this we can see that $\abs{u_n} \leq \abs{u_{n+1}} \leq \abs{u}$ and thus $(\abs{u_n})_{n\in\N}$ is an increasing sequence of non-negative functions which converges to $\abs{u}$. Therefore we can use Beppo-Lévi to get
		\begin{align*}
			\lim_{n\to\infty} \norm{u_n}_p
			&= \lim_{n\to\infty} \left( \int \abs{u_n}^pd\mu \right)^{\frac{1}{p}}\\
			&= \left(\lim_{n\to\infty} \int \abs{u_n}^pd\mu \right)^{\frac{1}{p}}\\
			&\overset{\ref{thm:beppo-levi}}{=} \int \left(\lim_{n\to\infty} \abs{u_n}^pd\mu \right)^{\frac{1}{p}}\\
			&= \int \left(\abs{u}^pd\mu \right)^{\frac{1}{p}}
			= \norm{u}_p
		\end{align*}
		\item From part one we know that $\lim_{n\to\infty} \norm{u_n}_p = \norm{u}_p$. Therefore, by Riesz's theorem, we have that $\lim_{n\to\infty}\norm{u_n - u}_p = 0$. From the definition of limit we get that for any $\varepsilon' > 0$, there exists an $N \in \N$ such that
		\begin{align*}
		\left( \int \abs{u_N - u}^p d\mu \right)^{\frac{1}{p}} < \varepsilon'.
		\end{align*}
		Raising everything to the power of $p$ we get
		\begin{align*}
		\int \abs{u_N - u}^p d\mu = \int \abs{u - u_N}^p d\mu < (\varepsilon')^p.
		\end{align*}
		
		Choose $\varepsilon' = \varepsilon^{\frac{1}{p}}$ and the proof follows.
	\end{enumerate}
\end{proof}


\section{Practice mid-term 2019-2020}

\begin{ex}
	Let $(X, \sa)$ be a measure space such that $\sa = \sigma(\calG)$ where $\calG$ is a collection of subsets of $X$ such that $\emptyset \in \calG$. Show that for any $A \in \sa$ there exists a countable collection $\calG_A \subseteq \calG$ such that $A \in \sigma(\calG_A)$.
\end{ex}

\begin{proof}
	This exercise presents a good opportunity to apply the good set principle (cf. \autoref{rem:good-set-principle}).
	
	Let
	\begin{align*}
		\calB := \{ A \in \sa \mid \exists \calG_A \subset \calG \text{ countable, } A \in \sigma(\calG_A)\}.
	\end{align*}
	
	First, we will show that $\calB$ is a \siga.
	\begin{enumerate}
		\item Clearly $X \in \sa$. Let $G_X = \{\emptyset\}$ which is clearly finite and $G_X \subset \calG$ by hypothesis. Then $\sigma(\calG_X) = \{\emptyset, \emptyset^\complement\} = \{\emptyset, X\} \implies X \in \calB$.
		\item Let $A \in \calB$. Then $A \in \sa$ by definition of $\calB$ and there exists $\calG_A \subset \calG$ countable with $A \in \sigma(\calG_A)$. Since $\sigma(\calG_A)$ is a \siga it contains $A^\complement$ and clearly $A^\complement \in \sa$. Let $\calG_{A^\complement} = \calG_A$ and therefore $A^\complement \in \calB$.
		\item Finally, let $(A_n)_{n\in\N} \subset \calB$. Clearly, $(A_n)_{n\in\N} \subset \sa$ and, for each $n \in \N$, there exists $\calG_{A_n} \subset \calG$ countable such that $A_n \in \sigma(\calG_{A_n})$. Also, since $\sa$ is a \siga, we have that $\bigcup_{n\in\N}A_n \in \sa$. Let
		\begin{align*}
			\calG_{\bigcup_{n\in\N} A_n} = \bigcup_{n\in\N} \calG_{A_n}.
		\end{align*}
		$\calG_{\bigcup_{n\in\N} A_n} \subset \calG$ is countable since every $\calG_{A_n} \subset \calG$ is countable and $\calG_{\bigcup_{n\in\N} A_n}$ is a countable union of countable sets. Therefore $\bigcup_{n\in\N} A_n \in \calB$.
	\end{enumerate}

	Now we show that $\calB = \sa$. Clearly, from the definition of $\calB$ we see that $\calB \subseteq \sa$. For the reverse containment, we will show that $\calG \subset \calB$ and therefore $\sa = \sigma(\calG) \subseteq \sigma(\calB) = \calB$. Let $A \in \calG \subset \sa$. Define $\calG_A = \{A\} \subset \calG$ and clearly finite and $A \in \sigma(\calG_A)$. Therefore $A \in \calB,\ \forall A \in \calG \implies \calG\subset \calB$.
\end{proof}

\begin{ex}
	Let $X$ be a set and $\calF$ a collection of real-valued functions on $X$ satisfying the following properties:
	\begin{enumerate}
		\item $\calF$ contains the constant functions,
		\item if $f, g \in \calF$ and $c \in \R$, then $f + g,\ fg, cf \in \calF$.
		\item if $f_n \in \calF$, and $f = \lim_{n \to \infty} f_n$, then $f \in \calF$.
	\end{enumerate}

	For $A \subseteq X$, denote by $\ind_{A}$ the indicator function of $A$, i.e.
	
	\begin{align*}
		\ind_A(x) = \begin{cases}
		1 \tif x \in A,\\
		0 \tif x \not\in A.
		\end{cases}
	\end{align*}
	Show that the collection $\sa = \{ A \subseteq X \mid \ind_A \in \calF\}$ is a \siga.
\end{ex}

\begin{proof}
	We shall prove that the three properties of a \siga hold in $\sa$:
	\begin{enumerate}
		\item $X \in \sa$ since $\ind_X(x) = 1$ constantly, and therefore $\ind_X \in \calF$.
		\item if $A \in \sa$, then $\ind_A \in \calF$. We can write $\ind_{A^\complement} = 1 - \ind_A$. Since $1, 1_A \in \calF$ we have $\ind_{A^\complement} \in \calF \implies A^\complement \in \sa$.
		\item Let $(A_n)_{n\in \N} \subset \sa$. Then $(\ind_{A_n})_{n\in\N} \in \calF$. We can write
		\begin{align*}
			\ind_{\bigcup_{n\in\N} A_n} = \sum_{n=1}^{\infty} \ind_{A_i}
		\end{align*}
		To prove $\ind_{\bigcup_{n\in\N} A_i} \in \calF$ we construct the following sequence of functions $(f_n)_{n\in\N} \in \calF$ given by
		\begin{align*}
			f_n = \sum_{i=1}^{n} \ind_{A_i} = \ind_{A_n} + f_{n-1}
		\end{align*}
		Clearly, $f_n \in \calF$ for all $n$ because of property 2. Then,
		\begin{align*}
			f = \lim_{n \to \infty} f_n = \sum_{i = 1}^\infty \ind_{A_i} = \ind_{\bigcup_{n\in\N} A_n} \in \calF,
		\end{align*}
		and thus, $\bigcup_{n\in\N} A_i \in \sa$.
	\end{enumerate}
\end{proof}

\begin{ex}
	Consider the measure space $([0,1], \borel([0,1]), \lambda)$ where $\borel([0,1])$ is the restriction of the Borel \siga to to $[0,1]$, and $\lambda$ is the restriction of the Lebesgue measure to $[0,1]$. Let $E_1, \dots, E_m$ be a collection of Borel measurable subsets of $[0,1]$ such that every element $x \in [0,1]$ belongs to at least $n$ sets in the collection $\{E_j\}_{j=1}^m$, where $n \leq m$. Show that there exists a $j \in \{1, \dots, m\}$ such that $\lambda(E_j) \geq \frac{n}{m}$.
\end{ex}

\begin{proof}
	Observe that if every $x \in [0, 1]$ belongs to at least $n$ sets in $(E_j)$ then we have that
	\begin{align*}
		\sum_{j = 1}^m \ind_{E_j}(x) \geq n,\ \forall x \in [0,1].
	\end{align*}
	We proceed by contradiction. Suppose $\lambda(E_j) < \frac{n}{m},\ \forall j \in \{1, \dots, m\}$. Then
	\begin{align*}
		n = \int_{[0,1]}n d\lambda
		\overset{\ref{rem:mu-integrals-monotone}}{\leq} \int_{[0,1]} \sum_{j=1}^{m}\ind_{E_j}(x)d\lambda
		= \sum_{j = 1}^m \lambda(E_j) < m \cdot \frac{n}{m} = n. 
	\end{align*}
	This is a contradiction, so there must exists at least one $j \in \{1, \dots, m\}$ such that $\lambda(E_j) \geq \frac{n}{m}$.
\end{proof}

\begin{ex}
	Consider the measure space $(\R, \borel(\R), \lambda)$ where $\borel(\R)$ is the Borel \siga over $\R$, and $\lambda$ is the one-dimensional Lebesgue measure. Let $f_n : \R \to \R$ be defined by
	\begin{align*}
		f_n(x) = \sum_{k = 0}^{2^n - 1}\frac{3k + 2^n}{2^n} \ind_{[k/2^n, (k+1)/2^n]}(x),\ n \geq 1.
	\end{align*}
	\begin{enumerate}
		\item Show that $f_n$ is measurable, and $f_n(x) \leq f_{n+1}(x)$, for all $x \in \R$.
		\item Show that $\int \sup_{n \geq 1} f_n d\lambda = \frac{5}{2}$.
	\end{enumerate}
\end{ex}

\begin{proof}$ $\newline
	\begin{enumerate}
		\item For a given $n$ and $k$ we define
		\begin{align*}
			a^n_k = \frac{3k + 2^n}{2^n}\text{ and } A^n_k = \left[\frac{k}{2^n}, \frac{k+1}{2^n}\right).
		\end{align*}
		Clearly, $A^n_k \in \borel(\R)$ for any $n$ and $k$ and $a^n_k > 0$. Therefore, $f_n$ is a simple function and thus $f_n$ is $\sa/\borel(\R)$-measurable.
		
		Notice that
		\begin{multline*}
			A_k^n = \left[\frac{k}{2^n}, \frac{k+1}{2^n}\right)
			= \left[\frac{2k}{2\cdot 2^n}, \frac{2k+2}{2\cdot 2^n}\right)\\
			= \left[\frac{2k}{2^{n+1}}, \frac{2k+1}{2^{n+1}}\right) \cupdot \left[\frac{2k+1}{2^{n+1}}, \frac{2k+2}{2^{n+1}}\right) = A_{2k}^{n+1} \cupdot A_{2k+1}^{n+1},
		\end{multline*}
		and
		\begin{align*}
			a_k^n = \frac{3k + 2^n}{2^n} = \frac{6k + 2^{n+1}}{2^{n+1}} = a_{2k}^{n+1} \leq a_{2k}^{n+1} + a_{2k+1}^{n+1}.
		\end{align*}
		Thus,
		\begin{align*}
			f_n = \sum_{k = 0}^{2^n - 1} a_k^n \ind_{A_k^n} \leq \sum_{k = 0}^{2^n - 1} a_{2k}^{n+1} \ind_{A_{2k}^{n+1}} + a_{2k+1}^{n+1} \ind_{A_{2k+1}^{n+1}} = f_{n+1}.
		\end{align*}
		
		\item Since $(f_n)$ is an increasing sequence of simple functions we now that
		\begin{align*}
			\int \sup_{n \geq 1} f_n d\mu = \sup_{n \geq 1} \int f_n d\mu = \sup_{n \geq 1} I_\mu(f_n) = \lim_{n \to \infty} I_\mu(f_n),
		\end{align*}
		and
		\begin{align*}
			I_\mu(f_n) = \sum_{k=0}^{2^n - 1} a_k^n \mu(A_k^n)
			&= \sum_{k = 0}^{2^n - 1} \frac{3k + 2^n}{2^n} \cdot \left(\frac{k+1}{2^n} - \frac{k}{2^n}\right)\\
			&= \sum_{k = 0}^{2^n - 1} \frac{3k}{2^n \cdot 2^n} + \frac{1}{2^n} = 1 + \frac{3}{2} \frac{2^n - 1}{2^n}
		\end{align*}.
		Thus,
		\begin{align*}
			\int \sup_{n \geq 1} f_n d \mu = \lim_{n \to \infty} I_\mu(f_n) = 1 + \frac{3}{2} = \frac{5}{2}.
		\end{align*}
	\end{enumerate}
\end{proof}

\begin{ex}
	Let $\mu$ and $\nu$ be two measures on the measure space $(E, \borel)$ sucht that $\mu(A) \leq \nu(A)$ for all $A \in \borel$. Show that if $f$ is any non-negative function on $(E, \borel)$, then $\int_E f d\mu \leq \int_E f d\nu$.
\end{ex}

\begin{proof}
	\footnote{Now that the solutions to this practice midterm have been published, I noticed there is a much more cleaner (and modular!) way of doing this. First prove it for $f = \ind_A$ for some $A\in \sa$. Then, prove it for $f \in \calE^+(\sa)$, for some simple function $f$. Finally, prove it for the general case. Each of these builds on the previous and you're left with something much more readable and useful.}By \autoref{thm:sombrero} we now that there exists a sequence of non-negative simple functions $(f_n)_{n\in\N} \subset \calE^+(\borel)$ such that $f_n \uparrow f$. Moreover, each of this simple functions has a representation of the form
	\begin{align*}
		f_n = \sum_{j=0}^{M_n} a_j \ind_{A_j}
	\end{align*}
	Thus,
	\begin{multline*}
		\int_E fd\mu
		= \lim_{n \to \infty} \int_E f_n d\mu
		= \lim_{n \to \infty} I_\mu(f_n) \\
		= \lim_{n \to \infty} \sum_{i = 1}^n \sum_{j=0}^{M_n} a_j \mu(A_j)
		\leq \lim_{n \to \infty} \sum_{i = 1}^n \sum_{j = 0}^{M_n} a_j \nu(A_j) \\
		= \lim_{n \to \infty} I_\nu(f_n) 
		= \lim_{n \to \infty} \int_E f_n d\nu
		= \int fd\nu
	\end{multline*}
\end{proof}

\section{Practice final 2019-2020}

\begin{ex}
	Consider the measure space $([1, \infty), \borel([1, \infty)), \lambda)$ where $\borel([1,\infty])$ is the Borel \siga and $\lambda$ is the Lebesgue measure restricted to $[1, \infty)$. Show that
	\begin{align*}
		\lim_{n\to\infty} \int \frac{n \sin (x/n)}{x^3}d\lambda(x) = 1.
	\end{align*}
	(Hint: $\lim_{n\to\infty} \sin(x)/x = 1$).
\end{ex}

\begin{proof}
	Since it is not clear how we would go about directly evaluating the Lebesgue integral, let us check if we can use the Riemann integral to calculate the value. Let $f_n(x) = \frac{n \sin (x/n)}{x^3}$. Recall from \autoref{cor:improper-riemann} that if $\rint_1^\N \abs{f_n(x)} d(x) < \infty$ for all $N \in \N$ then $\rint_1^\infty f_n(x) dx = \int_{[1, \infty)} f_n(x) d\lambda(x)$. So first, we show that $f_n$ is Riemann integrable from $1$ to $N$ for any $N$. We have that $\sin y \leq y$ for any $y \geq 0$ so
	\begin{align*}
		\abs{f_n(x)} = \abs{\frac{n \sin (x/n)}{x^3}} \leq \frac{n \cdot (x/n)}{x^3} = \frac{1}{x^2},\ \forall x \in [1, \infty)
	\end{align*}
	and thus
	\begin{align*}
		\rint_1^N \abs{f_n(x)} dx \leq \rint_1^\infty \frac{1}{x^2} = 1 < \infty,\ \forall N \in \N.
	\end{align*}
	So far, we have proven that for each $n \in \N$, $f_n \in \calL^1$. Moreover, by using the hint we get that
	\begin{align*}
		\lim_{n\to\infty} f_n(x)
		= \lim_{n\to\infty} \frac{\sin(x/n)}{x^2 \cdot (x/n)}
		= \frac{1}{x^2}\lim_{n\to\infty} \frac{\sin(x/n)}{x/n}
		= \frac{1}{x^2}.
	\end{align*}
	
	We reuse the argument from before to show that $\forall n \in \N$, $f_n(x) \leq \frac{1}{x^2}$. We do this to get a function $w \in \calL^1$ such that $\abs{f_n} \leq w,\ \forall n \in \N$. We are now in a position to apply Lebesgue's Dominated Convergence theorem (cf. \autoref{thm:lebesgue-dominated-convergence}). With no further ado,
	\begin{align*}
		\lim_{n\to\infty} \int_{[1, \infty)} \frac{n \sin(x/n)}{x^3}d\lambda(x)
		&\overset{\ref{thm:lebesgue-dominated-convergence}}{=} \int_{[1, \infty)} \lim_{n\to\infty} \frac{n\sin(x/n)}{x^3}d\lambda(x)\\
		&= \int_{[1,\infty)} \frac{1}{x^2}d\lambda(x)\\
		&\overset{\ref{cor:improper-riemann}}{=} \rint_1^\infty \frac{1}{x^2}dx = 1.
	\end{align*}
\end{proof}

\begin{ex}
	Consider $(\R, \borel(\R), \lambda)$ where $\borel(\R)$ is the Borel \siga and $\lambda$ is the one-dimensional Lebesgue measure.
	\begin{enumerate}
		\item Prove that for $f\in \calL^1(\lambda)$ and $n\in \Z$ one has
		\begin{align*}
			\int_{[0,1]} f(x + n)d\lambda(x) = \int_{[n, n + 1]}f(x) d\lambda(x).
		\end{align*}
		\item Let $f \in \calL^1(\lambda)$ and define $g(x) = \ind_{[0, 1]} \sum_{n\in \Z} f(x + n)$. Prove that $g\in \calL^1(\lambda)$ and that
		\begin{align*}
			\int_\R g(x) d\lambda(x) = \int_\R f(x) d\lambda(x).
		\end{align*}
	\end{enumerate}
\end{ex}

This exercise is not too difficult but it is very technical and we have to be careful that we choose the correct theorem for each step and that its hypotheses are met.

\begin{proof}$ $\newline
	\begin{enumerate}
		\item At first, one might be tempted to split $f = f^+ - f^-$, approximate by increasing sequences simple functions and go all the way back to the definition of the integral for simple functions. Even though this is possible, it might as well make one go nuts with the indices. Therefore, we use a standard argument in this course: prove it first for indicator functions, then for $f \in \calE^+$, then for $f \in \calM^+$, and finally for $f \in \calL^1$.
		\begin{enumerate}
			\item Let $f = \ind_A$ where $A \in \borel(\R)$. Then,
			\begin{align*}
				\int_{[0,1]} f(x + n) d\lambda (x)
				&= \int \ind_{[0,1]}(x) \cdot \ind_A(x + n) d\lambda(x)\\
				&= \int \ind_{[0,1]}(x) \cdot \ind_{A - n}(x)d\lambda(x)\\
				&= \int \ind_{[0,1] \cap (A - n)}(x) d\lambda(x)\\
				&= \lambda([0,1] \cap (A - n)).
			\end{align*}
			We can now use that the Lebesgue measure is translation invariant to get
			\begin{align*}
				\lambda([0,1] \cap (A - n))
				= \lambda(([0,1] + n) \cap A)
				= \lambda([n, n+1], \cap A),
			\end{align*}
			and therefore
			\begin{align*}
				\int_{[0,1]}f (x + n)
				&= \lambda([n, n + 1] \cap A)\\
				&= \int_{[n, n +1]} \ind_A(x)d\lambda(x)\\
				&= \int_{[n, n+1]} f(x)d\lambda(x).
			\end{align*}
			
			\item Now let $f(x) = \sum_{j=0}^N a_j \ind_{A_j}(x) \in \calE^+(\borel(\R))$. We have
			\begin{align*}
				\int_{[0,1]} f(x + n) d\lambda(x)
				&= \int_{[0,1]} \sum_{j = 0}^N a_j \ind_{A_j}(x + n)d\lambda(x)\\
				&= \sum_{j = 0}^N a_j \int_{[0,1]} \ind_{A_j}(x + n)d\lambda(x)\\
				&= \sum_{j = 0}^N a_j \int_{[n, n+1]} \ind_{A_j}(x) d\lambda(x)\\
				&= \int_{[n, n+1]} \sum_{j = 0}^N a_j \ind_{A_j}(x) d\lambda(x)\\
				&= \int_{[n, n+1]} f(x)d\lambda(x).
			\end{align*}
			
			\item Next let $f \in \calM^+(\borel(\R))$. By the Sombrero lemma (\autoref{thm:sombrero}) we have that there exists an increasing sequence of non-negative simple functions $(f_j)_{j\in\N} \subset \calE^+$ such that $f_j \uparrow f$. Thus, using Beppo-Lévi we get
			\begin{align*}
				\int_{[0,1]} f(x+n)d\lambda(x)
				&= \int_{[0,1]} \sup_{j \in \N} f_j(x+n)d\lambda(x)\\
				&\overset{\ref{thm:beppo-levi}}{=} \sup_{j\in\N} \int_{[0,1]} f_j(x + n)d\lambda(x)\\
				&= \sup_{j\in\N} \int_{[n, n+1]} f_j(x) d\lambda(x)\\
				&\overset{\ref{thm:beppo-levi}}{=} \int_{[n, n+1]} \sup_{j \in \N} f_j(x) d\lambda(x)
				&= \int_{[n, n+1]} f(x) d\lambda(x).
			\end{align*}
			
			\item Finally let $f\in \calL^1(\lambda)$. Then $f = f^+ - f^-$ with $f^+,\ f^- \in \calM^+(\borel(\R))$ and therefore
			\begin{align*}
				\int_{[0,1]} f(x + n)d\lambda(x)
				&= \int_{[0,1]}f^+(x + n)d\lambda(x) - \int_{[0,1]}f^-(x + n)d\lambda(x)\\
				&= \int_{[n, n+1]} f^+(x)d\lambda(x) - \int_{[n, n+1]}f^-(x) d\lambda(x)\\
				&= \int_{[n, n+1]} f(x)d\lambda(x).
			\end{align*}
		\end{enumerate}
	
		\item First we prove that $g \in \calL^1(\lambda)$. To be extra precise, we can rewrite
		\begin{align*}
			g(x) = \sum_{n\in\N} g_n(x),\ g_n(x) = \ind_{[0,1]}\cdot f(x+ n) + \ind_{[0,1]} f(x - n).
		\end{align*}
		Notice how we have changed the summation index to make it easier for us to apply \autoref{cor:integral-countable-sum} and we will use it just for that. Keep in mind that this theorem only works for non-negative functions, so we will use $\abs{g}$ to check integrability:
		\begin{align*}
			\int_\R \abs{g(x)} d\lambda(x)
			&= \int_\R \abs{\sum_{n\in\N} g_n(x)}d\lambda(x)\\
			&\overset{\ref{cor:minkowski-seq}}{\leq} \int_\R \sum_{n\in\N} \abs{g_n(x)}d\lambda(x)\\
			&\overset{\ref{cor:integral-countable-sum}}{=} \sum_{n\in\N} \int_\R \abs{g_n(x)} d\lambda(x)\\
			&\overset{\ref{thm:minkowski}}{\leq} \sum_{n\in\Z} \int_\R \ind_{[0,1]}\cdot f(x + n) d\lambda(x)\\
			&= \sum_{n\in\Z} \int_{[0,1]} f(x + n)d\lambda(x)\\
			&= \sum_{n\in\Z} \int_{[n, n+1]} f(x) d\lambda(x)\\
			&= \int_\R f(x) d\lambda(x) <\infty,
		\end{align*}
		where we have used the generalised Minkowski inequality twice, first in its infinite form and second in the standard one, when changing the summation index back to $\Z$. Thus, $\abs{g} \in \calL^1(\lambda) \implies u \in \calL^1(\lambda)$.
		
		Finally we check the equality. As we have seen that $g \in \calL^1$ we now have an integrable bound for [the absolute value of] our sequence $\sum_{n\in\N} g_n(x)$ so we can apply Lebesgue Dominated Convergence to get
		\begin{align*}
			\int_\R g(x) d\lambda(x)
			&= \int_\R \sum_{n\in\N} g_n(x) d\lambda(x)\\
			&\overset{\ref{thm:lebesgue-dominated-convergence}}{=} \sum_{n\in\N} \int_\R g_n(x)d\lambda(x)\\
			&= \sum_{n\in\Z} \int_{[0,1]} f(x + n)d\lambda(x)\\
			&= \sum_{n\in\Z} \int_{[n, n+1]} f(x) d\lambda(x)\\
			&= \int_\R f(x) d\lambda(x),
		\end{align*}
		where we have used the previous part.
	\end{enumerate}
\end{proof}

\section{Mid-term 2019-2020}

The exam took place on Friday, October 11 and we had two hours to complete it.

\section{Final 2019-2020}

The exam took place on Wednesday, November 6 and we had three hours to complete it.