% !TeX root = ../mi-notes.tex

\chapter{Measurable functions}

In this chapter we restrict our attention to measurable maps whose domain is any measurable space $(X, \sa)$ but whose domain is $(\R, \borel(\R))$. To distinguish them from the general case, we shall use the same terminology as \cite{schilling2017}, and call them measurable functions.

The following result is trivial but important, as the intervals we introduce in the following lemma are easy to work with.

\begin{lem}
	\label{lem:measurable-generators-functions}
	Let $(X, \sa)$ be a measurable space, and $u: X \to \R$. Then, $u$ is $\sa/\borel(\R)$-measurable
	\begin{align*}
		\iff &\inv{u}((a, \infty)) = \{x \in X \mid u(x) > a\} = \{u > a\} \in \sa\\
		\iff &\inv{u}([a, \infty)) = \{x \in X \mid u(x) > a\} = \{u \geq a\} \in \sa\\
		\iff &\inv{u}((-\infty, a)) = \{x \in X \mid u(x) > a\} = \{u < a\} \in \sa\\
		\iff &\inv{u}((-\infty, a]) = \{x \in X \mid u(x) > a\} = \{u \leq a\} \in \sa\\
	\end{align*}
\end{lem}

Notice that we have introduced some notation here, namely, for a function $u : X \to \R$ we denote by $\{u > a\}$ the set $\{x \in X \mid u(x) > a\}$. We define analogous notations for $\geq, <, \leq, =, \neq, \in, \not\in$ and more.

\begin{proof}
	The proof follows immediately from \autoref{lem:measurable-generators}. Recall that all the families of intervals of the lemma are generators of the Borel \siga on $\R$.
\end{proof}

\section{The extended real line $\Rb$}

Throughout this chapter, we will deal with the concepts of $\lim_n, \limsup_n, \liminf_n, \sup_n$ and $\inf_n$ which will often be infinite. If we agree that $-\infty < x < \infty,\ \forall x \in \R$ it makes sense to include the values $\pm \infty$ in $\R$ to build the extended space $\Rb = [-\infty, +\infty] = \R \cup \{\pm \infty\}$. We would like $\Rb$ to inherit as much as possible from the algebraic, topological and measurable structures of $\R$.


\subsection{Extension of the algebraic structure}

We extend the algebraic structure by extending the addition and multiplication tables as follows:

\begin{center}
	\begin{tabular}{c|ccc}
		$+$ & $x\in \R$ & $+\infty$ & $-\infty$ \\ 
		\hline 
		$y\in \R$ & $x+y$ & $+\infty$ & $-\infty$ \\ 
		$+\infty$ & $+\infty$ & $+\infty$ & not defined \\ 
		$-\infty$ & $-\infty$ & not defined & $-\infty$ \\ 
	\end{tabular} 
\end{center}

\begin{center}
	\begin{tabular}{c|cccc}
		$\cdot$ & $0$ & $x \in \R\setminus\{0\}$ & $+\infty$ & $-\infty$ \\ 
		\hline 
		$0$ & $0$ & $0$ & $0$* & $0$* \\ 
		$y \in \R\setminus\{0\}$ & $0$ & $x \cdot y$ & $\sgn(y)\cdot \infty$ & $-\sgn(y)\cdot \infty$ \\ 
		$+\infty$ & $0$* & $\sgn(x)\cdot \infty$ & $+\infty$ & $-\infty$ \\ 
		$-\infty$ & $0$* & $-\sgn(x)\cdot \infty$ & $-\infty$ & $+\infty$ \\ 
	\end{tabular} 
\end{center}

Caution: here we understand $\pm$ as limits but $0$ only as bona-fide $0$ (i.e. not as a limit, which would cause convergence problems). Conventions are tricky. Expresions of the form
\begin{align*}
	\infty - \infty \text{ or } \frac{\pm \infty}{\pm \infty}
\end{align*}
should be avoided.

\subsection{Extension of the topological structure}

\begin{dfn}[Neighbourhoods in $\Rb$]
	For some $x \in \Rb$ we say that a neighbourhood of $x$ is a set of the form
	\begin{align*}
		(x - \varepsilon, x + \varepsilon) \tif x \in \R\\
		(a, +\infty] \tif x = +\infty \\
		[-\infty, a) \tif x = -\infty
	\end{align*}
	for some $a, \varepsilon \in \R$.
\end{dfn}

\begin{dfn}[Open set in $\Rb$]
	We say that a set $U \subseteq \Rb$ is open if, for every point $x \in U$ there exists a neighbourhood $B(x)$ of $x$ such that $x \in B(x) \subseteq \Rb$.
\end{dfn}

\subsection{Extension of the measurable structure}

\begin{dfn}[Borel \siga on $\Rb$]
	The Borel \siga on $\Rb$ is defined by
	\begin{align*}
		\borel(\Rb) := \{ B^\ast = B \cup S \mid B \in \borel(\R) \land S \in \calS\},
	\end{align*}
	where $\calS = \left\{\emptyset, \{+\infty\}, \{-\infty\}, \{-\infty, +\infty\}\right\}$.
\end{dfn}

The reason this extension is still called a Borel \siga is justified by the above definition of $\borel(\Rb)$, by the extension of the topological structure and by the following lemma.

\begin{lem}
	$\borel(\Rb)$ is generated by sets of the form $[a, \infty]$ (or $(a, \infty]$ or $[-\infty, a]$ or $[-\infty, a)$), where $a \in \R$ or $a \in \Q$ (and hence those intervals are subsets of $\Rb$ or $\overline{\Q}$, resp.)
\end{lem}

\begin{proof}
	It is analogous to the one given for \autoref{thm:borel-interval-generators}.
\end{proof}

\subsection{Final remarks}

\begin{dfn}[Numerical function]
	A function $u : X \to \Rb$ that takes values on $\Rb$ is called a numerical function.
\end{dfn}

\begin{dfn}[Set of measurable functions]
	Let $(X, \sa)$ be a measurable space. We write
	\begin{align*}
		\calM &:= \calM(\sa) := \{u : X \to \R \mid u \text{ is } \sa/\borel(\R)\text{-measurable }\},\\
		\calM_{\Rb} &:= \calM_{\Rb}(\sa) := \{u : X \to \Rb \mid u \text{ is } \sa/\borel(\Rb)\text{-measurable }\}
	\end{align*}
	for the families of real-values and numerical-valued measurable functions on $X$.
\end{dfn}

\section{Simple functions}

Now we will see some important (yet simple) examples. Throughout this section, $(X, \sa)$ will be a measurable space.

\begin{eg}[Indicator functions]
	Let $A \in \sa$ and define $\ind_A : X \to \R$ by
	\begin{align*}
		\ind_A(x) = \begin{cases}
		0 \tif x \in A\\
		1 \tif x \not\in A
		\end{cases}.
	\end{align*}
	Then $\ind_A$ is $\sa/\borel(\R)$-measurable since
	\begin{align*}
		\inv{\ind_A}((-\infty, a)) = \{\ind_A < a\} = \begin{cases}
		\emptyset \tif a \leq 0\\
		A^\complement \tif 0 < a \leq 1\\
		\R \tif a > 1
		\end{cases} \in \sa
	\end{align*}
	
	In fact, from the proof we can see that $\ind_A$ is $\sa/\borel(\R)$-measurable if and only if $A \in \sa$. So measurability of $\ind_A$ as a \textbf{function} is equivalent to the measurability of $A$ as a \textbf{set} (recall that a set is measurable if it is part of the $\sa$ which is the domain of the measure at hand).
\end{eg}

The following example is so important that we will give it as a definition.

\begin{dfn}[Simple function]
	Let $A_1, \dots, A_M \in \sa$ be pairwise disjoint subsets of $X$ and $y_1, \dots, y_M \in\R$. A function $f: X \to \R$ of the form
	\begin{align*}
		f(x) = \sum_{i = 1}^M y_i \ind_{A_i}
	\end{align*}
	is called a simple function.
\end{dfn}

Note that
\begin{align*}
	f(x) = \begin{cases}
	y_i \tif x \in A_i \\
	0 \tif x \not\in \bigcupdot_{i = 1}^M A_i
	\end{cases}
\end{align*}
is an alternative definition. The $i$ that makes $x \in A_i$ hold is unique or non-existent since, $(A_i)$ are pairwise disjoint, but do not necessarily cover the whole $X$.

This small inconvenience (of the possibility that $i$ does not exist) is easily fixed by extending the collection of sets to be a partition by defining
\begin{align*}
	A_0 = X \setminus \bigcupdot_{i = 1}^M A_i \text{ and } y_0 = 0.
\end{align*}

Then one can write $f(x) = \sum_{i = 0}^M y_i \ind_{A_i}$. This is called the \textbf{standard representation} of the simple function $f$.

Notice that a simple function takes only finitely many values. The converse is also true.

\begin{lem}
	Any measurable function $g$ which takes only finitely many values $\{y_0, y_1, \dots, y_M\}$ can be expressed as a linear combination of simple functions.
\end{lem}

\begin{proof}
	Define $A_i := \{g = y_i\} = \{x \in X \mid g(x) = y_i\} = \inv{g}(\{y_i\})$. $A_i \in \sa, \forall i = 0, \dots, M$ since $\{y_i\} = (-\infty, y_i] \setminus (-\infty, y_i) \in \borel(\R)$ and $g$ is assumed to be measurable.
	
	Since $y_0, \dots y_M$ are all distinct, then $A_i \cap A_j = \emptyset$ if $i \neq j$. Furthermore, $g$ takes the value $y_i$ on $A_i$, and thus
	\begin{align*}
		g = \sum_{i = 0}^M y_i \ind_{A_i},
	\end{align*}
	which is in fact the standard representation of $g$.
\end{proof}

In general, the representation of a simple function as a linear combination of simple functions is not unique. Consider the function
\begin{align*}
	f(x) = \ind_{[0, 1]}(x) + \ind_{[0, \frac{2}{3}]}(x) + \ind_{[0, \frac{1}{3}]}(x) = \begin{cases}
	0 \tif x \notin [0, 1] \\
	3 \tif x \in [0, \frac{1}{3}] \\
	2 \tif x \in (\frac{1}{3}, \frac{2}{3}] \\
	1 \tif x \in (\frac{2}{3}, 1]
	\end{cases}
\end{align*}

which can also be given by its standard representation
\begin{align*}
	f(x) = 0 \cdot \ind_{\R\setminus [0, 1]} + 1 \cdot \ind_{(\frac{2}{3}, 1]} + 2 \cdot \ind_{(\frac{1}{3}, \frac{2}{3}]} + 3\cdot\ind_{[0, \frac{1}{3}]}
\end{align*}

Simple functions are the building blocks of all measurable functions (in the sense that any measurable function is the limit of a sequence of simple functions).

To end, we will honour simple functions with their family's own definition.

\begin{dfn}[Family of simple functions]
	We denote by $\calE(\sa) = \calE \subseteq \calM$ the family of all simple functions $f:(X,\sa) \to (X, \borel(\R))$.
\end{dfn}

\subsection{Properties of simple functions}

\begin{thm}[Properties of simple functions]
	\label{thm:prop-simple-functions}
	Let $f, g \in \calE(\sa)$. Then the following hold:
	\begin{enumerate}
		\item $f\pm g \in \calE(\sa)$ and $f\cdot g \in \calE(\sa)$
		\item $f^+ = \max(f, 0)$ and $f^- = \max (-f, 0)$ are in $\calE(\sa)$
		\item $\abs{f} \in \calE(\sa)$
	\end{enumerate}
\end{thm}

\begin{proof}Let
	\begin{align*}
		f = \sum_{i = 0}^M a_i \ind_{A_i},\qquad g = \sum_{j = 0}^N b_j \ind_{B_j}
	\end{align*}
	be the standard representations of $f$ and $g$, resp. Then,
	\begin{enumerate}
		\item
		\begin{align*}
			f \pm g = \sum_{i=0}^{M}\sum_{j=0}^{N} (a_i \pm b_j)\ind_{A_i \cap B_j},\qquad \sum_{i=0}^{M}\sum_{j=0}^{N} (a_i \cdot b_j)\ind_{A_i \cap B_j}
		\end{align*}
		are the standard representations of $f\pm g$ and $f \cdot g$, respectively and thus $f\pm g , f\cdot g \in \calE(\sa)$.
		
		\item
		\begin{align*}
			f^+ = \sum_{i \mid a_i \geq 0} a_i \ind_{A_i} \qquad f^- = \sum_{i \mid a_i \leq 0} - a_i \ind_{A_i}
		\end{align*}
		are the standard representations of $f^+$ and $f^-$, resp. and hence $f^+, f^- \in \calE(\sa)$.
		
		\item $\abs{f} = f^+ + f^- \in \calE(\sa)$ by the first two properties.
	\end{enumerate}
\end{proof}

\section{Sequences of simple functions. The sombrero lemma.}

\begin{thm}[Sombrero lemma]
	\label{thm:sombrero}
	
	Let $(X, \sa)$ be a measurable space and $u : X \to [0, \infty]$ a non-negative $\sa/\borel(\Rb)$-measurable function. Then, there exists an increasing sequence $(f_n)_{n \in \N} \subset \calE(\sa)$ of non-negative simple functions such that for any $x \in X$
	\begin{align*}
		u(x) = \lim_{n\to \infty} f_n(x) = \sup_{n\in\N} f_n(x).
	\end{align*}
\end{thm}

Note that in the previous statement, $f_n$ is a sequence of real-valued (as opposed to numerical) functions. Also, the limit is to be understood point-wise (as opposed to uniformly).

\begin{proof}
	TODO
\end{proof}

\begin{cor}
	Let $(X, \sa)$ be a measurable space. Then, for any numerical and $\sa/\borel(\Rb)$-measurable function $u: X \to \Rb$ there exists a sequence of simple functions $(f_n)_{n\in \N} \subset \calE(\sa)$ such that $\abs{f_n} \leq \abs{u}$ and
	\begin{align*}
		u(x) = \lim_{n\to \infty} f_n(x)
	\end{align*}
	
	Moreover, if $u$ is bounded then convergence is uniform (as opposed to point-wise).
\end{cor}

\begin{proof}
	Write $u = u^+ - u^-$, where $u^+, u^-$ are both non-negative functions. We first show that $u^+, u^-$ are $\sa/\borel(\Rb)$-measurable. We shall proceed by using \autoref{lem:measurable-generators-functions}, with generators of the form $(a, \infty])$. For any $a \in \Rb$ we have
	\begin{align*}
		\{u^+ > a\} = \begin{cases}
		X \tif a < 0 \\
		\{u \geq a\} \tif a \geq 0
		\end{cases}\in \sa
	\end{align*}
	Similarly,
	\begin{align*}
		\{u^- > a\} = \begin{cases}
		X \tif a < 0 \\
		\{-u \geq a\} = \{u < a\} \tif a \geq 0
		\end{cases}
		\in \sa.
	\end{align*}
	
	Since $u^+, u^-$ are non-negative measurable functions, \autoref{thm:sombrero} applies and there exist two sequences $(f_n)_{n\in\N}, (g_n)_{n\in\N} \subset \calE(\sa)$ such that $f_n \uparrow u^+$ and $g_n \uparrow u^-$. Hence
	\begin{align*}
		\lim_{n\to \infty} f_n - \lim_{n\to \infty} g_n = \lim_{n\to \infty} (f_n - g_n) = \lim_{n\to \infty} h_n = u^+ - u^- = u.
	\end{align*}
	Furthermore, $\abs{f_n - g_n} \leq \abs{f_n} - \abs{g_n} = f_n - g_n \leq u^+ + u^- = \abs{u}$. Finally, if $u$ is bounded then $\exists N \in \N$ such taht $u(x) \leq N, \forall x \in X$. Then from the proof of \autoref{thm:sombrero} (applied to $u^+$ and $u^-$) we see that $\forall n \geq N$ and $\forall x \in X$
	\begin{align*}
		\abs{f_n(x) - g_n(x) - u(x)} \leq \abs{f_n - u^+(x)} + \abs{g_n - u^-(x)} \leq \frac{1}{2^{n - 1}}.
	\end{align*}
	This implies uniform convergence.
\end{proof}

\paragraph{Convention} Given a sequence $(u_n)_{n\in\N} \subset \calM_{\Rb}(\sa)$, by $\sup_{n\in\N} u_n,\ \inf_{n\in\N} u_n, \limsup_{n\in\N} u_n$ and $\liminf_{n\in\N}$ we mean the point-wise defined functions $(\sup_{n\in\N} u_n)(x) = \sup_{n\in\N} u_n(x) = \sup\{u_n(x) \mid n \in \N\}$ and similarly for others.

Recall that
\begin{align}
	\liminf_{n\to \infty} u(x) = \sup_{n \geq 1} \inf_{m \geq n} u_m(x)
\end{align}
and
\begin{align}
	\limsup_{n\to\infty} u(x) = \inf_{n\geq 1}\sup_{m \geq n} u_m(x).
\end{align}

Also,
\begin{align*}
	v_n(x) &= \inf_{m \geq n} u_m(x) \uparrow \liminf_{n\to \infty} u(x),\\
	w_m(x) &= \sup_{m \geq n} u_m(x) \downarrow \limsup_{n\to\infty} u_n(x)
\end{align*}
and
\begin{align*}
	\inf_{n\in\N} u_n(x) \leq \liminf_{n\to\infty} u_n \leq \limsup_{n\to \infty} u_n(x) \leq \sup_{n\in\N} u_n(x).
\end{align*}


If $\liminf_{n\to \infty} u_n(x) = \limsup_{n\to \infty} u_n(x)$, then $\lim_{n\to \infty} u_n(x)$ exists and equals the common value.

\begin{cor}
	\label{cor:nasty-extended-measurable}
	Let $(X, \sa)$ be a measurable space and $(u_n)_{n\in\N} \subset \calM_{\Rb}(\sa)$ a sequence of $\sa/\borel(\Rb)$-measurable functions. Then, $\inf_{n\geq 1} u_n, \liminf_{n\to \infty} u_n, \limsup_{n\to \infty} u_n, \sup_{n \geq 1}$ are all $\sa/\borel(\Rb)$-measurable functions. 
\end{cor}

\begin{proof}
	First we prove that $\sup$ and $\inf$ are both $\sa/\borel(\Rb)$-measurable.
	\begin{align*}
		\{\sup_{n \geq 1} u_n \leq a\} = \bigcap_{n=1}^\infty \{u_n \leq a\} \in \sa\\
		\{\inf_{n \geq 1} u_n \geq a\} = \bigcap_{n=1}^\infty \{u_n \geq a\} \in \sa
	\end{align*}
	
	Now we use these to prove the rest.
	
	$\liminf_{n\to \infty} u_n = \sup_{n \geq 1} \inf_{m \geq n} u_m$. Set $v_n = \inf_{m \geq n} u_m$ as before and because of the above $v_n$ is $\sa/\borel(\Rb)$-measurable. Since $v_n \uparrow \liminf_{n\to \infty} u_n$ we have $\liminf_{n\to \infty} = \sup_{n \geq 1} v_n \in \sa$. Hence $\liminf_{n\to \infty} u_n$ is also $\sa/\borel(\Rb)$-measurable.
	
	Similarly, write $w_n = \sup_{m \geq n} u_n \in \sa$ because of the above. Then, since $w_n \downarrow \limsup_{n\to\infty} u_n$ we have $\limsup_{n\to \infty} u_n = \inf_{n \geq 1} w_n \in \sa$ and hence $\limsup_{n\to \infty} u_n$ is $\sa/\borel(\Rb)$-measurable.
\end{proof}

In fact, we can deduce more.

\begin{cor}
	\label{cor:properties-measurable}
	Let $u, v \in \calM_\Rb(\sa)$ be two $\sa/\borel(\Rb)$-measurable functions. Then $u \pm v,\ u \lor v = \max\{u, v\}$ and $u \land v = \min\{u, v\}$ are $\sa/\borel(\Rb)$-measurable.
\end{cor}

\begin{proof}
	By \autoref{thm:sombrero} there exist two sequences of non-negative simple functions $(f_n)$ and $(g_n)$ such that $f_n \uparrow u$ and $g_n \uparrow v$. Since $f_n$ and $g_n$ are simple functions, by \autoref{thm:prop-simple-functions} we have that $f_n \pm g_n$ is also a simple function with $\lim_{n\to \infty} f_n + g_n = u + v$. Moreover, $f_n + g_n$ is $\sa/\borel(\R)$-measurable (since $f_n, g_n$ are) and thus it is also $\sa/\borel(\Rb)$-measurable and hence $u+v$ is also $\sa/\borel(\Rb)$-measurable.
	
	Similarly, we can prove the corollary for $u\lor v$ and $u \land v$.
\end{proof}

\begin{remark}
	Applying the above to $u^+, u^-$ we see that $u$ is $\sa/\borel(\R)$-measurable if and only if $u^+$ and $u^-$ are $\sa/\borel(\R)$-measurable.
\end{remark}

\begin{cor}
	If $u, v \in \calM_\Rb(\sa)$, then the following sets are all measurable:
	\begin{align*}
		\{u \leq v\}, \{u < v\}, \{u = v\}, \{u > v\}, \{u \geq v\}
	\end{align*}
\end{cor}

\begin{proof}
	TODO
\end{proof}
