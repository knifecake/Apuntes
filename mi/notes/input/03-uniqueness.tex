% !TeX root = ../mi-notes.tex

\chapter{Uniqueness of measures}

In this chapter we shall introduce some technicalities to be able to extend the Lebesgue measure to arbitrary Borel sets in $\borel(\R^n)$. The first tool is a generalized version of a \siga, where union closure is only required for disjoint unions. We will explore the properties of this new construct and try to draw similarities to what we know about \sigas. Finally we will give a result on the conditions that a \siga must meet to be able to uniquely define a measure on it by defining the measure on the elements of a generator.

\section{Preliminaries}


\begin{dfn}[Dynkin system]
	\label{dfn:dynkin-system}
	A family $\calD \subset \powerset(X)$ is called a Dynkin system if it satisfies these three properties:
	\begin{enumerate}
		\item $X \in \calD$
		\item $\forall A \in \powerset(X),\ A \in \calD \implies A^\complement \in \calD$
		\item For any countable collection of pairwise disjoint sets $(A_n)$
		\begin{align*}
			(A_n)_{n\in\N} \subset \calD \implies \bigcupdot_{n \in \N} A_n \in \calD
		\end{align*}
	\end{enumerate}
\end{dfn}

\begin{remark}
	Every \siga is a Dynkin system but the converse need not be true.
\end{remark}


We have an analogue version to the generator theorem for \sigas in the case of Dynkin systems:

\begin{dfn}
	Let $\calG \subset \powerset(X)$ be a collection of subsets of $X$. We define the Dynkin system generated by $\calG$ as
	\begin{align*}
		\delta(\calG) = \bigcap_{\calG \subset \calC, \calC \text{Dynkin}} \calC
	\end{align*}
\end{dfn}

\begin{lem}
	Let $\calG \subset \powerset(X)$ be a collection of subsets of $X$. Then $\delta(\calG)$ is a Dynkin system and it is the smallest. Moreover, $\calG \subseteq \delta(\calG) \subseteq \sigma(\calG)$.
\end{lem}

\begin{proof}$ $\newline
	
	\begin{itemize}
		\item $\delta(\calG)$ is a Dynkin system because of the restriction on the intersection. Note that the intersection is non-empty because $\powerset(X)$ is a Dynkin system and that the intersection of Dynkin systems is also a Dynkin system (the proof is the same as the one for \sigas but using disjoint unions, see \autoref{rem:generators}).
		
		\item Now let's look at why it is the smallest. Suppose there is another Dynkin system $\calD$ that contains the collection $\calG$. Then $\calD$ is a Dynkin system and therefore intervenes in the intersection. Therefore $\delta(\calG) \subset \calD$.
		
		\item Finally, $\calG \subseteq \delta(\calG)$ follows from the restriction in the intersection and $\delta(\calG) \subseteq \sigma(\calG)$ follows from the fact that every \siga is a Dynkin system and hence $\delta(\calD) = \calD$ (by minimality) and $\delta(\sigma(\calG)) = \sigma(\calG)$. Therefore, $\calG \subseteq \sigma(\calG) \implies \delta(\calG) \subset \delta(\sigma(\calG)) = \sigma(\calG)$.
	\end{itemize}
\end{proof}

Also, as with \sigas, we have

\begin{remark}
	\label{remark:generators-dynkin}
	If $\calF \subseteq \calG$ then $\delta(\calF) \subseteq \delta(\calG)$.
\end{remark}

So, we already saw that every \siga is a Dynkin system and that Dynkin systems can be generated from a collection of subsets. A natural question to ask is, when is a Dynkin system a \siga? Let us introduce some terminology first.


\begin{dfn}[Stable under finite intersection]
	\label{dfn:istable}
	Let $\calD$ be a collection of sets. We say that $\calD$ is stable under finite intersection, or closed under finite intersection or $\cap$-stable if
	\begin{align*}
	C, D \in \calD \implies C \cap D \in \calD
	\end{align*}
	
	Some sources call such a collection a $\Pi$-system.
\end{dfn}

\begin{lem}
	Let $\calD$ be a Dynkin system. $\calD$ is \istable $\iff \calD$ is a \siga.
\end{lem}

\begin{proof}
	It is clear that going from right to left is true, since all \sigas are \istable.
	
	For the implication from left to right we proceed as follows. Assuming $\calD$ is a \istable Dynkin system, to prove that $\calD$ is a \siga we only need to generalise the last property to countable unions of arbitrary collections of sets in $\calD$, not just disjoint ones. Let $(A_n)_{n\in\N} \subset \calD$ be a sequence of sets in $\calD$. We define a new sequence $(E_n)_{n\in\N}$ by
	\begin{align*}
	E_1 = D_1 \text{ for } n = 1,\qquad E_n = D_n \setminus E_{n-1} \text{ for } n > 1.
	\end{align*}
	Rewriting the set difference as an intersection we have that
	\begin{align*}
	E_n = D_n \setminus E_{n-1} = D_n \setminus \bigcup_{i=1}^{n-1}D_i = D_n \cap \left(\bigcap_{i=1}^{n-1} D_i^\complement\right) \in \calD
	\end{align*}
	since $\calD$ is \istable.
	It is clear that $E_n$ are pairwise disjoint, i.e. $E_i \cap E_j = \emptyset$ for $i \neq j$. We also have that
	\begin{align*}
	\bigcupdot_{n \in \N} E_n = \bigcup_{n \in \N} D_n \in \calD
	\end{align*}
	Hence, $\calD$ is a \siga.
\end{proof}

\begin{remark}
	As a consequence we have that if $\calG \subset \powerset(X)$ and $\delta(\calG)$ is \istable then $\delta(\calG)$ is a \siga containing $\calG$ therefore $\sigma(\calG) \subseteq \delta(\calG)$. Since $\delta(\calG) \subseteq \sigma(\calG)$ always holds, we have that
	\begin{align*}
	\text{ if } \delta(\calG) \text{ is $\cap$-stable then } \delta(\calG) = \sigma(\calG)
	\end{align*}
\end{remark}

This is nice, but it would even be nicer if we could just argue about the generators, since Dynkin systems and \sigas are hard to reason about. We'll do just that.

\begin{lem}
	Let $\calG$ be a collection of subsets of $X$. If $\calG$ is \istable then $\delta(\calG) = \sigma(\calG)$.
\end{lem}

\begin{proof}
	We only need to show that if $\calG$ is \istable then $\delta(\calG)$ also is. Then, because of the previous remark we have $\delta(\calG) = \sigma(\calG)$.
	
	We shall proceed in steps.
	\begin{enumerate}
		\item \textbf{Claim 1.} For each $E \in \delta(\calG)$, the collection $\calD_E = \{ F \subseteq X \mid F \cap E \in \delta(\calG)\}$ is a Dynkin system. We prove the three properties of a Dynkin system.
		\begin{enumerate}
			\item Clearly $\emptyset = \emptyset \cap E \in \calD_E$
			\item For any $F \in \calD_E$ we have $F^\complement = X \setminus F \subseteq X$ and
			\begin{align*}
				 F^\complement \cap E =& (F^\complement \cup E^\complement) \cap E \\
				 &= (F \cap E)^\complement \cap E \\
				 &= (F \cap E) \cupdot E^\complement \in \delta(\calG),
			\end{align*}
			since $F \cap E \in \delta(\calG)$ by hypothesis and $E^\complement \in \delta(\calG)$ since $\delta(\calG)$ is a Dynkin system.
			\item Finally, for any collection of disjoint subsets $(F_n)_{n\in\N} \subset \calD_E$ we need to show that the disjoint union is still in $\calD_E$. We have that for all $n \in \N,\ F_n \cap E \in \delta(\calG)$ by hypothesis so
			\begin{align*}
				\bigcupdot_{n \in \N} F_n \cap E = \bigcupdot_{n\in\N} (F_n \cap E) \in \delta(\calG).
			\end{align*}
		\end{enumerate}
		\item \textbf{Claim 2.} $\calG \subset \calD_G,\ \forall G \in \calG$. Let $G' \in \calG$. Since $\calG$ is \istable we have that $G' \cap G \in \calG$ and hence $G' \cap G \in \delta(\calG) \implies G' \in \calD_G$.
	\end{enumerate}

	As a consequence of these claims we have that, since $\calG \subset \calD_G$ then $\delta(\calG) \subset \delta(\calD_G) = \calD_G$. Therefore, for any $E \in \delta(\calG)$ and any $G \in \calG$ we have that $E \cap G \in \delta(\calG)$. This also shows that $\delta(\calG) \subset \calD_E$, for any $E \in \delta(\calG)$. In other words we have that for any $E, F \in \delta(\calG),\ E \cap F \in \delta(\calG)$. Thus, $\delta(\calG)$ is \istable implying that $\delta(\calG)$ is a \siga and hence $\delta(\calG) = \sigma(\calG)$.
\end{proof}

\section{Uniqueness of measures}

Now we move on to define the requirements needed to be able to guarantee uniqueness of a measure over a \siga given a definition of the measure on a generator of that \siga.

\begin{thm}[Uniqueness of measures]
	\label{thm:uniqueness-measures}
	Let $(X, \sa)$ be a measurable space where $\sa = \sigma(\calG)$ for some collection $\calG$ of subsets of $X$ where $\calG$ satisfies the following:
	\begin{enumerate}
		\item $\calG$ is \istable (so $\delta(\calG) = \sigma(\calG)$), and
		\item there exists an exhausting sequence $(G_n)_{n \in \N} \subset \calG$ such that $G_n\uparrow X$ (so $X = \bigcup_{n\in\N}G_n$). 
	\end{enumerate}

	If $\mu, \nu$ are measures on $\sa = \sigma(\calG)$ such that $\mu(G) = \nu(G) < \infty,\ \forall G \in \calG$ then $\mu = \nu$, i.e. $\mu(A) = \nu(A),\ \forall A \in \sa$.
\end{thm}

\begin{proof}
	The plan is to use the good set principle (\autoref{rem:good-set-principle}) to prove that a set where the measures coincide is equal to the \siga $\sa$. Therefore we can conclude that $\mu = \nu$ everywhere.
	
	Define, for every $n \in \N$ (and hence for every $G_n \in \calG$)
	\begin{align}
		\calD_N = \{ A \in \sa \mid \mu(G_n \cap A) = \nu(G_n \cap A)\}.
	\end{align}
	Note that $\mu(G_n\cap A) \leq \mu(G_n) < \infty$ so $\calD_n$ is well defined. The goal is to see that $\calD_n = \sa$ but that is too hard. We will prove that $\calD_n$ is a Dynkin system.
	\begin{enumerate}
		\item Clearly $X \in \calD_n$ since $\mu(G_n \cap X) = \mu(G_n) = \nu(G_n) = \nu(G_n \cap X)$.
		\item Let $A \in \calD_n$. Then
		\begin{align*}
			\mu(G_n \cap A^\complement)
			&= \mu(G_n \cap (X\setminus A))\\
			&= \mu((G_n \cap X) \setminus (G_n \cap A))\\
			&\overset{\ref{thm:prop-measures}}{=} \mu(G_n) - \mu(G_n \cap A)\\
			&= \nu(G_n) - \nu(G_n \cap A)\\
			&\overset{\ref{thm:prop-measures}}{=} \nu(G_n \setminus (G_n \cap A))\\
			&= \nu (G_n \cap (X \setminus A))
			= \nu(G_n \cap A^\complement).
		\end{align*}
		\item Let $(A_j)_{j\in\N}$ be a pairwise disjoint sequence in $D_n$. Then\footnote{This is the reason why we prove that $\calD_n$ is a Dynkin system and not a \siga. Doing the latter would be much complicated because we wouldn't have $\sigma$-additivity.},
		\begin{align*}
			\mu\left( G_n \cap \bigcupdot_{j \in \N} A_j \right)
			&= \mu\left(\bigcupdot_{j \in \N} (G_n \cap A_j) \right)\\
			&= \sum_{j\in\N} \mu(G_n \cap A_j)\\
			&= \sum_{j\in\N} \nu(G_n \cap A_j)\\
			&= \nu\left(\bigcupdot_{j \in \N} (G_n \cap A_j) \right)
			= \nu\left( G_n \cap \bigcupdot_{j \in \N} A_j \right).
		\end{align*}
	\end{enumerate}

	Now we want to prove that $\calD_n = \sa$. By definition we already have that $\calD_n \subseteq \sa$. On the other hand, $\mu(G_n \cap G) = \nu(G_n \cap G),\ \forall G \in \calG$ (by hypothesis, the measures agree on the generators) so $\calG \subseteq \calD_n$. By \autoref{remark:generators-dynkin}, the previous implies that $\delta(\calG) \subseteq \delta(\calD_n) = \calD_n$, but since $\calG$ is \istable then $\delta(\calG) = \sigma(\calG)$ and thus $\sigma(\calG) = \sa \subseteq \calD_n$.
	
	So that means that
	\begin{align*}
		\forall A \in \sa,\ \mu(G_n \cap A) = \nu(G_n \cap A).
	\end{align*}
	We use \autoref{thm:prop-measures} to take the limit and
	\begin{align*}
		\forall A \in \sa,\ \mu(A)
		= \mu(X \cap A)
		&= \lim_{n\to\infty} \mu(G_n \cap A)\\
		&= \lim_{n\to\infty} \nu(G_n \cap A)
		= \nu(X \cap A)
		= \nu(A).
	\end{align*}
\end{proof}

Suppose that we don't have an exhausting sequence on the generator. A neat trick (that doesn't always work) is to extend the generator to $\calG \cup \{X\}$ and define the trivial exhausting sequence $G_n = X,\ \forall n \in \N$. If it holds that $\mu(X) < \infty$, i.e., if $\mu$ is a finite measure, then we are in a position to apply this theorem. See exercise 5.9 for an opportunity to apply this trick.

\begin{lem}$ $\newline
	\label{lem:lebesgue-translation-invariant}
	\begin{enumerate}
		\item The $n$-dimensional Lebesgue measure $\lambda^n$ is invariant under translations, i.e.
		\begin{align*}
			\lambda^n(x + B) = \lambda^n(B),\quad \forall x \in \R`n, B \in \borel(\R^n),
		\end{align*}
		where $x + B = \{x + y \mid y \in B\}$.
		
		\item Every measure $\mu$ on $(\R^n, \borel(\R^n))$ which is invariant under translations and satisfies $\kappa = \mu([0,1)^n) < \infty$ is a multiple of the Lebesgue measure $\mu = \kappa \lambda^n$.
	\end{enumerate}
\end{lem}

\begin{proof}$ $\newline
	\begin{enumerate}
		\item First of all, let us check that $\lambda(x + B)$ is well defined, i.e. that $B \in \borel(\R^n) \implies x + B \in \borel(\R^n)$. There is a clever way to do this. Define
		\begin{align*}
			\sa_x = \{B \in \borel(\R^n) \mid x + B \in \borel(\R^n)\} \subset \borel(\R^n).
		\end{align*}
		It is clear that $\sa_x$ is a \siga on $\R^n$ and that $\calJ \subset \sa_x$ since translations of half-open intervals are still half-open intervals and hence in $\calJ$ and thus in $\borel(\R^n)$. Therefore $\borel(\R^n) = \sigma(\calJ) \subset \sa_x \subset \borel(\R^n)$. Now we can start with the meat of the proof.
		
		Define $\nu(B) = \lambda^n(x + B)$ for any $B \in \borel(\R^n)$ and some fixed $x =(x_1, \dots, x_n) \in \R^n$. $\nu$ is a measure on $(\R^n, \borel(\R`n))$ since
		\begin{align*}
			\nu(\emptyset) = \lambda^n(x + \emptyset) = \lambda^n\left(\bigtimes_{i = 1}^n [x_i + a, x_i + a)\right) = \prod_{i=1}^n (x_i + a - (x_i + a)) = 0\\
			\text{ and }\nu\left(\bigcupdot_{j = 1}^\infty A_j \right) = \lambda^n\left(\bigcupdot_{j = 1}^\infty x + A_j\right) = \sum_{j = 1}^\infty \lambda^n (x + A_j) = \sum_{j = 1}^\infty \nu(A_j).
		\end{align*}
		Now take $I = \bigtimes_{i = 1}^n [a_i, b_i] \in \calJ$ and note that
		\begin{align*}
			\nu(I) = \lambda^n(x + I) &= \lambda^n \left(\bigtimes_{i = 1}^n [a_i + x_i, b_i + x_i]\right) \\
			&= \prod_{i=1}^n (b_i + x - (a_i + x)) = \prod_{i=1}^n (b_i - a_i) = \lambda^n(I)
		\end{align*}
		
		This means that, if we restrict ourselves to the generator $\calJ$, we have that $\nu\mid_\calJ = \lambda^n\mid_\calJ$.\footnote{Where $f\mid A$ denotes the restriction of $f: X \to Y$ to the new domain $A \subset X$.} Recall that $\borel(\R^n) = \sigma(\calJ)$ and that the generator $\calJ$ admits the exhausting sequence $[-k, k)_{k \in \N} \subset \calJ$ with $\nu([-k, k)) = \lambda^n([-k, k)) = (2k)^n < \infty$. Hence, using \autoref{thm:uniqueness-measures}, the measures $\nu$ and $\lambda^n$ must coincide in every $A \subset \borel(\R^n)$.
		
		\item Similarly, take $I \in \calJ$ but this time with rational endpoints $a_i, b_i \in \Q$. Then there is some $M \in \N$ and some $k(I) \in \N$ and points $x^i \in \R^n$ such that
		\begin{align}
			I =\bigcupdot_{i = 1}^{k(I)} \left(x^i + [0, \frac{1}{M})^n\right).
		\end{align}
		What we did here is pave $I$ with little squares of side length $\frac{1}{M}$ and lower left corner $x^i$, where $M$ could be the common denominator of $a_i$ and $b_i$. Now, $\mu$ (by hypothesis) and $\lambda$ (by part 1) we can write
		\begin{align*}
			\mu(I)
			&= \sum_{i=1}^{k(I)} \mu\left(x^i + \left[0, \frac{1}{M}\right)^n\right)
			= \sum_{i=1}^{k(I)}  \mu\left(\left[0, \frac{1}{M}\right)^n\right)
			= k(I) \mu\left( \left[0, \frac{1}{M}\right)^n \right) \\
			\lambda^n(I)
			&= \sum_{i=1}^{k(I)} \lambda^n\left(x^i + \left[0, \frac{1}{M}\right)^n\right)
			= \sum_{i=1}^{k(I)}  \lambda^n\left(\left[0, \frac{1}{M}\right)^n\right)
			= k(I) \lambda^n\left( \left[0, \frac{1}{M}\right)^n \right),
		\end{align*}
		and for $I = [0,1)^n$,
		\begin{align*}
			\mu([0,1)^n)
			= k([0,1)^n) \mu\left( \left[0, \frac{1}{M}\right)^n \right)
			= M^n \mu\left( \left[0, \frac{1}{M}\right)^n \right) \\
			\lambda^n([0,1)^n)
			= k([0,1)^n) \lambda^n\left( \left[0, \frac{1}{M}\right)^n \right)
			= M^n \lambda^n\left( \left[0, \frac{1}{M}\right)^n \right),
		\end{align*}
		since it takes $k([0,1)^n) = M^n$ rectangles of side $\frac{1}{M}$ to cover $[0,1)^n$.
		Thus
		\begin{align*}
			\frac{\mu(I)}{\mu([0,1)^n)} = \frac{k(I)}{M^n} &\implies \mu(I) = \frac{k(I)}{M^n}\mu([0,1)^n) \\
			\frac{\lambda^n(I)}{\lambda^n([0,1)^n)} = \frac{k(I)}{M^n} &\implies \lambda^n(I) = \frac{k(I)}{M^n} \underbrace{\lambda^n([0,1)^n)}_{=1} = \frac{k(I)}{M^n}.
		\end{align*}
		Thus, $\mu(I) = \mu([0,1)^n) \lambda^n(I) = \kappa \lambda^n(I)$ and application of \autoref{thm:uniqueness-measures} finishes the proof.
	\end{enumerate}
\end{proof}