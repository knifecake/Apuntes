% !TeX root = ../mi-notes.tex

\chapter{Integrals of non-negative functions}

We are now ready to define the integral of a non-negative function in terms of a measure.

\section{Integral of a non-negative simple function}

Since each $u \in \calM^+(\sa)$ or $\calM^+_\Rb(\sa)$ is a limit of an increasing sequence of simple functions (\autoref{thm:sombrero}), we concentrate first on the collection $\calE^+ = \calE^+(\sa)$ of all $\sa/\borel(\R)$-measurable non-negative simple functions.

Let $f(x) = \sum_{i = 0}^N a_i \ind_{A_i}$ be the standard representation of $f \in \calE^+$. Recall that this means that $(A_i)$ is a collection of sets in $\sa$ which define a partition of $X$ (i.e. $X = \bigcupdot_{i = 0}^N A_i)$. Furthermore, on $A_i$, $f$ takes the value $a_i$.

Let
\begin{align}
	I_\mu(f) := \sum_{i = 0}^N a_i \mu(A_i).
\end{align}
We want to interpret $I_mu(f)$ as $\int f d\mu$, but there might be a small problem, namely $f$ might have more than one standard representation so we need to check that the definition of $I_\mu(f)$ is independent of the specific representation, i.e. they all give the same answer.

\begin{lem}[Integrals of simple functions are representation-invariant]
	If $f = \sum_{i = 0}^M a_i \ind_{A_i} = \sum_{j = 0}^N b_j \ind{B_j}$ are two standard representations of a simple function then $\sum_{i = 0}^M a_i\mu(A_i) = \sum_{j = 0}^n b_j \mu(B_j)$.
\end{lem}

\begin{proof}
	Recall that since $(A_j)$ and $(B_j)$ are partitions we may write
	\begin{align*}
		A_i &= A_i \cap \bigcupdot_{j = 0}^M B_j = \bigcupdot_{j = 0}^M A_i \cap B_j, \\
		B_j &= B_j \cap \bigcupdot_{i = 0}^N A_i = \bigcupdot_{i = 0}^N A_i \cap B_j.
	\end{align*}
	Thus,
	\begin{multline*}
		\sum_{i = 0}^M a_i \mu(A_i)
		= \sum_{i = 0}^M a_i \mu\left(\bigcupdot_{j = 0}^M A_i \cap B_j\right)
		= \sum_{i = 0}^M a_i \sum_{j = 0}^N \mu(A_i \cap B_j)\\
		= \sum_{i = 0}^M \sum_{j = 0}^N a_i \mu(A_i \cap B_j)
		= \sum_{j = 0}^N \sum_{i = 0}^M a_i \mu(A_i \cap B_j).
	\end{multline*}
	Now,
	\begin{itemize}
		\item if $A_i \cap B_j = \emptyset$ then $a_i\mu(A_i \cap B_j) = 0 = b_j \mu(A_i \cap B_j)$, and
		\item if $A_i \cap B_j \neq \emptyset$ then $\exists x \in A_i \cap B_j$ so that $f(x) = a_i = b_j$. Thus $a_i \mu(A_i \cap B_j) = b_j \mu(A_i \cap B_j)$.
	\end{itemize}
	Therefore,
	\begin{multline*}
		\sum_{i = 0}^m a_i \mu(A_i)
		= \sum_{j = 0}^N \sum_{i = 0}^M a_i \mu(A_i \cap B_j)
		= \sum_{j = 0}^N \sum_{i = 0}^N b_j \mu(A_i \cap B_j)\\
		= \sum_{j = 0}^N b_j \sum_{i = 0}^M \mu(A_i \cap B_j)
		= \sum_{j = 0}^N b_j \mu(B_j)
	\end{multline*}
\end{proof}

So $I_\mu(f)$ is well-defined regardless of the chosen standard representation. We are about to be ready to define the integral for arbitrary non-negative functions (once we verify that the properties of simple functions that have allowed us to state \autoref{thm:sombrero} hold under integrals).

\begin{thm}[Properties of $I_\mu(f)$]
	\label{thm:prop-integrals-simple}
	Let $f, g \in \calE^+(\sa)$ and $\lambda \geq 0$, then
	\begin{enumerate}
		\item $I_\mu(\ind_A) = \mu(A)$, for any $A \in \sa$
		\item $I_\mu(\lambda f) = \lambda I_\mu(f)$ (positive homogenous degree 1)
		\item $I_\mu(f + g) = I_\mu(f) + I_\mu(f)$ (additive)
		\item if $g \leq g$ then $I_\mu(f) \leq I_\mu(g)$ (monotone)
	\end{enumerate}	
\end{thm}

\begin{proof}
	TODO
\end{proof}

\section{Integral of a non-negative function}

\begin{dfn}[$(\mu)$-integral of a non-negative function]
	\label{dfn:mu-integral}
	Let $u \in \calM^+_\Rb(\sa)$, then the $(\mu)$-integral of $u$ is defined by
	\begin{align}
		\int u d\mu := \sup \{ I_\mu(g) \mid g \leq u, g \in \calE^+(\sa)\} \in [0, \infty]
	\end{align}
\end{dfn}

Some alternative notations are
\begin{align*}
	\int u d\mu = \int_X u d\mu = \int_X u(x) d\mu(x).
\end{align*}

Note that here we are taking the supremum over all simple functions less than or equal to $\mu$. We will eventually show that it is enough to find a sequence $(f_n)_{n\in\N} \subset \calE^+(\sa)$ such that $f_n \uparrow u$, since we will prove that
\begin{align*}
	\int u d\mu = \lim_{n\to \infty} \int f_n d\mu = \sup_{n \geq 1} \int f_n d\mu.
\end{align*}

Before we move one, let's make sure that we did not break anything. Namely, does the definition of the integral above extend the definition of $I_\mu$ for non-negative simple functions, i.e. $I_\mu(f) = \int f d\mu$ if $f$ is a non-negative simple function. The answer is yes!

\begin{lem}
	\label{lem:def-integral-simple}
	If $f \in \calE^+(\sa)$ then $I_\mu(f) = \int f d\mu$.
\end{lem}

\begin{proof}
	Since $f \in \calE^+(\sa)$ and $f \leq f$, then by definition of the supremum, we have $I\mu(f) \leq \int f d\mu$. Now, by monotonicity of $I_\mu$, for any $g \in \calE^+(\sa)$ such that $g \leq f$ one has $I_\mu(g) \leq I_\mu(f)$. Thus,
	\begin{align*}
		\int f d\mu = \sup \{ I_\mu(g) \mid g \leq f,\ g \in \calE^+(\sa)\} \leq I_\mu(f).
	\end{align*}
	This shows that $I_\mu(f) = \int f d\mu$.
\end{proof}

\begin{remark}
	\label{rem:mu-integrals-monotone}
	It is easy to see that $\mu$-integrals are monotone. Since if $u \leq v$, then $\{g \in \calE^+(\sa) \mid g \leq u\} \subset \{h \in \calE^+(\sa) \mid h \leq v\}$. Hence,
	\begin{align*}
		\int u d\mu = \sup \{I_\mu(g) \mid g \in \calE^+(\sa), g \leq u\} \leq \sup \{I_\mu(h) \mid h \in \calE^+(\sa), h \leq v\} = \int v d\mu
	\end{align*}
\end{remark}

Now we come to the central theorem of this chapter.

\begin{thm}[Beppo-Lévi]
	\label{thm:beppo-levi}
	Let $(X, \sa, \mu)$ be a measure space and let $(u_n)_{n\in\N}$ be an increasing sequence in $\calM^+_\Rb(\sa)$. Then,
	\begin{align*}
		u := \sup_{n \geq 1} u_n = \lim_{n \to \infty} u_n \in \calM_\Rb^+(\sa),
	\end{align*}
	and
	\begin{align}
		\int u d\mu = \int \sup_{n \geq 1} u_n d\mu = \sup_{n \geq 1} \int u_n d\mu,
	\end{align}
	or, alternatively,
	\begin{align}
		\int u d\mu = \int \lim_{n \to \infty} u_n d\mu = \lim_{n \to \infty} \int u_n d\mu
	\end{align}
\end{thm}

Notice that the assumption that $u_n$ is an increasing sequence is necessary to be able to interchange limits with suprema.

\begin{proof}
	TODO. Important
\end{proof}

The following corollary is really just a more concise statement of the previous theorem.
\begin{cor}
	\label{cor:beppo-levi}
	Let $u \in \calM_\Rb^+(\sa)$ be a non-negative, numerical, $\sa/\borel(\Rb)$-measurable function and $(f_n)_{n\in\N} \subset \calE^+(\sa)$ be a sequence of non-negative simple functions such that $f_n \uparrow u$. Then,
	\begin{align}
		\int u d\mu = \lim_{n \to \infty} \int f_n d\mu
	\end{align}
\end{cor}

\begin{proof}
	Clearly the hypothesis of \autoref{thm:beppo-levi} are satisfied. (Just take $u_n = f_n$).
\end{proof}

Now we extend \autoref{thm:prop-integrals-simple} to the case for integrals of arbitrary non-negative functions.

\begin{thm}[Properties of integrals of non-negative functions]
	\label{thm:prop-integrals}
	Let $(X, \sa, \mu)$ be a measure space. Let $u, v \in \calM^+_\Rb(\sa)$. Then,
	\begin{enumerate}
		\item $\int \ind_{A} d\mu = I_\mu(A) = \mu(A)$
		\item $\int \alpha u d\mu = \alpha \int u d\mu,\ \forall \alpha \geq 0$.\footnote{Here we require $\alpha \geq 0$ to keep the functions non-negative. As soon as we generalise to arbitrary functions this assumption is not needed.}
		\item $\int (u + v)d\mu = \int u d\mu + \int v d\mu$
		\item if $u \leq v$, then $\int u d\mu \leq \int v d\mu$
	\end{enumerate}
\end{thm}

\begin{proof}$ $\newline
	\begin{enumerate}
		\item $\ind_A$ is clearly a simple function so, by definition $\int \ind_A d\mu = I_\mu(\ind_A) = \mu(A)$.
		\item By \autoref{thm:sombrero} there exists a sequence of simple functions $(f_n)_{n\in\N} \subset \calE^+$ such that $f_n \uparrow u$. Then,
		\begin{align*}
			\int \alpha u d\mu
			\overset{\ref{cor:beppo-levi}}{=} \lim_{n \to \infty} \int \alpha f_n d\mu
		    \overset{\ref{lem:def-integral-simple}}{=} \lim_{n \to \infty} I_\mu(\alpha f_n)
		    \overset{\ref{thm:prop-integrals-simple}}{=} \alpha \lim_{n \to \infty} I_\mu(f_n)
		    \overset{\ref{lem:def-integral-simple}}{=} \alpha \int u d \mu
		\end{align*}
		\item Once again by \autoref{thm:sombrero} there exist sequences $(f_n)_{n\in\N}, (g_n)_{n\in \N} \subset \calE^+$ such that $f_n \uparrow u$ and $g_n \uparrow v$. Then, $(f_n + g_n) \uparrow (u + v)$ and hence,
		\begin{align*}
			\int (u + v)d\mu
			\overset{\ref{cor:beppo-levi}}{=} \lim_{n \to \infty}  \int (f_n  + g_n)d\mu
			\overset{\ref{lem:def-integral-simple}}{=} \lim_{n \to \infty} I_\mu(f_n + g_n)\\
			\overset{\ref{thm:prop-integrals-simple}}{=} \lim_{n \to \infty} I_\mu(f_n) + \lim_{n \to \infty} I_\mu(g_n)
			\overset{\ref{lem:def-integral-simple}}{=} \int u d\mu + \int v d\mu
		\end{align*}
		\item Already proven in \autoref{rem:mu-integrals-monotone}.
	\end{enumerate}
\end{proof}

\begin{cor}
	\label{cor:integral-countable-sum}
	Let $(u_n)_{n\in\N} \subset \calM^+_\Rb(\sa)$. Then, $\sum_{n = 1}^{\infty} u_n \in \calM_\Rb^+(\sa)$ and
	\begin{align}
		\int \sum_{n = 1}^\infty u_n d\mu = \sum_{n = 1}^\infty \int u_n d\mu
	\end{align}
\end{cor}

\begin{proof}
	Let $v_m = \sum_{n=1}^{m} u_n$. Then $v_m \uparrow \sum_{n=1}^{\infty} u_n$ and thus
	\begin{align*}
		\int \sum_{n = 1}^\infty u_n d\mu
		= \int \lim_{m \to \infty} v_m d\mu
		\overset{\ref{cor:beppo-levi}}{=} \lim_{m \to \infty} \int v_m d\mu
		= \lim_{m \to \infty} \int \sum_{n = 1}^m u_n d\mu \\
		\overset{\ref{thm:prop-integrals}}{=} \lim_{m \to \infty} \sum_{n=1}^{m} \int u_n d\mu
		= \sum_{n = 1}^\infty \int u_n d\mu
	\end{align*}
\end{proof}

\begin{thm}[Fatou's lemma]
	\label{thm:fatou-lemma}
	Let $(u_n)_{n\in\N}$ be \textbf{any} sequence in $\calM_\Rb^+(\sa)$. Then,
	\begin{align}
		\label{eq:fatou1}
		u := \liminf_{n\to \infty} u_n \in \calM_\Rb^+(\sa)
	\end{align}
	and
	\begin{align}
		\label{eq:fatou2}
		\int \liminf_{n\to \infty} u_n d\mu \leq \liminf_{n\to \infty} \int u_n d\mu
	\end{align}
\end{thm}

The great thing about \autoref{thm:fatou-lemma} is that it makes no assumptions of monotonicity and can be applied to any sequence of non-negative, numerical, measurable functions.

\begin{proof}
	\autoref{eq:fatou1} follows from \autoref{cor:nasty-extended-measurable}.
	
	As for \autoref{eq:fatou2}, we have
	
	\begin{align*}
		\int \liminf_{n\to \infty} u_n d\mu
		&= \int \sup_{n\in\N} \inf_{j \geq n} u_j d\mu \\
		&\overset{\ref{thm:beppo-levi}}{=} \sup_{n\in\N} \int \inf_{j \geq n} u_j d\mu \\
		&\overset{\ref{thm:prop-integrals}}{\leq} \sup_{n\in\N} \inf_{l \geq n} \int u_l d\mu\\
		&= \liminf_{n\to\infty} \int u_n d\mu
	\end{align*}
	where we used the fact that $\inf_{j \geq n} \leq u_l$ for any $l \geq n$.
\end{proof}

We also have a similar result for $\limsup$, sometimes known as the reverse Fatou's lemma.

\begin{cor}[Reverse Fatou's lemma]
	\label{cor:reverse-fatou}
	
	Let $(u_n)_{n\in\N} \subset \calM^+(\sa)$ be a sequence of non-negative numerical measurable functions. If $u_n \leq u$ for all $n \in \N$ and some $v \in \calM^+(\sa)$ such that $\int v d\mu < \infty$, then
	\begin{align}
		\limsup_{n\to\infty} \int u_n d\mu \leq \int \limsup_{n\to \infty} u_n d\mu
	\end{align}
\end{cor}

\begin{proof}
	Let $w_n = v - u_n \geq 0$, i.e. $(w_n)_{n\in\N}$ is a sequence of non-negative measurable functions. By \autoref{thm:fatou-lemma} we get,
	\begin{align*}
		\int \liminf_{n\to \infty} w_n d\mu
		&\leq \liminf \int w_n d\mu \\
		&= \liminf_{n\to \infty} \left( \int v d\mu - \int u_n d\mu \right)\\
		&= \int vd\mu - \limsup_{n\to\infty} \int u_n d\mu.
	\end{align*}
	
	(Recall that $\liminf (-u_n) = -\limsup u_n$.) Thus,
	\begin{align*}
		\int v d\mu - \limsup_{n\to\infty} \int u_n d\mu
		&\geq \int \liminf_{n\to \infty} w_n d\mu \\
		&= \int \liminf_{n\to\infty} \left(v - w_n\right)d\mu \\
		&= \int vd\mu - \int \limsup_{n\to\infty} u_n d\mu.
	\end{align*}
	Since we assumed $\int v d\mu$ to be finite, we can subtract it from both sides and the lemma follows.
\end{proof}

Note how this time we had to make sure that the sequence was bounded by a function $v$ whose integral was finite. This was implicit in \autoref{thm:fatou-lemma}, only in this case the bounding is from below and the bounding function is $v = 0$.