\documentclass[a4paper]{book}

%\usepackage[utf-8]{inputenc}

\usepackage{amsmath}
\usepackage{amsfonts}
\usepackage{amssymb}
\usepackage{amsthm}
\usepackage{MnSymbol}
\usepackage{bbm}


\usepackage[dvipsnames]{xcolor}
\usepackage{thmtools}

\usepackage{hyperref}
\hypersetup{
	pdftitle={Notes on Measure and Integration},
	pdfauthor={Elias Hernandis},
	pdfpagemode=UseOutlines,
	bookmarksnumbered
}

\declaretheoremstyle[
bodyfont=\normalfont,
shaded={
	margin=8pt,
	bgcolor=White,
	rulecolor=Black,
	rulewidth=1pt
}]{mythm}


\declaretheoremstyle[
bodyfont=\normalfont,
shaded={
	margin=1em,
	bgcolor={rgb}{0.9,0.9,0.9}
}]{mydfn}

\declaretheoremstyle[
bodyfont=\normalfont,
spacebelow=1em,
spaceabove=1em,
]{myeg}

% DEFINICIONES DE ENTORNOS DE TEOREMAS
\declaretheorem[
name=Theorem,
refname={theorem,theorems},
Refname={Theorem,Theorems},
style=mythm,
numberwithin=chapter
]{thm}

\declaretheorem[
name=Lemma,
refname={lemma,lemmas},
Refname={Lemma,Lemmas},
style=myeg,
sibling=thm
]{lem}

\declaretheorem[
name=Remark,
refname={remark,remarks},
Refname={Remark,Remarks},
style=myeg,
sibling=thm
]{remark}

\declaretheorem[
name=Corollary,
refname={corollary,corollaries},
Refname={Corollary,Corollaries},
style=myeg,
sibling=thm
]{cor}


\declaretheorem[
name=Definition,
refname={definition,definitions},
Refname={Definition,Definitions},
style=mydfn,
numberwithin=chapter
]{dfn}

\declaretheorem[
name=Example,
refname={example,examples},
Refname={Example,Examples},
style=myeg,
numberwithin=chapter
]{eg}

\declaretheorem[
name=Exercise,
refname={exercise,exercises},
Refname={Exercise,Exercises},
style=myeg,
numberwithin=section
]{ex}



\renewcommand{\complement}{c}
\newcommand{\sa}{\mathcal{A}}
\newcommand{\abs}[1]{\left\vert{#1}\right\vert}
\newcommand{\norm}[1]{\left\Vert{#1}\right\Vert}
\newcommand{\sigas}{$\sigma$-algebras }
\newcommand{\siga}{$\sigma$-algebra }
\newcommand{\sigfin}{$\sigma$-finite}
\newcommand{\borel}{\mathcal{B}}
\newcommand{\N}{\mathbb{N}}
\newcommand{\Q}{\mathbb{Q}}
\newcommand{\R}{\mathbb{R}}
\newcommand{\B}{\mathbb{B}}
\newcommand{\calD}{\mathcal{D}}
\newcommand{\calG}{\mathcal{G}}
\newcommand{\calF}{\mathcal{F}}
\newcommand{\calC}{\mathcal{C}}
\newcommand{\calJ}{\mathcal{J}}
\newcommand{\calS}{\mathcal{S}}
\newcommand{\call}{\ell}
\newcommand{\calL}{\mathcal{L}}
\newcommand{\calB}{\mathcal{B}}
\newcommand{\calN}{\mathcal{N}}
\newcommand{\calM}{\mathcal{M}}
\newcommand{\calE}{\mathcal{E}}
\newcommand{\calI}{\mathcal{I}}
\newcommand{\tif}{&\text{ if }}
\newcommand{\Pro}{\mathbb{P}}
\newcommand{\istable}{$\cap$-stable}
\newcommand{\sr}{\mathcal{S}}
\newcommand{\muae}{\ \mu\text{-a.e. }}
\newcommand{\OS}{\mathcal{O}}
\newcommand{\CS}{\mathcal{C}}
\newcommand{\KS}{\mathcal{K}}
\newcommand{\inv}[1]{{#1}^{-1}}
\newcommand{\ind}{\mathbbm{1}}
\newcommand{\Rb}{{\overline{\R}}}
\newcommand{\sgn}{\text{sgn}}

\author{Elias Hernandis}
\title{Notes on Measure and integration}

\begin{document}
	\maketitle
	
	\section{Acknowledgements}
	
	These notes are based on the lectures of Karma Dajani (k.dajani1@uu.nl) during Fall of 2019 at Universiteit Utrecht. These lectures are in turn based on the book Measures, Integrals and Martingales by René L. Schiling (2nd edition).
	
	Readers are asked to report errata to eliashernandis@gmail.com.
	
	\section{Why these notes?}
	
	As mentioned before, these notes are based on lectures which are themselves based on a book. In fact, the lectures are almost one-to-one to the book, and there are handwritten notes available online.
	
	However, the process of writing these, as opposed to simply taking notes in class allows me to make sure I understand the connections between the material. In other courses, where the contents were not as well organised as in this one, I had to reorganise the contents of the corresponding version of the notes for that course. Even though this is mostly unnecessary in this course, the handwritten lecture notes allow for more formal proofs or verifications where there was no time to complete them. Also, the book suggest that the reader verify some assumptions made during the proofs of some remarks or lemmas. Hence this notes are most valuable as a complement to the book.
	
	Another valuable thing about these notes are the solutions to the assignments that we were given during the course. These have been compiled from corrections given to me by the TAs\footnote{If you want to see what a disaster it was when I started this subject, look at the scanned copies of submitted exercises...}.
	
	\section{Recommendations}
	
	The official course page is \cite{MIEducat}. In addition, we were offered exercise sets and more material through Blackboard. Grades were also published there.
	
	Always trust the book when there is a discrepancy. Though there may be errata (in fact, see \cite{MIMerrata} for a list of them), it is way more likely that I've made a mistake.
	
	The solutions to problems of the second edition are available online, see \cite{MIM2S}. All problems from the first edition are included in the second edition, albeit with a different numbering. Order is preserved, though.
	
	
	\tableofcontents
	
	% !TeX root = ../mi-notes.tex

\chapter{$\sigma$-algebras}

\section{What is a \siga?}

\begin{dfn}[Sigma algebra]
	Let X be a set and $\sa \subset \powerset(X)$ a collection of subsets of $X$. We say that $\sa$ is a \underline{sigma algebra} or \siga (on X) if it satisfies these three properties:
	\begin{enumerate}
		\item It contains the set, $X \in \sa$;
		\item it is closed under set complement $A \in \sa \implies X \setminus A = A^\complement \in \sa,\ \forall A \subset X$, and
		\item it is closed under \textbf{countable}\footnote{In these notes we shall follow the convention from \cite[bottom of p. 7]{schilling2017} and use countable to describe any set whose cardinality is less than or equal to that of the set of the natural numbers, i.e. $\# A \leq \# \N$. This means that countable does not necessarily mean infinite.} union $(A_n)_{n \in \N} \in \sa \implies \bigcup_{n \in \N} A_n \in \sa$.
	\end{enumerate}
\end{dfn}

\begin{dfn}[Measurable space]
	A \underline{measurable space} is a pair $(X, \sa)$ where $X$ is a set and $\sa$ is a \siga.
\end{dfn}

From now on we shall assume $X$ is a set and $\sa$ is a \siga on $X$.

\begin{remark}$ $ \newline
	\begin{itemize}
		\item From properties 1 and 2 we have $\emptyset = X^\complement\in \sa$.
		\item From properties 2 and 3 we have that $\sa$ is also closed under \textbf{countable} intersection. Let $(A_n)_{n \in \N} \subset \sa$. Then, using De Morgan's laws we have
		\begin{align*}
			\bigcap_{n \in \N} A_n = \left(\bigcup_{n \in \N} A_n^\complement\right)^\complement \in \sa
		\end{align*}
		because $A_n^\complement \in \sa$ because of 2, the union is included because of 3 and its complement is also included because of 2.
	\end{itemize}
\end{remark}

\begin{eg}[First examples of \sigas]$ $\newline
	\begin{enumerate}
		\item The power set $\powerset (X)$ is the largest \siga on $X$.
		\item $\{\emptyset, X\}$ is the smallest \siga on $X$.
		\item Given a subset $A \subset X$, the smallest \siga which contains information about $A$ is $\{\emptyset, X, A, A^\complement\}$. More on this later.
		\item Let $X$ be an uncountable set. Then $\sa = \{ A \subset X \mid A \text{ is countable or } A^\complement \text{ is coutable }\}$ is a \siga.
		
		\begin{proof} We shall prove the three properties of a \siga
			\begin{enumerate}
				\item $X \in \sa$ because $X^\complement = \emptyset$ is countable.
				\item If $A \in \sa$ and is countable the $A^\complement \in \sa$ because $(A^\complement)^\complement$ is countable. If $A \in \sa$ but $A$ is not countable then $A^\complement$ must be so, hence $A^\complement \in \sa$.
				\item Let $(A_n)_{n \in \N} \subset \sa$. To see if the union is included we distinguish to cases:
				\begin{enumerate}
					\item If $A_n$ is countable $\forall n \in N$ then $\bigcup_{n \in \N} A_n$ is also countable and therefore in $\sa$.\footnote{Recall that the countable union of countable sets is still a countable set.}
					\item If there is any $A_k$ that is uncountable, then the union is uncountable but
					\begin{align*}
						\left(\bigcup_{n \in \N} A_n\right)^\complement = \bigcap_{n \in \N} A_n^\complement \subset A_k^\complement
					\end{align*}
					Recall that $A_k^\complement$ must be countable because $A_k \in \sa$ and therefore $\left(\bigcup_{n \in \N} A_n\right)^\complement$ must also be countable. Thus, $\bigcup_{n \in \N} A_n \in \sa$.
				\end{enumerate}
			\end{enumerate}
		\end{proof}
		\item (The preimage\footnote{Recall some properties of preimages (that are not always true for images): Set difference and union (and therefore complement and intersection) are well defined and behave as expected. Moreover, union and intersection behave as expected even for countable arities.} \siga). Let $X, X'$ be sets, let $\sa'$ be a \siga on $X'$ and let $f : X \to X'$ be a function between the two sets. We claim that\footnote{Here $\inv{f}$ denotes the preimage of a set, not the inverse function. Recall that the preimage of a function $f: X \to X'$ is defined as $\inv{f}(A') = \{x \in X \mid f(x) \in A' \subset X'\}$} $\sa = \{\inv{f} (A') \mid A' \in \sa'\}$ is a \siga.
		\begin{proof}
			\begin{enumerate}
				\item $X = \inv{f}(X') \in \sa$ because $X' \in \sa$.
				\item If $A \in \sa$ then $A^\complement = X \setminus A = \inv{f}(X')\setminus \inv{f} (A') = \inv{f}(X' \setminus A') \in \sa$ because $X' \setminus A' = A'^\complement \in \sa'$.
				\item Let $(A_n)_{n \in \N} \subset \sa$. By definition there must be a collection $(A_n')_{n \in \N} \subset \sa'$ for which $A_n = \inv{f}(A_n'),\ \forall n \in \N$. Recall that
				\begin{align*}
					\bigcup_{n \in \N} A_n = \bigcup_{n \in \N} \inv{f}(A_n') = \inv{f} (\bigcup_{n \in \N} A_n') \in \sa
				\end{align*} because $\sa'$ is a \siga hence $\bigcup_{n \in \N} A_n' \in \sa'$.
			\end{enumerate}
		\end{proof}
	\end{enumerate}
\end{eg}

Now we will explore a result which will allow us to generate more examples from existing ones and clarify operations between \sigas.

\begin{thm}
	Given a set $X$. The arbitrary intersection $\bigcap_{\alpha \in I} \sa_\alpha$ of \sigas is again a \siga.
\end{thm}

\begin{proof}
	We shall go over the three properties of \sigas.
	\begin{enumerate}
		\item $X \in \sa_\alpha,\ \forall \alpha \in I$ hence $X \in \bigcap_{\alpha \in I} \sa_\alpha$
		\item If $A \in \bigcap_{\alpha \in I} \sa_\alpha$ then in particular $A \in \sa_\alpha,\ \forall \alpha \in I$. Because each $\sa_\alpha$ is a \siga, then $A^\complement \in A_n,\ \forall \alpha \in I \implies A_n^\complement \in \bigcap_{\alpha \in I} \sa_\alpha$.
		\item If $(A_n)_{n\in \N} \subset \bigcap_{\alpha \in I} \sa_\alpha$ then, as before, we have that $(A_n)_{n \in \N} \subset A_n,\ \forall \alpha \in I$. Hence, $\bigcup_{n \in \N} A_n \in \sa_\alpha \implies \bigcup_{n \in \N} A_n \in \bigcap_{\alpha \in I} \sa_\alpha$.
	\end{enumerate}
\end{proof}


\begin{dfn}[\siga generated by a collection of subsets]
	Let $X$ be a set and $\calG$ a collection of subsets of $X$. We denote by $\sigma(\calG)$ the smallest \siga which contains $\calG$ and we define it by
	\begin{align*}
		\sigma(\calG) := \bigcap \calC\text{ where } \calG \subset \calC \land \calC \text{ is a } \sigma\text{-algebra}
	\end{align*}
	We also say that $\calG$ is a generator of $\sigma(\calG)$ or that $\sigma(\calG)$ is generated by $\calG$.
\end{dfn}

\begin{thm}
	For any $\calG \subset \powerset(X)$, $\sigma (\calG)$ exists and is a the smallest \siga containing $\calG$.
\end{thm}

\begin{proof}
	Let $\sa = \sigma(\calG)$. Since $\powerset(X) \supset \calG$ and $\powerset(X)$ is a \siga, the intersection $\sa$ is non-empty and contains $\calG$. Because of the previous result, $\sa$ itself is also a \siga. (So we have existence). Furthermore, $\sa$ is the smallest \siga containing $\calG$ because if there were another \siga $\sa'$ containing $\calG$ it would be included in the intersection hence $\sa \subseteq \sa'$.
\end{proof}

\begin{remark}
	\label{rem:generators}
	$ $\newline
	\begin{enumerate}
		\item If $\calG$ is a \siga, then $\sigma(\calG) = \calG$.
		\item For any $A \in X,\ \sigma(\{A\}) = \{\emptyset, X, A, A^\complement\}$.
		\item Let $\calG, \mathcal{F} \subset \powerset(X)$. If $\calG \subseteq \mathcal{F}$ then $\sigma (\calG) \subseteq \sigma(\mathcal{F})$.
		\begin{proof}
			$\calG \subseteq \mathcal{F} \subseteq \sigma(\mathcal{F})$. So $\sigma(\mathcal{F})$ is a \siga containing $\calG$. But $\sigma(\mathcal{G})$ is the smallest \siga containing $\calG$ therefore $\sigma(\calG) \subseteq \sigma(\mathcal{F})$.
		\end{proof}
	\end{enumerate}
\end{remark}

\section{The Borel \siga on $\R^n$}

In what follows we will use some basic topology concepts. Lets give some names
\begin{itemize}
	\item $\OS^n := \{A \subset \R^n \mid A \text{ is open }\}$,
	\item $\CS^n := \{A \subset \R^n \mid A \text{ is closed }\}$, and
	\item $\KS^n := \{A \subset \R^n \mid A \text{ is compact }\}$.
\end{itemize}

The collection $\OS^n$ is a topology, meaning it satisfies the following properties:
\begin{enumerate}
	\item $\emptyset, \R^n \in \OS^n$
	\item It is closed under finite intersections, i.e. $V, W \in \OS^n \implies V \cap W \in \OS^n$.
	\item It is closed under arbitrary unions, i.e.
	\begin{align*}
		(A_\alpha)_{\alpha \in I} \in \OS^n \implies \bigcup_{\alpha \in I} A_\alpha \in \OS^n.
	\end{align*}
\end{enumerate}

We shall call the pair $(\R^n, \OS^n)$ a topological space. We now consider the smallest \siga containing $\OS^n$.

\begin{dfn}[Borel \siga]
	The \underline{Borel \siga} on $\R^n$ is the smallest \siga containing $\OS^n$. We denote it by $\sigma(\OS^n)$ or by $\borel(\R^n)$.
\end{dfn}

\begin{thm}
	\begin{align*}
		\borel(\R^n) := \sigma(\OS^n) = \sigma(\CS^n) = \sigma(\KS^n)
	\end{align*}
\end{thm}

\begin{proof}
	First, we prove the first equality, i.e. $\sigma(\OS^n) = \sigma(\CS^n)$ by proving mutual inclusion. To show $\sigma(\CS^n) \subset \sigma(\OS^n)$ it is enough to show that $\CS^n \subset \sigma(\OS^n)$ (recall remark \ref{rem:generators}). Let $C \in \CS^n$ be any closed set in $\R^n$. By definition $C^\complement$ is open hence $C^\complement \in \OS^n \subset \sigma(\OS^n)$. Because $\sigma(\OS^n)$ is a \siga it must be true that $(C^\complement)^\complement = C \in \sigma(\OS^n)$. The same holds for the other inclusion.
	
	Now we turn our attention to $\sigma(\CS^n) = \sigma(\KS^n)$. The inclusion $\sigma(\KS^n) \subset \sigma (\CS^n)$ is trivial because every compact set is closed in $\R^n$ (recall remark \ref{rem:generators}). For the other one, it is again enough to show that $\CS^n \in \sigma(\KS^n)$. Let $C \in \CS^n$ and define $C_k := C \cap \overline{B_k(0)}$ which is\footnote{Here $B_r(c_0)$ and $\overline{B_r(c_0)}$ denote the open and closed balls of radius $r$ and centre $c_0$, respectively. Clearly these are both bounded sets.} closed and bounded. By construction $C = \bigcup_{k \in \N} C_k \in \KS^n$ thus $\CS^n \in \sigma(\KS^n)$.
\end{proof}

We would now like to find smaller sets of generators for the Borel \siga on $\R^n$. Let us define the following collections (where $\bigtimes$ denotes de cartesian product of the intervals):


\begin{itemize}
	\item The collection of open rectangles (or cubes or hypercubes)
	\begin{align*}
		\calJ^{o,n} = \{\bigtimes_{i = 1}^n (a_i, b_i) \mid a_i, b_i \in \R\}
	\end{align*}
	\item The collection of (from the right) half-open rectangles
	\begin{align*}
		\calJ^{n} = \{\bigtimes_{i = 1}^n [a_i, b_i) \mid a_i, b_i \in \R\}
	\end{align*}
\end{itemize}

\begin{thm}
	\label{thm:borel-interval-generators}
	We have
	\begin{align*}
		\borel(\R^n) = \sigma(\calJ_{rat}^n) = \sigma(\calJ_{rat}^{o,n}) = \sigma(\calJ^n) = \sigma(\calJ^{o,n})
	\end{align*}
\end{thm}

\begin{proof}
	Let's begin by proving $\borel(\R^n) = \sigma(\calJ^{o,n}_{rat})$. Recall that $\borel(\R^n) = \sigma(\OS^n)$, so to prove the previous equality it suffices to prove the following two mutual inclusions:
	\begin{itemize}
		\item $\sigma(\calJ^{o,n}_{rat}) \subseteq \sigma(\OS^n)$. From remark \ref{rem:generators} we have that to prove this it suffices to say that every open rectangle is an open set and thus $\calJ_{rat}^{o,n} \subset \OS^n \implies \sigma(\calJ^{o,n}_{rat}) \subseteq \sigma(\OS^n)$.
		\item $\sigma(\OS^n) \subseteq \sigma(\calJ^{o,n}_{rat})$. To prove this we make the following claim:
		\begin{align*}
		U \in \OS^n \implies U = \bigcup_{I \in \calJ_{rat}^{o,n},\ I \subseteq U} I
		\end{align*}
		Again we shall attack this by proving the mutual inclusion of the two sets:
		\begin{itemize}
			\item It is clear that $\bigcup_{I \in \calJ_{rat}^{o,n},\ I \subset U} I \subseteq U$ because of the restriction on the union.
			\item For the reverse containment, we have that as $U$ is open, for any $x \in U$ there is a ball $B_\varepsilon(x) \subseteq U$. Because the rationals $\Q^n$ are dense in the reals $\R^n$ we can chose a rectangle $I \subset B_\varepsilon(x)$ and hence $U$ is contained in the union.
			
			It is clear that all the sets $I \subseteq U$ are also in $\sigma(\calJ_{rat}^{o,n})$. However, for the union of them to be inside de \siga we must ensure that the number of sets that participate is countable. Each rectangle $I$ can be fully determined by two of its corners, which in turn have coordinates in $\Q^n$. Therefore, the number of sets intervening in the union is $\# (\Q^n \times \Q^n) = \# \N$ and thus the union is again within the \siga.
		\end{itemize}
	\end{itemize}
	
	Because $\calJ_{rat}^{o,n} \subset \calJ^{o,n} \subset \OS^n$ we get for free that $\sigma(\calJ_{rat}^{o,n}) \subseteq \sigma(\calJ^{o,n}) \subseteq \sigma(\OS^n)$. We therefore conclude that
	\begin{align*}
	\sigma(\calJ_{rat}^{o,n}) = \sigma(\calJ^{o,n}) = \sigma(\OS^n)
	\end{align*}
	
	Now we would like to prove that half open sets also yield the same Borel \siga for $\R^n$.
	\begin{itemize}
		\item We begin by noticing that we can write open sets as infinite unions of half open ones
		\begin{align*}
		\bigtimes_{i = 1}^n (a_i , b_i) = \bigcup_{n \in \N} \bigtimes_{i = 1}^n [a_i + \frac{1}{n}, b_i)
		\end{align*}
		for both rectangles with rational and real endpoints. Thus, we have
		\begin{align*}
		\calJ_{rat}^{o,n} \subseteq \sigma(\calJ_{rat}^n) &\implies \sigma(\calJ_{rat}^{o,n}) \subseteq \sigma(\calJ_{rat}^n) \\
		\calJ^{o,n} \subseteq \sigma(\calJ^n) &\implies \sigma(\calJ^{o,n}) \subseteq \sigma(\calJ^n)
		\end{align*}
		(remember that \sigas are closed under countable unions).
		\item For the reverse containment we must notice that we can write (right) half-open sets as intersections of open ones
		\begin{align*}
		\bigtimes_{i = 1}^n [a_i, b_i] = \bigcap_{n \in \N} \bigtimes_{i = 1}^n (a_i - \frac{1}{n}, b_i)
		\end{align*}
		for rectangles with both rational and real endpoints. Similarly, we have
		\begin{align*}
		\calJ_{rat}^n \subseteq \sigma(\calJ_{rat}^{o,n}) &\implies \sigma(\calJ_{rat}^n) \subseteq \sigma(\calJ_{rat}^{o,n}) \\
		\calJ^n \subseteq \sigma(\calJ^{o,n}) &\implies \sigma(\calJ^n) \subseteq \sigma(\calJ^{o,n})
		\end{align*}
		\item We conclude that
		\begin{align*}
		\sigma(\calJ_{rat}^{o,n}) &= \sigma(\calJ_{rat}^n) \\ \sigma(\calJ^{o,n}) &= \sigma(\calJ^n)
		\end{align*}
	\end{itemize}
	With the previous equalities the theorem has been proved.
\end{proof}

To recap, there are a few important points on this proof:

\begin{itemize}
	\item First, we would like to have a more tangible generator for the Borel \siga on $\R^n$.
	\item We choose rectangles because they are easier to work with and have direct application on probability theory (we could also have chosen balls, for instance).
	\item They key to proving that two \sigas are the equal is to prove that each is contained in the other. To do this, we use the generators: if the generator of $\sa$ is contained in $\sigma(\sa')$ and the generator of $\sa'$ is contained in $\sigma(\sa)$, we are done.
	\item To prove this containments we have had to write sets from the generator as unions of sets from the other \siga. It is key to make sure that these unions only iterate over a countable number of elements.
\end{itemize}

\begin{eg}[Another characterisation of the Borel \siga on $\R^n$]
	In this example we shall see that $\B = \{B_r(x) \mid x \in \R^n, r \in \R^+\}$ is also a generator of $\borel(\R^n)$.
\end{eg}

\begin{proof}
	We proceed as before, first defining an auxiliary collection where the radii are all rational and the centres have rational coordinates:
	\begin{align*}
	\B' = \{B_r(x) \mid x \in \Q^n, r \in \Q^+\}
	\end{align*}
	It is trivial that $\B' \subset \B \subset \OS^n$. Therefore we have that
	\begin{align*}
	\sigma(\B') \subseteq \sigma(\B) \subseteq \sigma(\OS^n) = \borel(\R^n)
	\end{align*}
	For the reverse inclusions, we will focus on proving that $\OS^n \subseteq \sigma(\B')$. For this we claim that any open set $U \in \OS^n$ can be written as
	\begin{align*}
	U = \bigcup_{B \in \B', B \subseteq U} B
	\end{align*}
	We need to verify two things. That the previous equality is true and that the number of sets that intervene in the union is countable.
	\begin{enumerate}
		\item It is clear that any set $B\subset \B'$ is also in $U$ by the definition of the union. For the reverse inclusion, we shall choose a point $q \in \Q^n$ such that $\lVert x - q \rVert < r / 3$. This is possible because $\Q^n$ is dense in $\R^n$. Next we will choose a radius $r' \in \Q$ such that $r' < r$. This is also possible for the same reason. Now we consider $B = B_{r'}(q) \in \B'$ which is assured to contain $x$.
		
		\item Moreover, each of these balls is fully determined by an $(n+1)$ tuple of rationals (namely the center coordinates and the radius). Hence, the number of sets in the union is $\#(\Q^n \times \Q) = \# \N$.
	\end{enumerate}
\end{proof}
	% !TeX root = ../mi-notes.tex

\chapter{Measures}

\section{Definition. Properties.}

\begin{dfn}[Measure]
	\label{dfn:measure}
	Let $(X, \sa)$ be a measurable space. A set function $\mu: \sa \to [0, \infty)$ is called a \underline{measure} (on X) if
	\begin{enumerate}
		\item $\mu(\emptyset) = 0$ and
		\item \textbf{($\sigma$-aditivity)} if $(A_n)_{n\in \N} \subset \sa$ is a pairwise disjoint (i.e. $A_i \cap A_j = \emptyset$ if $i \neq j$) sequence, then
		\begin{align*}
			\mu\left(\bigcupdot_{n \in \N} A_n\right) = \sum_{n\in \N} \mu(A_n)
		\end{align*}
	\end{enumerate}
\end{dfn}

And, analogously to \sigas

\begin{dfn}[Measure space]
	Let $(X, \sa)$ be a measurable space and $\mu$ a measure. The triple $(X, \sa, \mu)$ is called a \underline{measure space}.
\end{dfn}

In the definition of measure (\ref{dfn:measure}), the symbol $\bigcupdot$ indicates that the union is disjoint.

Some special measures get cool names, for instance:

\begin{itemize}
	\item If $\mu(X) < \infty$ we say that $\mu$ is \textbf{finite} and that $(X, \sa, \mu)$ is a \textbf{finite measure space}.
	\item IF $\mu(X) = 1$ then $(X, \sa, \mu)$ is a probability space and we usually denote it by $(\Omega, \sa, \Pro)$.
\end{itemize}

From the two conditions on \ref{dfn:measure}, one can derive many properties, but before, let us introduce some notation.

Let $(A_n)_{n \in \N}$ be a sequence of sets. We say that $(A_n)$ is an \textbf{increasing}, resp. \textbf{decreasing}, \textbf{sequence} and denote it by $A_n \uparrow A$, resp. $A_n\downarrow A$ according to the following definition.
\begin{align}
	A_n \uparrow A \iff A_1 \subseteq A_2 \subseteq \dots &\text{ and } A = \bigcup_{n \in \N} A_n \\
	A_n \downarrow A \iff A_1 \supseteq A_2 \supseteq \dots &\text{ and } A = \bigcap_{n \in \N} A_n
\end{align}

We say that an increasing sequence $A_n \uparrow A$ is an \textbf{exhausting sequence} if $A_n \subseteq X$ and $A = X$. We write $A_n \uparrow X$ for an exhausting sequence.


\begin{dfn}[$\sigma$-finite measure]
	\label{dfn:sigma-finite}
	A measure $\mu$ is called $\sigma$-finite if there exists an exhausting sequence $(A_n)_{n \in \N} \subset \sa$ such that $\mu(A_n) < \infty,\ \forall n \in \N$. A measure space with this kind of measure is called a $\sigma$-finite measure space.
\end{dfn}

\begin{remark}
	Finiteness is stronger than $\sigma$-finiteness\footnote{By $\sigma$-finiteness we mean that a measure satisfies \autoref{dfn:sigma-finite}, not that it is $s$-finite, which we wont see in this course. See \cite{WSF} for more details.}. Namely, any finite measure is $\sigma$-finite since one can choose the sequence $A_n = X,\ \forall n \in \N$ which is an exhausting sequence $A_n\uparrow X$ and $\mu(A_n) < \infty,\ \forall n \in \N$. On the other hand, an example of a $\sigma$-finite measure which is not finite is the Lebesuge measure (see \autoref{eg:lebesgue-measure}).
\end{remark}

\begin{thm}[Properties of measures]
	\label{thm:prop-measures}
	Let $(X, \sa, \mu)$ be a measure space and $A, B, A_n, B_n \in \sa,\ \forall n \in \N$. Then,
	\begin{enumerate}
		\item \textbf{(finite additivity)} $A \cap B = \emptyset \implies \mu(A \cupdot B) = \mu(A) + \mu(B)$,
		\item \textbf{(monotonicity)} if $A \subseteq B$ then $\mu(A) \leq \mu(B)$,
		\item if $A \subset B$ and $\mu(A) < \infty$ then $\mu(B\setminus A) = \mu(B) - \mu(A)$,
		\item \textbf{(strong additivity)} $\mu(A \cup B) + \mu(A \cap B) = \mu(A) + \mu(B)$,
		\item \textbf{(finite subadditivity)} $\mu(A \cup B) \leq \mu(A) + \mu(B)$,
		\item \textbf{(continuity from below)} if $A_n \uparrow A$ then $\mu(A) = \mu(\bigcup_{n \in \N} A_n) = \sup_{n\in\N} \mu(A_n) = \lim_{n\to \infty} \mu(A_n)$,
		\item \textbf{(continuity from above)} if $\mu(A) < \infty, \forall A \in \sa$ and $A_n \downarrow A$ then $\mu(A) = \mu(\bigcap_{n \in \N} A_n) = \inf_{n\in\N} \mu(A_n) = \lim_{n\to \infty} \mu(A_n)$, and
		\item \textbf{(sigma subadditivity)} $\mu(\bigcup_{n \in \N} A_n) \leq \sum_{n\in \N} \mu(A_n)$.
	\end{enumerate}
\end{thm}

\begin{proof}$ $\newline
	\begin{enumerate}
		\item Let $(A_n)_{n\in\N} \subset \sa$ with $A_1 = A,\ A_2 = B$ and $A_i = \emptyset$ for $i > 2$. It is clear that $A_n$ are disjoint since $A\cap B = \emptyset$ and $A_n \cap \emptyset = \emptyset,\ \forall n \in \N$.
		\item Write $B = (B\setminus A) \cupdot A$. Then, because of $\sigma$-aditivity we have
		\begin{align*}
		\mu(B) = \mu(B\setminus A) + \mu(A) \geq \mu(A) \text{ since } \mu(B\setminus A) \geq 0
		\end{align*} 
		\item As previously write $\mu(B) = \mu(B\setminus A) + \mu(A)$. Because $\mu(A) < \infty$ we can subtract it on both sides to get $\mu(B) -\mu(A) = \mu(B\setminus A)$.
		\item Write $A \cup B = A \setminus B \cupdot B \setminus A \cupdot A \cap B$ hence $\mu(A \cup B) = \mu(A \setminus B) + \mu(B \setminus A) + \mu(A \cap B)$. Add $\mu(A \cap B)$ on both sides and group terms:
		\begin{align*}
		\mu(A \cup B) + \mu(A \cap B) = \underbrace{\mu(A \setminus B) + \mu(A \cap B)}_{\mu(A)} + \underbrace{\mu(B \setminus A) + \mu(A \cap B)}_{\mu(B)}
		\end{align*}
		\item $\mu(A\cap B) \leq \mu(A \cap B)+ \mu(A \cup B) = \mu(A) + \mu(B)$
		\item Define the sequence $(B_n)_{n\in\N} \subset \sa$ by $B_1 = A_1$, $B_n = A_n\setminus A_{n-1}$ for $n > 1$. It is clear that $B_n$ is pairwise disjoint and that $\bigcupdot_{n \in \N} B_n = \bigcup_{n \in \N} A_n = A$. Hence
		\begin{align*}
		\mu(A) &= \mu(\bigcup_{n \in \N} A_n) = \mu(\bigcupdot_{n \in \N} B_n) = \sum_{n\in \N} \mu(B_n) = \lim_{m \to \infty} \sum_{n = 1}^m\mu(B_n) \\
		&= \lim_{m \to \infty} \mu (\bigcupdot_{n = 1}^m B_n) = \lim_{m \to \infty} \mu (A_m) = \sup_{n\in\N} \mu(A_n)
		\end{align*}
		since $\mu(A_n)$ is an increasing sequence. The introduction of the limits in the previous chain of equalities has to be done carefully, as we are building on the definition of limits for sequences of numbers. The equality between the first and the second lines comes from $\sigma$-additivity.
		\item Let $D_n = A_1 \setminus A_n,\ \forall n \in \N$. Then $D_N$ is an increasing sequence with
		\begin{multline*}
		\bigcup_{n\in\N} D_n = \bigcup_{n \in \N} (A_1\setminus A_n) = \bigcup_{n \in \N} (A_1\cap A_n^\complement) = A_1 \cap \bigcup_{n \in \N} A_n^c \\
		= B_1 \cap \left(\bigcap_{n \in \N} A_n\right)^\complement = A \setminus \bigcap_{n\in\N} A_n
		\end{multline*}
		Thus,
		\begin{multline*}
		\mu(A \setminus \bigcap_{n\in\N} A_n) \overset{(3)}{=} \mu(A) - \mu(\bigcap_{n\in\N} A_n) = \mu(\bigcup_{n \in \N} D_n) \\
		\overset{(4)}{=} \lim_{n\to \infty}\mu(D_n) = \lim_{n\to \infty} \mu(A_1 \setminus A_n) \overset{(3)}{=} \lim_{n\to \infty} (\mu(A) - \mu(A_n)) = \mu(A) - \lim_{n\to \infty} \mu(A_n)
		\end{multline*}
		Substracting $\mu(A) <\infty$ from both sides we have
		\begin{align*}
		\mu(\bigcap_{n \in \N} A_n) = \lim_{n\to \infty} \mu(A_n)
		\end{align*}
		\item Let $(A_n)_{n\in\N}$ be any countable subcollection of $\sa$. Define
		\begin{align*}
		E_n = \bigcup_{m = 1}^n A_m \in \sa
		\end{align*}
		Then $E_n \uparrow \bigcup_{n\in\N} E_n = \bigcup_{n\in\N} A_n$. Thus, by (6) we have
		\begin{align*}
		\mu(\bigcup_{n\in\N} A_n) = \mu(\bigcup_{n\in\N} E_n) = \lim_{n\to \infty} \mu(E_n) = \lim_{n\to \infty} \mu\left(\bigcup_{m = 1}^n A_m\right) \overset{(5)}{\leq}\lim_{n\to \infty} \sum_{m=1}^{n}\mu(A_m)
		\end{align*}
	\end{enumerate}
\end{proof}


\section{Examples}

Throughout this section, let $(X, \sa)$ be a measurable space.

\begin{eg}[Dirac measure]
	\label{eg:dirac-measure}
	We define the \textbf{Dirac measure} for a given $x_0 \in X$ as follows:
	\begin{align}
	\delta_{x_0}(A) = \begin{cases}
	0 \tif x_0\not\in A \\
	1 \tif x_0 \in A
	\end{cases},\qquad \forall A \in \sa
	\end{align}
	Clearly $\delta_{x_0}$ is a measure since it satisfies the two properties. First, $\delta_{x_0}(\emptyset) = 0$ since $\forall x_0 \in X,\ x_0 \not\in \emptyset$. Second, for any pairwise disjoint collection of sets $(A_n)_{n\in\N} \subset \sa$ we have two possibilities:
	\begin{itemize}
		\item If $x_0 \not\in \bigcupdot_{n \in \N} A_n$ then clearly $x_0 \not\in A_n,\ \forall n \in N$ so
		\begin{align*}
		0 = \mu\left(\bigcupdot_{n \in \N} A_n\right) = \sum_{n\in \N} \mu(A_n) = 0
		\end{align*}
		\item Otherwise, if $x_0 \in \bigcupdot_{n \in \N} A_n$ then there must be only one $n_0 \in \N$ such that $x_0 \in A_{n_0}$ since $(A_n)$ is a pairwise disjoint collection. Thus,
		\begin{align*}
		1 = \mu\left(\bigcupdot_{n \in \N} A_n\right) = \sum_{n\in \N} \mu(A_n) = \mu(A_{n_0}) + \sum_{n \neq n_0} \mu(A_n) = 1 + 0
		\end{align*}
	\end{itemize}
\end{eg}

\begin{eg}[Counting measure]
	Let $X = \R$ and choose $\sa = \{A \subset \R \mid A\text{ is countable or } A^\complement \text{ is countable }\}$. We already saw on the Chapter 1 that $\sa$ is a \siga. Now define $\mu:\sa \to [0, \infty)$ as
	\begin{align}
		\mu(A) = \begin{cases}
		0 \tif \# A \leq \# \N \\
		1 \tif \# A^\complement \leq \# \N
		\end{cases}
	\end{align}
	We have that $\mu(\emptyset) = 0$ since $\#\emptyset$ is countable. As for $\sigma$-additivity we must recall from set theory that the union of countable sets is also countable, so:
	\begin{itemize}
		\item If $\#A_n \leq \#\N,\ \forall n \in \N$ then
		\begin{align*}
			0 = \mu(\bigcupdot_{n \in \N} A_n) = \sum_{n\in \N} \mu(A_n) = 0
		\end{align*} 
		\item If there exists an $n_0 \in \N$ such that $A_{n_0}^\complement$ is countable then
		\begin{align*}
			\left(\bigcupdot_{n \in \N} A_n\right)^\complement = \bigcap_{n \in \N} A_n^\complement \subseteq A_{n_0}^\complement
		\end{align*}
		and hence $\left(\bigcupdot_{n \in \N} A_n\right)^\complement$ is countable. Furthermore, since the collection is pairwise disjoint, $\forall n \in \N, n \neq n_0$ we have $A_n^\complement \subseteq A_{n_0}$ so $\#A_n^\complement,\ \forall n \neq n_0$. Thus
		\begin{align*}
			1 = \mu(\bigcupdot_{n \in \N} A_n) = \sum_{n\in\N} \mu(A_n) = \mu(A_{n_0}) + \sum_{n \neq n_0} \mu(A_n) = 1 + 0
		\end{align*}
	\end{itemize}
\end{eg}

\begin{eg}[Discrete probability measure]
	Let $(\Omega, \sa, \Pro)$ be a measure space where $\Omega = \{\omega_1, \omega_2, \dots \}$ is a countable set, $\sa = \powerset(\Omega)$ (which is of course a \siga) and let $(p_1, p_2, \dots)$ be a probability vector where $\sum_{n\in \N} p_n = 1$ (and $p_i$ is the probability of $\omega_i$). Define the measure $\Pro : \sa \to [0, \infty)$ where
	\begin{align*}
		\Pro(A) = \sum_{\omega_i \in A} p_i = \sum_{i=1}^\infty p_i \delta_{\omega_i}(A)
	\end{align*}
	
	Let's verify that $\Pro$ is a measure. First, $\Pro(\emptyset) = 0$ since $\emptyset$ cannot contain any $\omega_i$. As for $\sigma$-additivity, we have
	\begin{align*}
		\Pro\left(\bigcupdot_{n = 1}^\infty A_n\right) = \sum_{i=1}^\infty p_i \delta_{x_i}\left(\bigcupdot_{n = 1}^\infty A_n\right) = \sum_{i = 1}^\infty p_i \sum_{n = 1}^\infty \delta_{x_i}(A_n) \\
		= \sum_{n = 1}^\infty \left(\sum_{i = 1}^\infty p_i\delta_{x_i} A_n\right) = \sum_{n = 1}^\infty \Pro(A_n)
	\end{align*}
	Nota that here we can exchange the summations because all the terms are non-negative so the convergence problems (oscillating convergence, that is) are eliminated.
\end{eg}

\begin{eg}[Lebesgue measure]
	\label{eg:lebesgue-measure}
	For now we shall define this measure only on $n$-dimensional rectangles. The generalisation to arbitrary Borel sets will come in chapter 4.
	
	Consider the measure space $(\R^n, \borel(\R^n), \lambda^n)$ where $\R^n$ and $\borel(\R^n)$ are the usual suspects and $\lambda^n : \borel(\R^n) \to [0, \infty)$ is defined as:
	\begin{align}
		\lambda^n\left(\bigtimes_{i = 1}^n [a_i, b_i)\right) = \prod_{i=1}^{n} (b_i - a_i)
	\end{align}
	or, more informally, the hypervolume of the rectangle in question.
	
	We are not ready to verify that $\lambda^n$ is a measure over $\borel(\R^n)$ yet. In fact, we will need to wait until the end of Chapter 4, where, once we have the generalisation to arbitrary Borel sets, we shall prove that it is a measure.
	
	We are however, in a position to prove that $\lambda^n$ is $\sigma$-finite. Let $A_i = [-i, i)^n$ for all $n \in \N$. Clearly, $(A_i)_{n \in \N}$ is an exhausting sequence since $A_i \uparrow \R^n$. Also, $\lambda^n(A_i) = (2i)^n < \infty, \forall n \in \N$. Therefore $\lambda^n$ is $\sigma$-finite.
	
	Note that $\lambda^n$ is not finite, since $\lambda(R^n) \not< \infty$.
\end{eg}
	% !TeX root = ../mi-notes.tex

\chapter{Uniqueness of measures}

In this chapter we shall introduce some technicalities to be able to extend the Lebesgue measure to arbitrary Borel sets in $\borel(\R^n)$. The first tool is a generalized version of a \siga, where union closure is only required for disjoint unions. We will explore the properties of this new construct and try to draw similarities to what we know about \sigas. Finally we will give a result on the conditions that a \siga must meet to be able to uniquely define a measure on it by defining the measure on the elements of a generator.

\section{Preliminaries}


\begin{dfn}[Dynkin system]
	A family $\calD \subset \powerset(X)$ is called a Dynkin system if it satisfies these three properties:
	\begin{enumerate}
		\item $X \in \calD$
		\item $\forall A \in \powerset(X),\ A \in \calD \implies A^\complement \in \calD$
		\item For any countable collection of pairwise disjoint sets $(A_n)$
		\begin{align*}
			(A_n)_{n\in\N} \subset \calD \implies \bigcupdot_{n \in \N} A_n \in \calD
		\end{align*}
	\end{enumerate}
\end{dfn}

Note that every \siga is a Dynkin system but the converse need not be true.

We have an analogue version to the generator theorem for \sigas in the case of Dynkin systems:

\begin{dfn}
	Let $\calG \subset \powerset(X)$ be a collection of subsets of $X$. We define the Dynkin system generated by $\calG$ as
	\begin{align*}
		\delta(\calG) = \bigcap_{X \in \calC, \calC \text{Dynkin}} \calC
	\end{align*}
\end{dfn}

\begin{thm}
	Let $\calG \subset \powerset(X)$ be a collection of subsets of $X$. Then $\delta(\calG)$ is a Dynkin system and it is the smallest.
\end{thm}

\begin{proof}
	$\delta(\calG)$ is a Dynkin system because of the restriction on the intersection. Note that the intersection is non-empty because $\powerset(X)$ is a Dynkin system and that the intersection of Dynkin systems is also a Dynkin system (the proof is the same as the one for \sigas but using disjoint unions).
	
	Now let's look at why it is the smallest. Suppose there is another Dynkin system $\calD$ that contains the collection $\calG$. Then $\calD$ is a Dynkin system and therefore intervenes in the intersection. Therefore $\delta(\calG) \subset \calD$.
\end{proof}

So, we already saw that every \siga is a Dynkin system and that Dynkin systems can be generated from a collection of subsets. A natural question to ask is, when is a Dynkin system a \siga? Let us introduce some terminology first.


\begin{dfn}[Stable under finite intersection]
	Let $\calD$ be a collection of sets. We say that $\calD$ is stable under finite intersection, or closed under finite intersection or $\cap$-stable if
	\begin{align*}
	C, D \in \calD \implies C \cap D \in \calD
	\end{align*}
	
	Some sources call such a collection a $\Pi$-system.
\end{dfn}

\begin{lem}
	Let $\calD$ be a Dynkin system. $\calD$ is \istable $\iff \calD$ is a \siga.
\end{lem}

\begin{proof}
	It is clear that going from right to left is true, since all \sigas are \istable.
	
	For the implication from left to right we proceed as follows. Assuming $\calD$ is a \istable Dynkin system, to prove that $\calD$ is a \siga we only need to generalise the last property to countable unions of arbitrary collections of sets in $\calD$, not just disjoint ones. Let $(A_n)_{n\in\N} \subset \calD$ be a sequence of sets in $\calD$. We define a new sequence $(E_n)_{n\in\N}$ by
	\begin{align*}
	E_1 = D_1 \text{ for } n = 1,\qquad E_n = D_n \setminus E_{n-1} \text{ for } n > 1.
	\end{align*}
	Rewriting the set difference as an intersection we have that
	\begin{align*}
	E_n = D_n \setminus E_{n-1} = D_n \setminus \bigcup_{i=1}^{n-1}D_i = D_n \cap \left(\bigcap_{i=1}^{n-1} D_i^\complement\right) \in \calD
	\end{align*}
	since $\calD$ is \istable.
	It is clear that $E_n$ are pairwise disjoint, i.e. $E_i \cap E_j = \emptyset$ for $i \neq j$. We also have that
	\begin{align*}
	\bigcupdot_{n \in \N} E_n = \bigcup_{n \in \N} D_n \in \calD
	\end{align*}
	Hence, $\calD$ is a \siga.
\end{proof}

\begin{remark}
	As a consequence we have that if $\calG \subset \powerset(X)$ and $\delta(\calG)$ is \istable then $\delta(\calG)$ is a \siga containing $\calG$ therefore $\sigma(\calG) \subseteq \delta(\calG)$. Since $\delta(\calG) \subseteq \sigma(\calG)$ always holds, we have that
	\begin{align*}
	\text{ if } \delta(\calG) \text{ is $\cap$-stable then } \delta(\calG) = \sigma(\calG)
	\end{align*}
\end{remark}

This is nice, but it would even be nicer if we could just argue about the generators, since Dynkin systems and \sigas are hard to reason about. We'll do just that.

\begin{thm}
	Let $\calG$ be a collection of subsets of $X$. If $\calG$ is \istable then $\delta(\calG) = \sigma(\calG)$.
\end{thm}

\begin{proof}
	We only need to show that if $\calG$ is \istable then $\delta(\calG)$ also is. Then, because of the previous remark we have $\delta(\calG) = \sigma(\calG)$.
	
	We shall proceed in steps.
	\begin{enumerate}
		\item \textbf{Claim 1.} For each $E \in \delta(\calG)$, the collection $\calD_E = \{ F \subseteq X \mid F \cap E \in \delta(\calG)\}$ is a Dynkin system. We prove the three properties of a Dynkin system.
		\begin{enumerate}
			\item Clearly $\emptyset = \emptyset \cap E \in \calD_E$
			\item For any $F \in \calD_E$ we have $F^\complement = X \setminus F \subseteq X$ and
			\begin{align*}
				 F^\complement \cap E =& (F^\complement \cup E^\complement) \cap E \\
				 &= (F \cap E)^\complement \cap E \\
				 &= (F \cap E) \cupdot E^\complement \in \delta(\calG),
			\end{align*}
			since $F \cap E \in \delta(\calG)$ by hypothesis and $E^\complement \in \delta(\calG)$ since $\delta(\calG)$ is a Dynkin system.
			\item Finally, for any collection of disjoint subsets $(F_n)_{n\in\N} \subset \calD_E$ we need to show that the disjoint union is still in $\calD_E$. We have that for all $n \in \N,\ F_n \cap E \in \delta(\calG)$ by hypothesis so
			\begin{align*}
				\bigcupdot_{n \in \N} F_n \cap E = \bigcupdot_{n\in\N} (F_n \cap E) \in \delta(\calG).
			\end{align*}
		\end{enumerate}
		\item \textbf{Claim 2.} $\calG \subset \calD_G,\ \forall G \in \calG$. Let $G' \in \calG$. Since $\calG$ is \istable we have that $G' \cap G \in \calG$ and hence $G' \cap G \in \delta(\calG) \implies G' \in \calD_G$.
	\end{enumerate}

	As a consequence of these claims we have that, since $\calG \subset \calD_G$ then $\delta(\calG) \subset \delta(\calD_G) = \calD_G$. Therefore, for any $E \in \delta(\calG)$ and any $G \in \calG$ we have that $E \cap G \in \delta(\calG)$. This also shows that $\delta(\calG) \subset \calD_E$, for any $E \in \delta(\calG)$. In other words we have that for any $E, F \in \delta(\calG),\ E \cap F \in \delta(\calG)$. Thus, $\delta(\calG)$ is \istable implying that $\delta(\calG)$ is a \siga and hence $\delta(\calG) = \sigma(\calG)$.
\end{proof}

\section{Uniqueness of measures}

Now we move on to define the requirements needed to be able to guarantee uniqueness of a measure over a \siga given a definition of the measure on a generator of that \siga.

\begin{thm}[Uniqueness of measures]
	Let $(X, \sa)$ be a measurable space where $\sa = \sigma(\calG)$ for some collection $\calG$ of subsets of $X$ where $\calG$ satisfies the following:
	\begin{enumerate}
		\item $\calG$ is \istable (so $\delta(\calG) = \sigma(\calG)$), and
		\item there exists an exhausting sequence $(G_n)_{n \in \N} \subset \calG$ such that $G_n\uparrow X$ (so $X = \bigcup_{n\in\N}G_n$). 
	\end{enumerate}

	If $\mu, \nu$ are measures on $\sa = \sigma(\calG)$ such that $\mu(G) = \nu(G) < \infty,\ \forall G \in \calG$ then $\mu = \nu$, i.e. $\mu(A) = \nu(A),\ \forall A \in \sa$.
\end{thm}

\begin{proof}
	TODO
\end{proof}

Suppose that we don't have an exhausting sequence on the generator. A neat trick (that doesn't always work) is to extend the generator to $\calG \cup \{X\}$ and define the trivial exhausting sequence $G_n = X,\ \forall n \in \N$. If it holds that $\mu(X) < \infty$, i.e., if $\mu$ is a finite measure, then we are in a position to apply this theorem. See exercise 5.9 for an opportunity to apply this trick.

\begin{thm}$ $\newline
	\begin{enumerate}
		\item The $n$-dimensional Lebesgue measure $\lambda^n$ is invariant under translations, i.e.
		\begin{align*}
			\lambda^n(x + B) = \lambda^n(B),\quad \forall x \in \R`n, B \in \borel(\R^n),
		\end{align*}
		where $x + B = \{x + y \mid y \in B\}$.
		
		\item Every measure $\mu$ on $(\R^n, \borel(\R^n))$ which is invariant under translations and satisfies $\kappa = \mu([0,1]^n) < \infty$ is a multiple of the Lebesgue measure $\mu = \kappa \lambda^n$.
	\end{enumerate}
\end{thm}

\begin{proof}$ $\newline
	\begin{enumerate}
		\item First of all, let us check that $\lambda(x + B)$ is well defined, i.e. that $B \in \borel(\R^n) \implies x + B \in \borel(\R^n)$. There is a clever way to do this. Define
		\begin{align*}
			\sa_x = \{B \in \borel(\R^n) \mid x + B \in \borel(\R^n)\} \subset \borel(\R^n).
		\end{align*}
		It is clear that $\sa_x$ is a \siga on $\R^n$ and that $\calJ \subset \sa_x$ since translations of half-open intervals are still half-open intervals and hence in $\calJ$ and thus in $\borel(\R^n)$. Therefore $\borel(\R^n) = \sigma(\calJ) \subset \sa_x \subset \borel(\R^n)$. Now we can start with the meat of the proof.
		
		Define $\nu(B) = \lambda^n(x + B)$ for any $B \in \borel(\R^n)$ and some fixed $x =(x_1, \dots, x_n) \in \R^n$. $\nu$ is a measure on $(\R^n, \borel(\R`n))$ since
		\begin{align*}
			\nu(\emptyset) = \lambda^n(x + \emptyset) = \lambda^n\left(\bigtimes_{i = 1}^n [x_i + a, x_i + a)\right) = \prod_{i=1}^n (x_i + a - (x_i + a)) = 0\\
			\text{ and }\nu\left(\bigcupdot_{j = 1}^\infty A_j \right) = \lambda^n\left(\bigcupdot_{j = 1}^\infty x + A_j\right) = \sum_{j = 1}^\infty \lambda^n (x + A_j) = \sum_{j = 1}^\infty \nu(A_j).
		\end{align*}
		Now take $I = \bigtimes_{i = 1}^n [a_i, b_i] \in \calJ$ and note that
		\begin{align*}
			\nu(I) = \lambda^n(x + I) &= \lambda^n \left(\bigtimes_{i = 1}^n [a_i + x_i, b_i + x_i]\right) \\
			&= \prod_{i=1}^n (b_i + x - (a_i + x)) = \prod_{i=1}^n (b_i - a_i) = \lambda^n(I)
		\end{align*}
		
		This means that, if we restrict ourselves to the generator $\calJ$, we have that $\nu\mid_\calJ = \lambda^n\mid_\calJ$.\footnote{Where $f\mid A$ denotes the restriction of $f: X \to Y$ to the new domain $A \subset X$.} Recall that $\borel(\R^n) = \sigma(\calJ)$ and that the generator $\calJ$ admits the exhausting sequence $[-k, k)_{k \in \N} \subset \calJ$ with $\nu([-k, k)) = \lambda^n([-k, k)) = (2k)^n < \infty$. Hence, using the previous theorem, the measures $\nu$ and $\lambda^n$ must coincide in every $A \subset \borel(\R^n)$.
		
		\item TODO
	\end{enumerate}
\end{proof}
	% !TeX root = ../mi-notes.tex

\chapter{Existence of measures}

\section{Preliminaries}

In this chapter we shall explore a new structure, the semi ring and a new function, the premeasure. They are analogous to a \siga and a measure, respectively, but weaker. Then we shall prove that under some conditions, premeasures can be extended to measures and semirings to \sigas. The most important implication of this result, called Caratheodory's theorem, is that the Lebesgue measure that we have so far only defined in the open $n$-dimensional intervals can be extended to any Borel set in $\borel(\R^n)$ and thus is a proper measure.


\begin{dfn}[Semi-ring]
	\label{dfn:semiring}
	Let $\sr \subset X$ be a collection of subsets of a set $X$. We say that $\sr$ is a semi-ring if the following are satisfied:
	\begin{enumerate}
		\item $\emptyset \in \sr$,
		\item $S, T \in \sr \implies S \cap T \in \sr$ (or $\sr$ is \istable), and
		\item if $S, T \in \sr$ then there exists a finite collection of pairwise disjoint sets $S_1, \dots, S_M \in \sr$ such that $S\setminus T = \bigcupdot_{j = 1}^M S_j$ (so $S\setminus T$ is the disjoint union of a finite collection in $\sr$).
	\end{enumerate}
\end{dfn}

We will see that $\calJ$ and $\calJ_{rat}$ are semi-rings.

\begin{dfn}[Premeasure]
	Let $X$ be a set, $\sr$ a semiring of subsets of $X$, and $\mu: \sr \to [0, \infty)$ be a function. We say that $\mu$ is a premeasure if the following are satisfied:
	\begin{enumerate}
		\item $\mu(\emptyset) = 0$,
		\item if $(S_n)_{n \in \N}$ is a pairwise disjoint collection of sets in $\sr$ then
		\begin{align*}
			\mu(\bigcupdot_{j \in \N} S_j) = \sum_{j \in \N}^n \mu(S_j),
		\end{align*}
		in other words, $\sigma$-additivity.
	\end{enumerate}
\end{dfn}

What is missing for a premeasure to become a measure is the fact that it is not defined on a \siga on $X$, but rather on a weaker structure, the semi-ring $\sr$.

\section{The Caratheodory theorem}

\begin{thm}[Caratheodory]
	\label{thm:caratheodory}
	Let $X$ be a set, $\sr$ a semi-ring on $X$ and $\mu$ a premeasure defined on $\sr$. Then $\mu$ has an extension to a measure $\mu$ defined on $\sigma(\sr)$. If $\sr$ contains an exhausting sequence $S_n \uparrow X$ with $\mu(S_n) < \infty$ then the extension is unique.
\end{thm}

And that's it. We could end the chapter here or make it much longer by proving Caratheodory's theorem. We may do so, if I manage to find the time to write it down, but otherwise look it up \cite[p. 41]{schilling2017}

What we will do is apply the theorem to the Lebesgue measure and give an outline of the proof.

\begin{remark}
	The $n$-dimensional Lebesgue measure satisfies the hypothesis for the premeasure in Caratheodory's theorem.
\end{remark}

\begin{proof}
	TODO
\end{proof}

TODO: outline of the proof
	% !TeX root = ../mi-notes.tex

\chapter{Measurable mappings}

In mathematics mappings between sets are a central topic. Moreover, when sets have a specific structure, we want mappings that preserve the that same structure between the two sets. For instance we have
\begin{itemize}
	\item groups and homomorphisms, which hold the group operation: $f(a \cdot b) = f(a) \cdot f(b)$;
	\item topological spaces and continuous functions, which hold the topology of the spaces in question: for any open set $V$ and continuous function $f$, $\inv{f}(V)$ is also an open set;
	\item and naturally we wish that between measurable spaces, there are appropriate mappings that preserve the measurable structure (\siga).
\end{itemize}

Let's dive right into it.

\begin{dfn}[Measurable map]
	Let $(X, \sa)$ and $(X', \sa')$ be measurable spaces. A map $T : X \to X'$ is said to be $\sa/\sa'$-measurable (or measurable unless this is too ambiguous) if
	\begin{align*}
		\inv{T}(A') = \{x \in X \mid T(x) \in A'\} \in \sa,\qquad \forall A' \in \sa',
	\end{align*}
	i.e. if the preimage of every measurable set in $\sa'$ is a measurable set in $\sa$.
\end{dfn}

In the next chapter we will particularise this to mappings $T:X \to \R$ and measurable spaces $(X, \sa)$ and $(\R, \borel(\R^n))$ to pave the way for integration.

\begin{remark}$ $\newline
	\begin{enumerate}
		\item In probability theory, a measurable map is usually called a random variable.
		\item Consider the collection $\inv{T}(\sa') = \{\inv{T}(A') \mid A' \in \sa'\}$. Recall the example from chapter 1, in which we proved that the preimage of a \siga is a \siga. We can rephrase the definition of measurability as
		\begin{align*}
			T \text{ is measurable } \iff \inv{T}\sa' \subset \sa
		\end{align*}
		\item If we write $T: (X, \sa) \to (X', \sa')$ then we usually mean that $T$ is $\sa/\sa'$-measurable.
		
		\item If $T: (\R^n, \borel(\R^n)) \to (\R, \borel(\R))$, then we simply say $T$ is Borel-measurable.
	\end{enumerate}
\end{remark}

\begin{lem}
	Let $(X, \sa)$ and $(X', \sa')$ be two measurable spaces with $\sa' = \sigma(\calG')$. A map $T: X \to X'$ is $\sa/\sa'$-measurable $\iff \inv{T}(G') \in \sa,\ \forall G' \in \calG' \iff \inv{T}(\calG') \subseteq \sa$.
\end{lem}

So it is enough to check measurability on the generators only.
	% !TeX root = ../mi-notes.tex

\chapter{Measurable functions}

In this chapter we restrict our attention to measurable maps whose domain is any measurable space $(X, \sa)$ but whose domain is $(\R, \borel(\R))$. To distinguish them from the general case, we shall use the same terminology as \cite{schilling2017}, and call them measurable functions.

The following result is trivial but important, as the intervals we introduce in the following lemma are easy to work with.

\begin{lem}
	\label{lem:measurable-generators-functions}
	Let $(X, \sa)$ be a measurable space, and $u: X \to \R$. Then, $u$ is $\sa/\borel(\R)$-measurable
	\begin{align*}
		\iff &\inv{u}((a, \infty)) = \{x \in X \mid u(x) > a\} = \{u > a\} \in \sa\\
		\iff &\inv{u}([a, \infty)) = \{x \in X \mid u(x) > a\} = \{u \geq a\} \in \sa\\
		\iff &\inv{u}((-\infty, a)) = \{x \in X \mid u(x) > a\} = \{u < a\} \in \sa\\
		\iff &\inv{u}((-\infty, a]) = \{x \in X \mid u(x) > a\} = \{u \leq a\} \in \sa\\
	\end{align*}
\end{lem}

Notice that we have introduced some notation here, namely, for a function $u : X \to \R$ we denote by $\{u > a\}$ the set $\{x \in X \mid u(x) > a\}$. We define analogous notations for $\geq, <, \leq, =, \neq, \in, \not\in$ and more.

\begin{proof}
	The proof follows immediately from \autoref{lem:measurable-generators}. Recall that all the families of intervals of the lemma are generators of the Borel \siga on $\R$.
\end{proof}

\section{The extended real line $\Rb$}

Throughout this chapter, we will deal with the concepts of $\lim_n, \limsup_n, \liminf_n, \sup_n$ and $\inf_n$ which will often be infinite. If we agree that $-\infty < x < \infty,\ \forall x \in \R$ it makes sense to include the values $\pm \infty$ in $\R$ to build the extended space $\Rb = [-\infty, +\infty] = \R \cup \{\pm \infty\}$. We would like $\Rb$ to inherit as much as possible from the algebraic, topological and measurable structures of $\R$.


\subsection{Extension of the algebraic structure}

We extend the algebraic structure by extending the addition and multiplication tables as follows:

\begin{center}
	\begin{tabular}{c|ccc}
		$+$ & $x\in \R$ & $+\infty$ & $-\infty$ \\ 
		\hline 
		$y\in \R$ & $x+y$ & $+\infty$ & $-\infty$ \\ 
		$+\infty$ & $+\infty$ & $+\infty$ & not defined \\ 
		$-\infty$ & $-\infty$ & not defined & $-\infty$ \\ 
	\end{tabular} 
\end{center}

\begin{center}
	\begin{tabular}{c|cccc}
		$\cdot$ & $0$ & $x \in \R\setminus\{0\}$ & $+\infty$ & $-\infty$ \\ 
		\hline 
		$0$ & $0$ & $0$ & $0$* & $0$* \\ 
		$y \in \R\setminus\{0\}$ & $0$ & $x \cdot y$ & $\sgn(y)\cdot \infty$ & $-\sgn(y)\cdot \infty$ \\ 
		$+\infty$ & $0$* & $\sgn(x)\cdot \infty$ & $+\infty$ & $-\infty$ \\ 
		$-\infty$ & $0$* & $-\sgn(x)\cdot \infty$ & $-\infty$ & $+\infty$ \\ 
	\end{tabular} 
\end{center}

Caution: here we understand $\pm$ as limits but $0$ only as bona-fide $0$ (i.e. not as a limit, which would cause convergence problems). Conventions are tricky. Expresions of the form
\begin{align*}
	\infty - \infty \text{ or } \frac{\pm \infty}{\pm \infty}
\end{align*}
should be avoided.

\subsection{Extension of the topological structure}

\begin{dfn}[Neighbourhoods in $\Rb$]
	For some $x \in \Rb$ we say that a neighbourhood of $x$ is a set of the form
	\begin{align*}
		(x - \varepsilon, x + \varepsilon) \tif x \in \R\\
		(a, +\infty] \tif x = +\infty \\
		[-\infty, a) \tif x = -\infty
	\end{align*}
	for some $a, \varepsilon \in \R$.
\end{dfn}

\begin{dfn}[Open set in $\Rb$]
	We say that a set $U \subseteq \Rb$ is open if, for every point $x \in U$ there exists a neighbourhood $B(x)$ of $x$ such that $x \in B(x) \subseteq \Rb$.
\end{dfn}

\subsection{Extension of the measurable structure}

\begin{dfn}[Borel \siga on $\Rb$]
	The Borel \siga on $\Rb$ is defined by
	\begin{align*}
		\borel(\Rb) := \{ B^\ast = B \cup S \mid B \in \borel(\R) \land S \in \calS\},
	\end{align*}
	where $\calS = \left\{\emptyset, \{+\infty\}, \{-\infty\}, \{-\infty, +\infty\}\right\}$.
\end{dfn}

The reason this extension is still called a Borel \siga is justified by the above definition of $\borel(\Rb)$, by the extension of the topological structure and by the following lemma.

\begin{lem}
	$\borel(\Rb)$ is generated by sets of the form $[a, \infty]$ (or $(a, \infty]$ or $[-\infty, a]$ or $[-\infty, a)$), where $a \in \R$ or $a \in \Q$ (and hence those intervals are subsets of $\Rb$ or $\overline{\Q}$, resp.)
\end{lem}

\begin{proof}
	It is analogous to the one given for \autoref{thm:borel-interval-generators}.
\end{proof}

\subsection{Final remarks}

\begin{dfn}[Numerical function]
	A function $u : X \to \Rb$ that takes values on $\Rb$ is called a numerical function.
\end{dfn}

\begin{dfn}[Set of measurable functions]
	Let $(X, \sa)$ be a measurable space. We write
	\begin{align*}
		\calM &:= \calM(\sa) := \{u : X \to \R \mid u \text{ is } \sa/\borel(\R)\text{-measurable }\},\\
		\calM_{\Rb} &:= \calM_{\Rb}(\sa) := \{u : X \to \Rb \mid u \text{ is } \sa/\borel(\Rb)\text{-measurable }\}
	\end{align*}
	for the families of real-values and numerical-valued measurable functions on $X$.
\end{dfn}

\section{Simple functions}

Now we will see some important (yet simple) examples. Throughout this section, $(X, \sa)$ will be a measurable space.

\begin{eg}[Indicator functions]
	Let $A \in \sa$ and define $\ind_A : X \to \R$ by
	\begin{align*}
		\ind_A(x) = \begin{cases}
		0 \tif x \in A\\
		1 \tif x \not\in A
		\end{cases}.
	\end{align*}
	Then $\ind_A$ is $\sa/\borel(\R)$-measurable since
	\begin{align*}
		\inv{\ind_A}((-\infty, a)) = \{\ind_A < a\} = \begin{cases}
		\emptyset \tif a \leq 0\\
		A^\complement \tif 0 < a \leq 1\\
		\R \tif a > 1
		\end{cases} \in \sa
	\end{align*}
	
	In fact, from the proof we can see that $\ind_A$ is $\sa/\borel(\R)$-measurable if and only if $A \in \sa$. So measurability of $\ind_A$ as a \textbf{function} is equivalent to the measurability of $A$ as a \textbf{set} (recall that a set is measurable if it is part of the $\sa$ which is the domain of the measure at hand).
\end{eg}

The following example is so important that we will give it as a definition.

\begin{dfn}[Simple function]
	Let $A_1, \dots, A_M \in \sa$ be pairwise disjoint subsets of $X$ and $y_1, \dots, y_M \in\R$. A function $f: X \to \R$ of the form
	\begin{align*}
		f(x) = \sum_{i = 1}^M y_i \ind_{A_i}
	\end{align*}
	is called a simple function. 
	
	We denote by $\calE(\sa) = \calE \subseteq \calM$ the family of all simple functions $f:(X,\sa) \to (X, \borel(\R))$.
\end{dfn}

Note that
\begin{align*}
	f(x) = \begin{cases}
	y_i \tif x \in A_i \\
	0 \tif x \not\in \bigcupdot_{i = 1}^M A_i
	\end{cases}
\end{align*}
is an alternative definition. The $i$ that makes $x \in A_i$ hold is unique or non-existent since, $(A_i)$ are pairwise disjoint, but do not necessarily cover the whole $X$.

This small inconvenience (of the possibility that $i$ does not exist) is easily fixed by extending the collection of sets to be a partition by defining
\begin{align*}
	A_0 = X \setminus \bigcupdot_{i = 1}^M A_i \text{ and } y_0 = 0
\end{align*}
and adding the set $A_0$ to the sum in the definition of $f$.

This leads to the following definition
\begin{dfn}[Standard representation of a simple function]
	\label{dfn:standard-representation-simple}
	Let $f \in \calE$ be a simple function given by
	\begin{align}
		\label{eq:standard-repr}
		f = \sum_{n = 0}^n a_n \ind_{A_n}
	\end{align}
	with $A_i \cap A_j = \emptyset$ and $X = \bigcupdot_{n = 0}^M A_n$. Then \autoref{eq:standard-repr} is called a standard representation of the simple function $f$.
\end{dfn}

\begin{remark}
	Note that this representation is not unique. However, we will see that this does not matter.
\end{remark}

Notice that a simple function takes only finitely many values. The converse is also true.

\begin{lem}
	Any measurable function $g$ which takes only finitely many values $\{y_0, y_1, \dots, y_M\}$ can be expressed as a linear combination of simple functions.
\end{lem}

\begin{proof}
	Define $A_i := \{g = y_i\} = \{x \in X \mid g(x) = y_i\} = \inv{g}(\{y_i\})$. $A_i \in \sa, \forall i = 0, \dots, M$ since $\{y_i\} = (-\infty, y_i] \setminus (-\infty, y_i) \in \borel(\R)$ and $g$ is assumed to be measurable.
	
	Since $y_0, \dots y_M$ are all distinct, then $A_i \cap A_j = \emptyset$ if $i \neq j$. Furthermore, $g$ takes the value $y_i$ on $A_i$, and thus
	\begin{align*}
		g = \sum_{i = 0}^M y_i \ind_{A_i},
	\end{align*}
	which is in fact the standard representation of $g$.
\end{proof}

In general, the representation of a simple function as a linear combination of simple functions is not unique. Consider the function
\begin{align*}
	f(x) = \ind_{[0, 1]}(x) + \ind_{[0, \frac{2}{3}]}(x) + \ind_{[0, \frac{1}{3}]}(x) = \begin{cases}
	0 \tif x \notin [0, 1] \\
	3 \tif x \in [0, \frac{1}{3}] \\
	2 \tif x \in (\frac{1}{3}, \frac{2}{3}] \\
	1 \tif x \in (\frac{2}{3}, 1]
	\end{cases}
\end{align*}

which can also be given by its standard representation
\begin{align*}
	f(x) = 0 \cdot \ind_{\R\setminus [0, 1]} + 1 \cdot \ind_{(\frac{2}{3}, 1]} + 2 \cdot \ind_{(\frac{1}{3}, \frac{2}{3}]} + 3\cdot\ind_{[0, \frac{1}{3}]}
\end{align*}

Simple functions are the building blocks of all measurable functions (in the sense that any measurable function is the limit of a sequence of simple functions).

\subsection{Properties of simple functions}

\begin{thm}[Properties of simple functions]
	\label{thm:prop-simple-functions}
	Let $f, g \in \calE(\sa)$. Then the following hold:
	\begin{enumerate}
		\item $f\pm g \in \calE(\sa)$ and $f\cdot g \in \calE(\sa)$
		\item $f^+ = \max(f, 0)$ and $f^- = \max (-f, 0)$ are in $\calE(\sa)$
		\item $\abs{f} \in \calE(\sa)$
	\end{enumerate}
\end{thm}

\begin{proof}Let
	\begin{align*}
		f = \sum_{i = 0}^M a_i \ind_{A_i},\qquad g = \sum_{j = 0}^N b_j \ind_{B_j}
	\end{align*}
	be the standard representations of $f$ and $g$, resp. Then,
	\begin{enumerate}
		\item
		\begin{align*}
			f \pm g = \sum_{i=0}^{M}\sum_{j=0}^{N} (a_i \pm b_j)\ind_{A_i \cap B_j},
			\qquad \sum_{i=0}^{M}\sum_{j=0}^{N} (a_i \cdot b_j)\ind_{A_i \cap B_j}
		\end{align*}
		are the standard representations of $f\pm g$ and $f \cdot g$, respectively and thus $f\pm g , f\cdot g \in \calE(\sa)$.
		
		\item
		\begin{align*}
			f^+ = \sum_{i \mid a_i \geq 0} a_i \ind_{A_i} \qquad f^- = \sum_{i \mid a_i \leq 0} - a_i \ind_{A_i}
		\end{align*}
		are the standard representations of $f^+$ and $f^-$, resp. and hence $f^+, f^- \in \calE(\sa)$.
		
		\item $\abs{f} = f^+ + f^- \in \calE(\sa)$ by the first two properties.
	\end{enumerate}
\end{proof}

\section{Sequences of simple functions. The sombrero lemma.}

\begin{thm}[Sombrero lemma]
	\label{thm:sombrero}
	
	Let $(X, \sa)$ be a measurable space and $u : X \to [0, \infty]$ a non-negative $\sa/\borel(\Rb)$-measurable function. Then, there exists an increasing sequence $(f_n)_{n \in \N} \subset \calE(\sa)$ of non-negative simple functions such that for any $x \in X$
	\begin{align*}
		u(x) = \lim_{n\to \infty} f_n(x) = \sup_{n\in\N} f_n(x).
	\end{align*}
\end{thm}

Note that in the previous statement, $f_n$ is a sequence of real-valued (as opposed to numerical) functions. Also, the limit is to be understood point-wise (as opposed to uniformly).

\begin{proof}
	TODO
\end{proof}

\begin{cor}
	Let $(X, \sa)$ be a measurable space. Then, for any numerical and $\sa/\borel(\Rb)$-measurable function $u: X \to \Rb$ there exists a sequence of simple functions $(f_n)_{n\in \N} \subset \calE(\sa)$ such that $\abs{f_n} \leq \abs{u}$ and
	\begin{align*}
		u(x) = \lim_{n\to \infty} f_n(x)
	\end{align*}
	
	Moreover, if $u$ is bounded then convergence is uniform (as opposed to point-wise).
\end{cor}

\begin{proof}
	Write $u = u^+ - u^-$, where $u^+, u^-$ are both non-negative functions. We first show that $u^+, u^-$ are $\sa/\borel(\Rb)$-measurable. We shall proceed by using \autoref{lem:measurable-generators-functions}, with generators of the form $(a, \infty])$. For any $a \in \Rb$ we have
	\begin{align*}
		\{u^+ > a\} = \begin{cases}
		X \tif a < 0 \\
		\{u \geq a\} \tif a \geq 0
		\end{cases}\in \sa
	\end{align*}
	Similarly,
	\begin{align*}
		\{u^- > a\} = \begin{cases}
		X \tif a < 0 \\
		\{-u \geq a\} = \{u < a\} \tif a \geq 0
		\end{cases}
		\in \sa.
	\end{align*}
	
	Since $u^+, u^-$ are non-negative measurable functions, \autoref{thm:sombrero} applies and there exist two sequences $(f_n)_{n\in\N}, (g_n)_{n\in\N} \subset \calE(\sa)$ such that $f_n \uparrow u^+$ and $g_n \uparrow u^-$. Hence
	\begin{align*}
		\lim_{n\to \infty} f_n - \lim_{n\to \infty} g_n = \lim_{n\to \infty} (f_n - g_n) = \lim_{n\to \infty} h_n = u^+ - u^- = u.
	\end{align*}
	Furthermore, $\abs{f_n - g_n} \leq \abs{f_n} - \abs{g_n} = f_n - g_n \leq u^+ + u^- = \abs{u}$. Finally, if $u$ is bounded then $\exists N \in \N$ such taht $u(x) \leq N, \forall x \in X$. Then from the proof of \autoref{thm:sombrero} (applied to $u^+$ and $u^-$) we see that $\forall n \geq N$ and $\forall x \in X$
	\begin{align*}
		\abs{f_n(x) - g_n(x) - u(x)} \leq \abs{f_n - u^+(x)} + \abs{g_n - u^-(x)} \leq \frac{1}{2^{n - 1}}.
	\end{align*}
	This implies uniform convergence.
\end{proof}

\paragraph{Convention} Given a sequence $(u_n)_{n\in\N} \subset \calM_{\Rb}(\sa)$, by $\sup_{n\in\N} u_n,\ \inf_{n\in\N} u_n, \limsup_{n\in\N} u_n$ and $\liminf_{n\in\N}$ we mean the point-wise defined functions $(\sup_{n\in\N} u_n)(x) = \sup_{n\in\N} u_n(x) = \sup\{u_n(x) \mid n \in \N\}$ and similarly for others.

Recall that
\begin{align}
	\liminf_{n\to \infty} u(x) = \sup_{n \geq 1} \inf_{m \geq n} u_m(x)
\end{align}
and
\begin{align}
	\limsup_{n\to\infty} u(x) = \inf_{n\geq 1}\sup_{m \geq n} u_m(x).
\end{align}

Also,
\begin{align*}
	v_n(x) &= \inf_{m \geq n} u_m(x) \uparrow \liminf_{n\to \infty} u(x),\\
	w_m(x) &= \sup_{m \geq n} u_m(x) \downarrow \limsup_{n\to\infty} u_n(x)
\end{align*}
and
\begin{align*}
	\inf_{n\in\N} u_n(x) \leq \liminf_{n\to\infty} u_n \leq \limsup_{n\to \infty} u_n(x) \leq \sup_{n\in\N} u_n(x).
\end{align*}


If $\liminf_{n\to \infty} u_n(x) = \limsup_{n\to \infty} u_n(x)$, then $\lim_{n\to \infty} u_n(x)$ exists and equals the common value.

\begin{cor}
	\label{cor:nasty-extended-measurable}
	Let $(X, \sa)$ be a measurable space and $(u_n)_{n\in\N} \subset \calM_{\Rb}(\sa)$ a sequence of $\sa/\borel(\Rb)$-measurable functions. Then, $\inf_{n\geq 1} u_n, \liminf_{n\to \infty} u_n, \limsup_{n\to \infty} u_n, \sup_{n \geq 1}$ are all $\sa/\borel(\Rb)$-measurable functions. 
\end{cor}

\begin{proof}
	First we prove that $\sup$ and $\inf$ are both $\sa/\borel(\Rb)$-measurable.
	\begin{align*}
		\{\sup_{n \geq 1} u_n \leq a\} = \bigcap_{n=1}^\infty \{u_n \leq a\} \in \sa\\
		\{\inf_{n \geq 1} u_n \geq a\} = \bigcap_{n=1}^\infty \{u_n \geq a\} \in \sa
	\end{align*}
	
	Now we use these to prove the rest.
	
	$\liminf_{n\to \infty} u_n = \sup_{n \geq 1} \inf_{m \geq n} u_m$. Set $v_n = \inf_{m \geq n} u_m$ as before and because of the above $v_n$ is $\sa/\borel(\Rb)$-measurable. Since $v_n \uparrow \liminf_{n\to \infty} u_n$ we have $\liminf_{n\to \infty} = \sup_{n \geq 1} v_n \in \sa$. Hence $\liminf_{n\to \infty} u_n$ is also $\sa/\borel(\Rb)$-measurable.
	
	Similarly, write $w_n = \sup_{m \geq n} u_n \in \sa$ because of the above. Then, since $w_n \downarrow \limsup_{n\to\infty} u_n$ we have $\limsup_{n\to \infty} u_n = \inf_{n \geq 1} w_n \in \sa$ and hence $\limsup_{n\to \infty} u_n$ is $\sa/\borel(\Rb)$-measurable.
\end{proof}

In fact, we can deduce more.

\begin{cor}
	\label{cor:properties-measurable}
	Let $u, v \in \calM_\Rb(\sa)$ be two $\sa/\borel(\Rb)$-measurable functions. Then $u \pm v,\ u \lor v = \max\{u, v\}$ and $u \land v = \min\{u, v\}$ are $\sa/\borel(\Rb)$-measurable.
\end{cor}

\begin{proof}
	By \autoref{thm:sombrero} there exist two sequences of non-negative simple functions $(f_n)$ and $(g_n)$ such that $f_n \uparrow u$ and $g_n \uparrow v$. Since $f_n$ and $g_n$ are simple functions, by \autoref{thm:prop-simple-functions} we have that $f_n \pm g_n$ is also a simple function with $\lim_{n\to \infty} f_n + g_n = u + v$. Moreover, $f_n + g_n$ is $\sa/\borel(\R)$-measurable (since $f_n, g_n$ are) and thus it is also $\sa/\borel(\Rb)$-measurable and hence $u+v$ is also $\sa/\borel(\Rb)$-measurable.
	
	Similarly, we can prove the corollary for $u\lor v$ and $u \land v$.
\end{proof}

\begin{remark}
	Applying the above to $u^+, u^-$ we see that $u$ is $\sa/\borel(\R)$-measurable if and only if $u^+$ and $u^-$ are $\sa/\borel(\R)$-measurable.
\end{remark}

\begin{cor}
	If $u, v \in \calM_\Rb(\sa)$, then the following sets are all measurable:
	\begin{align*}
		\{u \leq v\}, \{u < v\}, \{u = v\}, \{u > v\}, \{u \geq v\}
	\end{align*}
\end{cor}

\begin{proof}
	Not that $u, v\in \calM_\Rb \implies u - v \in \calM_\Rb$ therefore we may rewrite the previous sets as
	\begin{align*}
		\{u - v \leq 0\}, \{u - v < 0\}, \{u - v = 0\}, \{u - v > 0\}, \{u - v \geq 0\}.
	\end{align*}
	With the exception of $\{u - v = 0\}$, all the others are generators of the Borel \siga so, by \autoref{lem:measurable-generators}, they are measurable. As for $\{u - v = 0\}$ we have that
	\begin{align*}
		\{u - v = 0\}
		= \inv{(u - v)}(\{0\})
		= \inv{(u - v)}\left(\bigcap_{n = 1}^\infty [0, \frac{1}{n})\right) \in \sa
	\end{align*}
	since $\bigcap_{n = 1}^\infty [0, \frac{1}{n}) \in \borel(\R)$.
\end{proof}

\section{Examples}

\begin{eg}
	(This is actually exercise 9.8 in \cite[p. 79]{schilling2017}).
	Every function $u: \N \to \R$ on $(\N, \powerset(\N))$ is $\powerset(\N)/\borel(\R)$-measurable.
\end{eg}

\begin{proof}
	Since our \siga is $\powerset(\N)$, all subsets of $\N$ are measurable. Therefore
	\begin{align*}
		\{f \leq \alpha\} = \{k \in \N \mid f(k) \leq \alpha\} \subset \N \implies \{f \leq \alpha\} \in \powerset(\N)
	\end{align*}
	and we may apply \autoref{lem:measurable-generators-functions}.
\end{proof}

	% !TeX root = ../mi-notes.tex

\chapter{Integrals of non-negative functions}

We are now ready to define the integral of a non-negative function in terms of a measure.

\section{Integral of a non-negative simple function}

Since each $u \in \calM^+(\sa)$ or $\calM^+_\Rb(\sa)$ is a limit of an increasing sequence of simple functions (\autoref{thm:sombrero}), we concentrate first on the collection $\calE^+ = \calE^+(\sa)$ of all $\sa/\borel(\R)$-measurable non-negative simple functions.

Let $f(x) = \sum_{i = 0}^N a_i \ind_{A_i}$ be the standard representation of $f \in \calE^+$. Recall that this means that $(A_i)$ is a collection of sets in $\sa$ which define a partition of $X$ (i.e. $X = \bigcupdot_{i = 0}^N A_i)$. Furthermore, on $A_i$, $f$ takes the value $a_i$.

Let
\begin{align}
	I_\mu(f) := \sum_{i = 0}^N a_i \mu(A_i).
\end{align}
We want to interpret $I_mu(f)$ as $\int f d\mu$, but there might be a small problem, namely $f$ might have more than one standard representation so we need to check that the definition of $I_\mu(f)$ is independent of the specific representation, i.e. they all give the same answer.

\begin{lem}[Integrals of simple functions are representation-invariant]
	If $f = \sum_{i = 0}^M a_i \ind_{A_i} = \sum_{j = 0}^N b_j \ind{B_j}$ are two standard representations of a simple function then $\sum_{i = 0}^M a_i\mu(A_i) = \sum_{j = 0}^n b_j \mu(B_j)$.
\end{lem}

\begin{proof}
	Recall that since $(A_j)$ and $(B_j)$ are partitions we may write
	\begin{align*}
		A_i &= A_i \cap \bigcupdot_{j = 0}^M B_j = \bigcupdot_{j = 0}^M A_i \cap B_j, \\
		B_j &= B_j \cap \bigcupdot_{i = 0}^N A_i = \bigcupdot_{i = 0}^N A_i \cap B_j.
	\end{align*}
	Thus,
	\begin{multline*}
		\sum_{i = 0}^M a_i \mu(A_i)
		= \sum_{i = 0}^M a_i \mu\left(\bigcupdot_{j = 0}^M A_i \cap B_j\right)
		= \sum_{i = 0}^M a_i \sum_{j = 0}^N \mu(A_i \cap B_j)\\
		= \sum_{i = 0}^M \sum_{j = 0}^N a_i \mu(A_i \cap B_j)
		= \sum_{j = 0}^N \sum_{i = 0}^M a_i \mu(A_i \cap B_j).
	\end{multline*}
	Now,
	\begin{itemize}
		\item if $A_i \cap B_j = \emptyset$ then $a_i\mu(A_i \cap B_j) = 0 = b_j \mu(A_i \cap B_j)$, and
		\item if $A_i \cap B_j \neq \emptyset$ then $\exists x \in A_i \cap B_j$ so that $f(x) = a_i = b_j$. Thus $a_i \mu(A_i \cap B_j) = b_j \mu(A_i \cap B_j)$.
	\end{itemize}
	Therefore,
	\begin{multline*}
		\sum_{i = 0}^m a_i \mu(A_i)
		= \sum_{j = 0}^N \sum_{i = 0}^M a_i \mu(A_i \cap B_j)
		= \sum_{j = 0}^N \sum_{i = 0}^N b_j \mu(A_i \cap B_j)\\
		= \sum_{j = 0}^N b_j \sum_{i = 0}^M \mu(A_i \cap B_j)
		= \sum_{j = 0}^N b_j \mu(B_j)
	\end{multline*}
\end{proof}

So $I_\mu(f)$ is well-defined regardless of the chosen standard representation. We are about to be ready to define the integral for arbitrary non-negative functions (once we verify that the properties of simple functions that have allowed us to state \autoref{thm:sombrero} hold under integrals).

\begin{thm}[Properties of $I_\mu(f)$]
	\label{thm:prop-integrals-simple}
	Let $f, g \in \calE^+(\sa)$ and $\lambda \geq 0$, then
	\begin{enumerate}
		\item $I_\mu(\ind_A) = \mu(A)$, for any $A \in \sa$
		\item $I_\mu(\lambda f) = \lambda I_\mu(f)$ (positive homogenous degree 1)
		\item $I_\mu(f + g) = I_\mu(f) + I_\mu(f)$ (additive)
		\item if $f \leq g$ then $I_\mu(f) \leq I_\mu(g)$ (monotone)
	\end{enumerate}	
\end{thm}

\begin{proof}
	\begin{enumerate}
		\item Clearly in $f = \ind_A$ there is only one term in the sum and $a_1 = 1$ hence $I_\mu(\ind_A) = 1\mu(A) = \mu(A)$.
		\item
		\begin{align*}
			I_\mu(\lambda f)
			&= I_\mu\left( \lambda \sum_{n=0}^M a_n \ind_{A_n} \right)\\
			&= I_\mu\left( \sum_{n = 0}^M (\lambda a_n) \ind_{A_n} \right)\\
			&= \sum_{n = 0}^M (\lambda a_n)\mu(A_n)\\
			&= \lambda \sum_{n = 0}^M a_N \mu(A_n)
			= \lambda I_\mu(f)
		\end{align*}
		
		\item Recall from \autoref{thm:prop-simple-functions} that
		\begin{align*}
			f+g = \sum_{i=0}^{M}\sum_{j=0}^{N} (a_i + b_j)\ind_{A_i \cap B_j}.
		\end{align*}
		Therefore,
		\begin{align*}
			I_\mu(f+g)
			&= \sum_{i=0}^{M}\sum_{j=0}^{N} (a_i + b_j)\mu(A_i \cap B_j) \\
			&= \sum_{i=0}^M a_i \sum_{j=0}^N \mu(A_i \cap B_j) + \sum_{j=0}^N b_j \sum_{i=0}^M \mu(A_i \cap B_j)\\
			&= \sum_{i=0}^M a_i \mu(A_i) + \sum_{j=0}^N b_j \mu(B_j)\\
			&= I_\mu(f) + I_\mu(g). 
		\end{align*}
		
		\item We write $g = f + (g - f)$, where $g-f \in \calE$ by \autoref{thm:prop-simple-functions} and $g -f \geq 0$ since $f \leq g$. Therefore, $f - g \in \calE^+$ and
		\begin{align*}
			\int_\mu(f) \leq \int_\mu(f) + \underbrace{\int_\mu(f-g)}_{\geq 0} = \int_\mu(g).
		\end{align*}
	\end{enumerate}
\end{proof}

\section{Integral of a non-negative function}

\begin{dfn}[$(\mu)$-integral of a non-negative function]
	\label{dfn:mu-integral}
	Let $u \in \calM^+_\Rb(\sa)$, then the $(\mu)$-integral of $u$ is defined by
	\begin{align}
		\int u d\mu := \sup \{ I_\mu(g) \mid g \leq u, g \in \calE^+(\sa)\} \in [0, \infty]
	\end{align}
\end{dfn}

Some alternative notations are
\begin{align*}
	\int u d\mu = \int_X u d\mu = \int_X u(x) \mu(dx) = \int_X u(x) d \mu(x).
\end{align*}

Note that here we are taking the supremum over all simple functions less than or equal to $\mu$. We will eventually show that it is enough to find a sequence $(f_n)_{n\in\N} \subset \calE^+(\sa)$ such that $f_n \uparrow u$, since we will prove that
\begin{align*}
	\int u d\mu = \lim_{n\to \infty} \int f_n d\mu = \sup_{n \geq 1} \int f_n d\mu.
\end{align*}

Before we move one, let's make sure that we did not break anything. Namely, does the definition of the integral above extend the definition of $I_\mu$ for non-negative simple functions, i.e. $I_\mu(f) = \int f d\mu$ if $f$ is a non-negative simple function. The answer is yes!

\begin{lem}
	\label{lem:def-integral-simple}
	If $f \in \calE^+(\sa)$ then $I_\mu(f) = \int f d\mu$.
\end{lem}

\begin{proof}
	Since $f \in \calE^+(\sa)$ and $f \leq f$, then by definition of the supremum, we have $I\mu(f) \leq \int f d\mu$. Now, by monotonicity of $I_\mu$, for any $g \in \calE^+(\sa)$ such that $g \leq f$ one has $I_\mu(g) \leq I_\mu(f)$. Thus,
	\begin{align*}
		\int f d\mu = \sup \{ I_\mu(g) \mid g \leq f,\ g \in \calE^+(\sa)\} \leq I_\mu(f).
	\end{align*}
	This shows that $I_\mu(f) = \int f d\mu$.
\end{proof}

\begin{remark}
	\label{rem:mu-integrals-monotone}
	It is easy to see that $\mu$-integrals are monotone\footnote{The reason we anticipate this property and not wait for \autoref{thm:prop-integrals} is that we need it for the proof of \autoref{thm:beppo-levi}.}. Since if $u \leq v$, then $\{g \in \calE^+(\sa) \mid g \leq u\} \subset \{h \in \calE^+(\sa) \mid h \leq v\}$. Hence,
	\begin{align*}
		\int u d\mu = \sup \{I_\mu(g) \mid g \in \calE^+(\sa), g \leq u\} \leq \sup \{I_\mu(h) \mid h \in \calE^+(\sa), h \leq v\} = \int v d\mu
	\end{align*}
\end{remark}

Now we come to the central theorem of this chapter.

\begin{thm}[Beppo-Lévi]
	\label{thm:beppo-levi}
	Let $(X, \sa, \mu)$ be a measure space and let $(u_n)_{n\in\N}$ be an increasing sequence in $\calM^+_\Rb(\sa)$. Then,
	\begin{align*}
		u := \sup_{n \geq 1} u_n = \lim_{n \to \infty} u_n \in \calM_\Rb^+(\sa),
	\end{align*}
	and
	\begin{align}
		\int u d\mu = \int \sup_{n \geq 1} u_n d\mu = \sup_{n \geq 1} \int u_n d\mu,
	\end{align}
	or, alternatively,
	\begin{align}
		\int u d\mu = \int \lim_{n \to \infty} u_n d\mu = \lim_{n \to \infty} \int u_n d\mu
	\end{align}
\end{thm}

Notice that the assumption that $u_n$ is an increasing sequence is necessary to be able to interchange limits with suprema.

\begin{proof}
	TODO. Important.
\end{proof}

The following corollary is really just a more concise restatement of \autoref{thm:beppo-levi}.
\begin{cor}
	\label{cor:beppo-levi}
	Let $u \in \calM_\Rb^+(\sa)$ be a non-negative, numerical, $\sa/\borel(\Rb)$-measurable function and $(f_n)_{n\in\N} \subset \calE^+(\sa)$ be a sequence of non-negative simple functions such that $f_n \uparrow u$. Then,
	\begin{align}
		\int u d\mu = \lim_{n \to \infty} \int f_n d\mu
	\end{align}
\end{cor}

\begin{proof}
	Clearly the hypothesis of \autoref{thm:beppo-levi} are satisfied. (Just take $u_n = f_n$).
\end{proof}

Now we extend \autoref{thm:prop-integrals-simple} to the case for integrals of arbitrary non-negative functions.

\begin{thm}[Properties of integrals of non-negative functions]
	\label{thm:prop-integrals}
	Let $(X, \sa, \mu)$ be a measure space. Let $u, v \in \calM^+_\Rb(\sa)$. Then,
	\begin{enumerate}
		\item $\int \ind_{A} d\mu = I_\mu(A) = \mu(A)$
		\item $\int \alpha u d\mu = \alpha \int u d\mu,\ \forall \alpha \geq 0$.\footnote{Here we require $\alpha \geq 0$ to keep the functions non-negative. As soon as we generalise to arbitrary functions this assumption is not needed.}
		\item $\int (u + v)d\mu = \int u d\mu + \int v d\mu$
		\item if $u \leq v$, then $\int u d\mu \leq \int v d\mu$
	\end{enumerate}
\end{thm}

This proof is left as exercise 9.3 in \cite[p. 79]{schilling2017}.

\begin{proof}$ $\newline
	\begin{enumerate}
		\item $\ind_A$ is clearly a simple function so, by definition $\int \ind_A d\mu = I_\mu(\ind_A) = \mu(A)$.
		\item By \autoref{thm:sombrero} there exists a sequence of simple functions $(f_n)_{n\in\N} \subset \calE^+$ such that $f_n \uparrow u$. Then,
		\begin{align*}
			\int \alpha u d\mu
			\overset{\ref{cor:beppo-levi}}{=} \lim_{n \to \infty} \int \alpha f_n d\mu
		    \overset{\ref{lem:def-integral-simple}}{=} \lim_{n \to \infty} I_\mu(\alpha f_n)
		    \overset{\ref{thm:prop-integrals-simple}}{=} \alpha \lim_{n \to \infty} I_\mu(f_n)
		    \overset{\ref{lem:def-integral-simple}}{=} \alpha \int u d \mu
		\end{align*}
		\item Once again by \autoref{thm:sombrero} there exist sequences $(f_n)_{n\in\N}, (g_n)_{n\in \N} \subset \calE^+$ such that $f_n \uparrow u$ and $g_n \uparrow v$. Then, $(f_n + g_n) \uparrow (u + v)$ and hence,
		\begin{align*}
			\int (u + v)d\mu
			\overset{\ref{cor:beppo-levi}}{=} \lim_{n \to \infty}  \int (f_n  + g_n)d\mu
			\overset{\ref{lem:def-integral-simple}}{=} \lim_{n \to \infty} I_\mu(f_n + g_n)\\
			\overset{\ref{thm:prop-integrals-simple}}{=} \lim_{n \to \infty} I_\mu(f_n) + \lim_{n \to \infty} I_\mu(g_n)
			\overset{\ref{lem:def-integral-simple}}{=} \int u d\mu + \int v d\mu
		\end{align*}
		\item Already proven in \autoref{rem:mu-integrals-monotone}.
	\end{enumerate}
\end{proof}

\begin{cor}
	\label{cor:integral-countable-sum}
	Let $(u_n)_{n\in\N} \subset \calM^+_\Rb(\sa)$. Then, $\sum_{n = 1}^{\infty} u_n \in \calM_\Rb^+(\sa)$ and
	\begin{align}
		\int \sum_{n = 1}^\infty u_n d\mu = \sum_{n = 1}^\infty \int u_n d\mu
	\end{align}
\end{cor}

This proof is left exercise 9.6 in \cite[p. 79]{schilling2017}.

\begin{proof}
	Let $v_m = \sum_{n=1}^{m} u_n$. Then $v_m \uparrow \sum_{n=1}^{\infty} u_n$ and thus
	\begin{align*}
		\int \sum_{n = 1}^\infty u_n d\mu
		= \int \lim_{m \to \infty} v_m d\mu
		\overset{\ref{cor:beppo-levi}}{=} \lim_{m \to \infty} \int v_m d\mu
		= \lim_{m \to \infty} \int \sum_{n = 1}^m u_n d\mu \\
		\overset{\ref{thm:prop-integrals}}{=} \lim_{m \to \infty} \sum_{n=1}^{m} \int u_n d\mu
		= \sum_{n = 1}^\infty \int u_n d\mu
	\end{align*}
\end{proof}

\begin{thm}[Fatou's lemma]
	\label{thm:fatou-lemma}
	Let $(u_n)_{n\in\N}$ be \textbf{any} sequence in $\calM_\Rb^+(\sa)$. Then,
	\begin{align}
		\label{eq:fatou1}
		u := \liminf_{n\to \infty} u_n \in \calM_\Rb^+(\sa)
	\end{align}
	and
	\begin{align}
		\label{eq:fatou2}
		\int \liminf_{n\to \infty} u_n d\mu \leq \liminf_{n\to \infty} \int u_n d\mu
	\end{align}
\end{thm}

The great thing about \autoref{thm:fatou-lemma} is that it makes no assumptions of monotonicity and can be applied to any sequence of non-negative, numerical, measurable functions.

\begin{proof}
	\autoref{eq:fatou1} follows from \autoref{cor:nasty-extended-measurable}.
	
	As for \autoref{eq:fatou2}, we have
	
	\begin{align*}
		\int \liminf_{n\to \infty} u_n d\mu
		&= \int \sup_{n\in\N} \inf_{j \geq n} u_j d\mu \\
		&\overset{\ref{thm:beppo-levi}}{=} \sup_{n\in\N} \int \inf_{j \geq n} u_j d\mu \\
		&\overset{\ref{thm:prop-integrals}}{\leq} \sup_{n\in\N} \inf_{l \geq n} \int u_l d\mu\\
		&= \liminf_{n\to\infty} \int u_n d\mu
	\end{align*}
	where we used the fact that $\inf_{j \geq n} \leq u_l$ for any $l \geq n$.
\end{proof}

We also have a similar result for $\limsup$, sometimes known as the reverse Fatou's lemma.

\begin{cor}[Reverse Fatou's lemma]
	\label{cor:reverse-fatou}
	
	Let $(u_n)_{n\in\N} \subset \calM^+(\sa)$ be a sequence of non-negative numerical measurable functions. If $u_n \leq u$ for all $n \in \N$ and some $v \in \calM^+(\sa)$ such that $\int v d\mu < \infty$, then
	\begin{align}
		\limsup_{n\to\infty} \int u_n d\mu \leq \int \limsup_{n\to \infty} u_n d\mu
	\end{align}
\end{cor}

\begin{proof}
	Let $w_n = v - u_n \geq 0$, i.e. $(w_n)_{n\in\N}$ is a sequence of non-negative measurable functions. By \autoref{thm:fatou-lemma} we get,
	\begin{align*}
		\int \liminf_{n\to \infty} w_n d\mu
		&\leq \liminf \int w_n d\mu \\
		&= \liminf_{n\to \infty} \left( \int v d\mu - \int u_n d\mu \right)\\
		&= \int vd\mu - \limsup_{n\to\infty} \int u_n d\mu.
	\end{align*}
	
	(Recall that $\liminf (-u_n) = -\limsup u_n$.) Thus,
	\begin{align*}
		\int v d\mu - \limsup_{n\to\infty} \int u_n d\mu
		&\geq \int \liminf_{n\to \infty} w_n d\mu \\
		&= \int \liminf_{n\to\infty} \left(v - w_n\right)d\mu \\
		&= \int vd\mu - \int \limsup_{n\to\infty} u_n d\mu.
	\end{align*}
	Since we assumed $\int v d\mu$ to be finite, we can subtract it from both sides and the lemma follows.
\end{proof}

Note how this time we had to make sure that the sequence was bounded by a function $v$ whose integral was finite. This was implicit in \autoref{thm:fatou-lemma}, only in this case the bounding is from below and the bounding function is $v = 0$.

\section{Examples}

\begin{eg}
	Consider the measure space $(X, \sa, \delta_y)$ where $\delta_y$ is the Dirac measure (cf. \autoref{eg:dirac-measure}) for some fixed $y \in X$. We claim that
	\begin{align}
		\int u d\delta_y = \int u(x)\delta_y(x) = u(y), \forall u \in \calM_\Rb^+(\sa).
	\end{align}
\end{eg}

\begin{proof}
	First consider the case $f \in \calE^+$ with standard representation $f = \sum_{n = 0}^M a_n \ind_{A_n}$ with $A_i \cap A_j = \emptyset$ if $i \neq j$ (cf. \autoref{dfn:standard-representation-simple}). We know that $y$ lies in exactly one $A_n$, say $y \in A_{n_0}$. Then
	\begin{align*}
		\int f d\delta_y
		= I_{\delta_y}(f)
		= \sum_{n=0}^M a_n \delta_y(A_n)
		= a_{n_0} = f(y).
	\end{align*}
	Now we take any sequence of simple functions $f_n \uparrow u$. By \autoref{cor:beppo-levi} we have
	\begin{align*}
		\int u d\delta_y = \lim_{n\to\infty} f_n d\delta_y = \lim_{n\to\infty} f_n(y) = u(y)
	\end{align*}
\end{proof}

\begin{eg}
	Consider the measure space $(\N, \powerset(\N), \mu)$ with
	\begin{align*}
		\mu = \sum_{n=1}^\infty a_n \delta_n,
	\end{align*}
	so that $\mu(\{n\}) = a_n,\ \forall n \in \N$.
	All the measurable functions $u : \N\to \R$ are of the form
	\begin{align*}
		u(x) = \sum_{n=1}^{\infty}u_n \ind_{\{n\}}(x),\ x \in \N.
	\end{align*}
	Then\footnote{Note how we do not expand $\int u d\mu = I_\mu(u)$, since $u$ is not a simple function (too many terms, although as we see, it does not really matter).},
	\begin{align*}
		\int u d\mu
		&= \int \sum_{n = 1}^\infty u_n \ind_{\{n\}}d\mu\\
		&\overset{\ref{cor:integral-countable-sum}}{=} \sum_{n = 1}^\infty \int u_n \ind_{\{n\}}d\mu\\
		&= \sum_{n = 1}^\infty u_n \mu(\{n\})\\
		&= \sum_{n = 1}^\infty u_n \cdot a_n.
	\end{align*}
\end{eg}
	% !TeX root = ../mi-notes.tex


\chapter{Integrals of measurable functions}

\label{chp:integrals}

In this chapter we will extend the notion of integral to the general case $u \in \calM_\Rb(\sa)$ (as opposed to just non-negative functions, which were discussed in the previous chapter).

We have already seen that any function $u \in \calM_\Rb(\sa)$ may be written as the difference of two non-negative functions $u = u^+ - u^-$, hence we have the tendency to define

\begin{align*}
	\int u d\mu = \int u^+ d\mu - \int u^- d \mu
\end{align*}

However, since $\int u^+ d\mu, \int u^- d\mu \in [0, \infty]$ and therefore both can be $+\infty$ we must be careful with the definition of the integral for a general $u$.


\begin{dfn}[$\mu$-integrable]
	\label{dfn:mu-integrable}
	Let $u \in \calM_\Rb(\sa)$, then u is said to be $\mu$-integrable if
	\begin{align}
		\max\left\{ \int u^+ d\mu, \int u^- d\mu \right\} < \infty,
	\end{align}
	i.e. both are finite. In this case we define
	\begin{align}
		\int u d\mu := \int u^+ d\mu - \int u^- d\mu \in (-\infty, +\infty).
	\end{align}
\end{dfn}

\begin{remark}
	If
	\begin{align*}
		\min\left\{ \int u^+ d\mu, \int u^- d\mu \right\} < \infty,
	\end{align*}
	i.e. at most one of the integrals is $+\infty$, then one can still define
	\begin{align*}
	\int u d\mu := \int u^+ d\mu - \int u^- d\mu \in [-\infty, +\infty].
	\end{align*}
\end{remark}

We will denote the set of all $\mu$-integrable functions by
\begin{align}
	\calL^1_\Rb(\mu) := \{u \in \calM_\Rb(\sa) \mid u \text{ is } \mu-\text{integrable}\}.
\end{align}
Similarly, if we wish to restrict ourselves to real-valued functions we will write
\begin{align}
\calL^1(\mu) := \{u \in \calM_\R(\sa) \mid u \text{ is } \mu-\text{integrable}\}.
\end{align}

\begin{remark}
	Note that for $u \geq 0$, $\int u d\mu$ always exists, but it can have the value $+\infty$. So for $u \in \calM_\Rb^+(\sa)$ we have
	\begin{align*}
		u \in \calL_\Rb^1(\mu) \iff \int u d\mu < \infty.
	\end{align*}
	Some authors still call a positive function $\mu$-integrable if it takes the value $+\infty$. We will not use this convention. Take care to update your understanding by comparing \autoref{dfn:mu-integral} and \autoref{dfn:mu-integrable}.
\end{remark}

As before, if we need to stress the integration variable (the space where the integral is defined), we write
\begin{align*}
	\int u d\mu = \int_X u(x)d\mu(x) = \int_X u(x) \mu(dx).
\end{align*}

The following theorem gives us for ways to check that a function $u \in \calM_\Rb(\sa)$ is $\mu$-integrable.

\begin{thm}[Characterisation of $\mu$-integrability]
	\label{thm:characterisation-mu-integral}
	Let $u \in \calM_\Rb(\sa)$, then the following are equivalent:
	
	\begin{enumerate}
		\item $u \in \calL^1_\Rb(\mu)$;
		\item $u^+, u^- \in \calL_\Rb^1(\mu)$;
		\item $\abs{u} \in \calL_\Rb^1(\mu)$;
		\item $\exists w \in \calL_\Rb^1(\mu)$ with $w \geq 0$ and $\abs{u} \leq w$.
	\end{enumerate}
\end{thm}

\begin{proof}$ $
	\begin{itemize}
		\item 1. $\iff$ 2. follows from the definition of $\mu$-integrability.
		\item 2. $\iff$ 3. follows from $\abs{u} = u^+ + u^-$ and $u^+, u^- \leq \abs{u}$ and the monotonicity for non-negative measurable functions.
		\item 3. $\iff$ 4. follows from monotonicity of the integral of non-negative measurable functions.
	\end{itemize}
\end{proof}

\begin{thm}[Properties of the $\mu$-integral]
	\label{thm:properties-mu-integral}
	Let $(X, \sa, \mu)$ be a measure space, $u, v \in \calL_\Rb^1(\mu)$ and $\alpha \in \R$. Then, the following hold
	
	\begin{enumerate}
		\item $\alpha u \in \calL_\Rb^1(\mu)$ and $\int \alpha u d\mu = \alpha \int u d\mu$.
		\item $u + v \in \calL_\Rb^1(\mu)$ and $\int (u+v)d\mu = \int ud\mu + \int vd\mu$.
		\item $\min\{u, v\}, \max\{u,v\} \in \calL_\Rb^1(\mu)$
		\item if $u \leq v$, then $\int u d\mu \leq \int v d\mu$.
		\item \begin{align*}
			\abs{\int u d\mu} \leq \int \abs{u}d\mu.
		\end{align*}
	\end{enumerate}
\end{thm}

\begin{proof}
	For 1. and 2., we first use \autoref{thm:characterisation-mu-integral} to prove that $\alpha u$ and $u + v$ are in $\calL^1_\Rb(\mu)$. Then, we rewrite each in terms of positive and negative parts to prove that the integrals coincide.
	\begin{enumerate}
		\item We will prove that $\alpha u \in \calL_\Rb^1(\mu) \iff \abs{\alpha u} \calL_\Rb^1(\mu)$. Now,
		\begin{align*}
			\int \abs{\alpha u} d\mu
			= \int \abs{\alpha} \abs{u} d\mu
			\overset{\ref{thm:prop-integrals}}{=} \abs{\alpha} \int \abs{u} d\mu < \infty.
		\end{align*}
		Thus, by \autoref{thm:characterisation-mu-integral} we have that $\alpha u \in \calL^1_\Rb(\mu)$. To check $\int \alpha u d\mu = \alpha \int u d\mu$ we consider two cases:
		\begin{itemize}
			\item If $\alpha \geq 0$ then $(\alpha u)^+ = \alpha u^+$ and $(\alpha u)^- = \alpha u^-$ hence
			\begin{align*}
				\int \alpha u d\mu
				= \int \alpha u^+d\mu - \int \alpha u^- d\mu
				= \alpha \int u^+d\mu - \alpha \int u^- d\mu\\
				= \alpha \left( \int u^+ d\mu - \int u^- d\mu \right)
				= \alpha \int u d\mu.
			\end{align*}
			\item If $\alpha < 0$ then $(\alpha u)^+ = -\alpha u^-$ and $(\alpha u)^- = -\alpha u^+$. Therefore,
			\begin{align*}
				\int \alpha u d\mu
				&= \int \left( (\alpha u)^+ - (\alpha u)^- \right)d\mu\\
				&= \int \underbrace{- \alpha}_{> 0} u^- d\mu - \int \underbrace{- \alpha}_{> 0} u^+ d\mu \\
				&= (-\alpha)\int u^- d\mu - (-\alpha)\int u^+ d\mu\\
				&= (-\alpha)\left(\int u^- d\mu - \int u^+ d\mu\right)\\
				&= (-\alpha)\left(-\int u d\mu\right) = \alpha\int u d\mu
			\end{align*}
		\end{itemize}
	
		\item $\abs{u + v} \leq \abs{u} + \abs{v}$ so by linearity for non-negative integrals we have
		\begin{align*}
			\int \abs{u + v}d\mu \leq \int \abs{u} d\mu  + \int \abs{v}d\mu < \infty.
		\end{align*}
		Hence, by \autoref{thm:characterisation-mu-integral}, $u + v \in \calL^1_\Rb(\mu)$. As before, $(u + v) = (u + v)^+ - (u + v)^- = u^+ - u^- + v^+ - v^-$ hence
		\begin{align*}
			(u + v)^+ + u^- + v^- =  (u + v)^- = u^+ + v^+,
		\end{align*}
		so
		\begin{align*}
			\int \left[(u + v)^+ + u^- + v^-\right]d\mu =  \int \left[(u + v)^- = u^+ + v^+\right] d\mu
		\end{align*}
		where all functions are non negative. Applying \autoref{thm:prop-integrals} we have
		\begin{align*}
			\int (u + v)^+ d\mu + \int u^- d\mu + \int v^- d\mu = \int (u+v)^- d\mu + \int u^+ d\mu + \int v^+ d\mu.
		\end{align*}
		Rewriting again (every term is finite since $u, v, u+v \in \calL^1_\Rb(\mu)$ so subtraction is not a problem), we have
		\begin{align*}
			\int (u + v)^+d\mu - \int (u + v)^- d\mu = \int u^+ d\mu - \int u^- d\mu + \int v^+ d\mu - \int v^- d\mu,
		\end{align*}
		or,
		\begin{align*}
			\int (u + v)d\mu = \int u d\mu + \int v d\mu.
		\end{align*}
		
		\item Observe that both $\max(u, v), \min (u, v) \leq \abs{u} + \abs{v}$ so
		\begin{align*}
			\int \max(u, v) d\mu &\leq \int \abs{u}d\mu + \int \abs{v}d\mu < \infty\\
			\int \min(u, v) d\mu &\leq \int \abs{u}d\mu + \int \abs{v}d\mu < \infty
		\end{align*}
		by \autoref{thm:characterisation-mu-integral}. Thus, $\max(u, v), \min(u, v) \in \calL^1_\Rb(\mu)$.
		
		\item Suppose $u \leq v$ then $u^+ \leq v^+$ and $u^- \geq v^-$, so
		\begin{align*}
			\int u d\mu
			= \int u^+ d\mu - \int v^- d\mu \leq \int v^+ d\mu - \int v^- d\mu = \int v d\mu.
		\end{align*}
		
		\item Recall that one can defined $\abs{a} = \max (a , -a)$. Then
		\begin{align*}
			\abs{\int u d\mu}
			&= \max \left\{ \int u d\mu, -\int u d\mu \right\} \\
			&= \max \left\{ \int u d\mu, \int -u d\mu \right\} \\
			&= \max \left\{ \int \abs{u} d \mu, \int \abs{u} d\mu \right\}
			= \int \abs{u} d\mu.
		\end{align*}
	\end{enumerate}
\end{proof}

\begin{remark}
	Note that in property 2. we did had the assumption that $(u + v)$ does not take the values $\pm \infty$. Hence we cannot say that $\calL^1_\Rb(\mu)$ is a linear space.
	
	However, $\calL^1(\mu)$, the set of all $\mu$-integrable real-valued functions is a linear space.
\end{remark}

\section{Restricting the domain}

Let $(X, \sa, \mu)$ be a measure space and $u \in \calL^1_\Rb(\mu)$. For any $A \in \sa$, the function $\ind_A \cdot u \in \calM_\Rb(\sa)$ and $\abs{\ind_A \cdot u} \leq \abs{u}$. Hence, by \autoref{thm:characterisation-mu-integral} (4.) we have that $\ind_A \cdot u \in \calL^1_\Rb(\sa)$.

Similarly, if $(X, \sa, \mu)$ is a measure space and $u \in \calM_\Rb^+(\sa)$, for any $A \in \sa$ we can define the function $\ind_A \cdot u \in \calM_\Rb^+(\sa)$ by \autoref{cor:properties-measurable} since $\ind_A \cdot u = \min (\ind_A, u)$ in this case (since $u$ is non-negative).

When we wish to restrict the domain of integration, we write,
\begin{align}
	\int_A u d\mu := \int \ind_A u d\mu.
\end{align}

Note that if $u \in \calL^1_\Rb(\mu)$ $\int_A u d\mu < \infty$ but if $u \in \calM^+_\Rb(\sa)$ then $\int_A u d\mu$ is well defined but can take the value $+\infty$.

We do we keep coming back to the case $u \in \calM^+_\Rb(\sa)$? Well, because in this case we can associate a new measure on $(X, \sa)$ as follows:

\begin{lem}
	Let $(X, \sa, \mu)$ be a measure space and $u \in \calM^+_\Rb(\sa)$. Define $\nu : \sa \to [0, \infty]$ by
	\begin{align}
		\nu(A) := \int_A ud\mu = \int \ind_A u d\mu.
	\end{align}
	Then $\nu$ is a measure on $(X,\sa)$. Moreover, if $\int u d\mu < \infty$, i.e. $u$ is, \textbf{additionally}, in  $\calL_\Rb^1(\sa)$, then $\nu$ is a finite measure.
\end{lem}

\begin{proof}
	As usual,
	\begin{enumerate}
		\item
		\begin{align*}
			\nu(\emptyset) = \int_\emptyset u d\mu = \int 0 \cdot u d\mu = 0
		\end{align*}
		
		\item Let $(A_n)_{n\in\N}$ be a sequence of pairwise disjoint sets in $\sa$. Then
		\begin{align*}
			\ind_{\bigcupdot_{n\in\N} A_n} = \sum_{n\in\N} \ind_{A_n},
		\end{align*}
		and thus,
		\begin{align*}
			\nu\left(\bigcupdot_{n\in\N} A_n\right)
			&= \int_{\bigcupdot_{n\in\N} A_n} u d\mu \\
			&= \int \ind_{\bigcupdot_{n\in\N} A_n} u d\mu \\
			&= \int \left(\sum_{n\in\N} \ind_{A_n}\right) u d \mu \\
			&= \int \left(\sum_{n\in\N}\ind_{A_n} u\right) d\mu \\
			&\overset{\ref{cor:integral-countable-sum}}{=} \sum_{n\in\N} \int \ind_{A_n} u d\mu
			= \sum_{n\in\N} \nu(A_n).
		\end{align*}
	\end{enumerate}
\end{proof}

\section{Examples}

\begin{eg}
	Let $(X, \sa, \delta_A)$ be a measure space where $\delta_A$ is the Dirac measure for a fixed $A \in \sa$ (see \autoref{eg:dirac-measure}). Then for $u: \sa \to \R, u \geq 0$ we have
	\begin{align*}
		\int u d\delta_A = \int u(x) \delta_A(dx) = u(A).
	\end{align*} 
	Then, if we generalise $u \in \calM_\Rb(\sa)$ we have
	\begin{align*}
		\int u d\delta_A = \int (u^+ - u^-)d\delta_A = \int u^+d\delta_A - \int u^- d \delta_A = u^+(A) - u^-(A) = u(A).
	\end{align*}
	
	We conclude that $u \in \calL_\Rb^1(\delta_A) \iff \abs{u(A)} < \infty$.
\end{eg}

\begin{eg}
	Consider the measure space $(\N, \powerset(\N), \mu)$ where $\mu$ is the discreet measure given by
	\begin{align*}
		\mu = \sum_{n\in\N} \alpha_n \delta_{\{n\}}.
	\end{align*}
	
	We have seen that
	\begin{align*}
		\int \abs{u} d\mu = \sum_{n \in \N} \alpha_n \abs{u(n)}.
	\end{align*}
	Thus, $u \in \calL_\Rb^1(\mu) \iff \sum_{n \in \N} \alpha_n \abs{u(n)} < \infty$. Note that we can write
	\begin{align*}
		u(m) = \sum_{n \in \N} u(n) \ind_{\{n\}}(m),
	\end{align*}
	so if $\alpha_1 = \alpha_2 = \dots = 1$ then $u \in \calL_\Rb^1(\mu) \iff \sum_{n \in \N} \abs{u(n)} < \infty$. In particular, it is necessary that $u(n) \in \R,\ \forall n \in \N$. We can in this case express
	\begin{align*}
		\calL^1_\Rb(\mu) = \left\{ (x_n)_{n\in\N} \mid x_n \in \R \land \sum_{n\in\N} \abs{x_n} < \infty \right\} := \call^1(\N).
	\end{align*}
	We call $\call^1(\N)$ the set of all summable sequences. This space is important in functional analysis.
\end{eg}

\begin{eg}
	If $(X, \sa, \mu)$ is a finite measure space, then any bounded function $u \in \calM_\Rb(\sa)$ is $\mu$-integrable. That is, if for any given $u \in \calM_\Rb(\sa)$ there exists $M \in \R,\ M > 0$ such that $\abs{u(x)} \leq M,\ \forall x \in \sa$ then $u$ is $\mu$-integrable.
	\begin{align*}
		\int \abs{u}d\mu = \int \abs{u} \ind_X d\mu \leq \int M \int_X d\mu = M \int \ind_X d\mu = M \mu(X) < \infty.
	\end{align*}
\end{eg}
	% !TeX root = ../mi-notes.tex

\chapter{Null sets and the notation almost everywhere}

In this chapter we formally introduce the concept of a null set and the handy notation almost everywhere, which gives us a way to formalise ideas that we get intuitively.

\begin{dfn}
	\label{dfn:mu-null-set}
	Let $(X, \sa, \mu)$ be a measure space. The collection of $\mu$-null sets is defined by
	\begin{align}
		\calN_\mu = \left\{ A \in \sa \mid \mu(A) = 0 \right\}
	\end{align}
\end{dfn}

Note that $\calN_\mu$ is a \siga, not necessarily on $X$, contained in $\sa$.

\begin{dfn}
	\label{dfn:almost-everyhere}
	We say that a property $\Pi = \Pi(x)$ holds $\mu$ almost everywhere, in short $\muae$ or a.e., if there exits
	\begin{align}
		N \in \calN_\mu \text{ such that } \{x \in X \mid \Pi(x) \text{ fails }\} \subset N.
	\end{align}
\end{dfn}

Note that we do not require that we do not require that $\{x \in X \mid \Pi(x) \text{ fails }\}$ be a measurable set, it only has to be a subset of a measurable set. In practice, this set is indeed a $\mu$-null set itself.

\begin{eg}
	Let $\Pi$ be the property that $u(x) = v(x)$ for $u, v \in \calM_\Rb(\sa)$. Then both $\{x \in X \mid u(x) = v(x)\}$ and $\{x \in X \mid u(x) \neq v(x)\}$ are measurable (since $u, v : \sa \to \Rb$). In this case, saying that $\Pi$ holds a.e. means that $\{x \in X \mid u(x) \neq v(x)\} \in \calN_\mu$.
\end{eg}

Before moving on, we will prove a very handy inequality.

\begin{thm}[Markov inequality]
	\label{thm:markov-inequality}
	Let $u \in \calL_\Rb^1(\mu),\ A \in \sa$ and $c > 0$. Then,
	\begin{align}
		\mu(\{\abs{u} \geq c\} \cap A) \leq \frac{1}{c} \int_A \abs{u} d\mu.
	\end{align}
	
	In particular if $A = X$ then
	\begin{align*}
		\mu(\{\abs{u} \geq c\}) \leq \frac{1}{c} \int \abs{u}d\mu.
	\end{align*}
\end{thm}

Informally, this means that an integrable function does not blow up (too much), i.e. an integrable function only attains large values on a set with a small measure.

\begin{proof}
	\begin{align*}
		\mu(\{u \geq c\} \cap A)
		&= \int 1_{\{\abs{u} \geq c\} \cap A} d\mu \\
		&= \frac{c}{c} \int \ind_{\{\abs{u} \geq c\}} \cdot \ind_A d\mu\\
		&= \frac{1}{c} \int_A c \ind_{\{\abs{u} \geq c\}} d\mu\\
		&\leq \frac{1}{c} \int_A \underbrace{\abs{u}\ind_{\{\abs{u} \geq c\}}}_{\geq \abs{u}} d\mu
		\leq \frac{1}{c} \int_A \abs{u} d\mu
	\end{align*}
\end{proof}

\begin{thm}
	\label{thm:null-integral-ae}
	Let $u \in \calL_\Rb^1(\mu)$ be a numerical integrable function on a measure space $(X, \sa, \mu)$. Then,
	\begin{enumerate}
		\item
		\begin{align}
			\int \abs{u}d\mu = 0 \iff \abs{u} = 0 \muae \iff \mu\left(\left\{ x \in X \mid u(x) \neq 0 \right\}\right) = 0
		\end{align}
		\item For any $N \in \calN_\mu$, we have
		\begin{align}
			\ind_N u \in \calL^1_\Rb(\mu) \text{ and }\int_N u d\mu = 0.
		\end{align}
	\end{enumerate}
\end{thm}

\begin{proof}
	Let us start with 2. Define $f_n = \min\{ \abs{u}, n\}$. Then, $f_n \in \calM_\Rb^+(\sa)$ and $f_n \uparrow \abs{u}$ and $\ind_N \cdot f_n \uparrow \ind_N \cdot \abs{u}$. By \autoref{thm:beppo-levi}, 
	\begin{align*}
		\int_N \abs{u} d\mu
		&= \int \ind_N \abs{u}d\mu\\
		&= \int \sup_{n\in \N} \ind_N f_n d\mu\\
		&= \sup_{n\in\N} \int \ind_N f_n d\mu\\
		&\leq \sup_{n\in \N} \int n \ind_N d\mu = \sup_{n\in\N} n \mu(N) = \sup_{n\in\N} 0 = 0.
	\end{align*}
	Thus, $\int \ind_N \abs{u} d\mu = 0 \implies \ind_N u \in  \calL_\Rb^1(\mu)$. Now,
	\begin{align*}
		0 \leq \abs{ \int_n u d\mu } \leq \int_N \abs{u} d\mu = 0,
	\end{align*}
	thus $\int_n u d\mu = 0$.
\end{proof}

Now we move on to 1. Assume $\int \abs{u}d\mu = 0$. Then,
\begin{align*}
	\mu(\{u \neq 0 \})
	&= \mu(\{ \abs{u} > 0 \}) \\
	&= \mu \left( \bigcup_{n = 1}^\infty \{\abs{u} \geq \frac{1}{n}\} \right) \\
	&\overset{\ref{thm:prop-measures}}{\leq} \sum_{n=1}^\infty \mu(\{ \abs{u} \geq \frac{1}{n} \}) \\
	&\overset{\ref{thm:markov-inequality}}{\leq} \sum_{n=1}^\infty n \int \abs{u}d\mu \\
	&= \sum_{n=1}^\infty 0 = 0.
\end{align*}

Thus $\abs{u} = 0$ a.e.

Conversely, assume $\abs{u} = 0$ a.e. Then $\{\abs{u} \neq 0 \} \in \calN_\mu$ and 
\begin{align*}
	\int \abs{u} d\mu
	&= \int \left( \abs{u} \ind_{\{\abs{u} = 0\}} + \abs{u} \ind_{\{ \abs{u} = 0 \}} \right)d \mu\\
	&= \int \underbrace{\abs{u}}_{= 0} \ind_{\{\abs{u} = 0\}} d\mu + \underbrace{\int \abs{u} \ind_{\{ \abs{u} = 0 \}} d\mu}_{= 0 \text{ by part 2.}} = 0
\end{align*}

\textbf{This theorem is very important and used widely as it allows us to change the value of a function on a $\mu$-null set without changing the value of the integral.}

\begin{cor}
	Let $u, v \in \calM_\Rb(\sa)$ be such that $u = v$ a.e. Then,
	\begin{enumerate}
		\item if $u, v \geq 0$, then $\int u d\mu = \int v d\mu$ (though it is possible that both sides are $+\infty$).
		\item $u \in \calL^1_\Rb(\sa) \iff v \in \calL_\Rb^1(\sa)$ and, also in this case $\int u d\mu = \int v d\mu$.
	\end{enumerate}
\end{cor}

\begin{proof}
	Part 1. Assume $u, v \in \calM_\Rb^+(\sa)$ with $u = v$ a.e.
	
	\begin{align*}
		\int u d\mu
		&= \int \left( u \ind_{\{u = v\}} + u \ind_{\{u \neq v\}} \right)d\mu \\
		&= \int \underbrace{u \ind_{\{u = v\}}}_{= v} d\mu + \underbrace{\int u \ind_{\{u \neq v\}} d\mu}_{0 \text{ by \autoref{thm:null-integral-ae} }} \\
		&= \int v \ind_{\{u = v\}} d\mu + \underbrace{\int v \ind_{\{u \neq v\}} d\mu}_{0 \text{ by \autoref{thm:null-integral-ae} }} \\
		&= \int \left( v \ind_{\{u = v\}} + v \ind_{\{u \neq v\}} \right)d\mu
		= \int v d\mu.
	\end{align*}
	
	Part 2 follows from part 1. Note that if $u = v$ a.e. then $u^+ = v^+$ and $u^+ = v^-$ a.e. So if $u \in \calL_\Rb^1(\mu)$, then by part 1,
	\begin{align*}
		\max\left(\int u^+d\mu, \int u^-d\mu\right) = \max \left( \int v^+ d\mu, \int v^- d\mu \right), 
	\end{align*}
	which implies that $v \in \calL^1_\Rb(\mu)$ and
	\begin{align*}
		\int u d\mu
		= \int u^+ d\mu - \int u^- d\mu
		\overset{\text{part 1}}{=} \int v^+d\mu - \int v^- d\mu
		= \int v d\mu.
	\end{align*}
\end{proof}

Another consequence of \autoref{thm:null-integral-ae} is that integrable functions take values in $\R$ a.e.

\begin{cor}
	Let $u \in \calL^1_\Rb(\mu)$. Then $u$ takes values in $\R \muae$, i.e.
	\begin{align*}
		\mu(\{x \in X \mid \abs{u(x)} = +\infty\}) = \mu(\{x \in X \mid \abs{u} = \infty\}) = 0.
	\end{align*}
	
	In particular, we can find $v \in \calL^1_\Rb(\mu)$ such that $u = v \muae$ and $\int u d\mu = \int v d\mu$.
\end{cor}

\begin{proof}
	TODO.
\end{proof}

As a consequence of this corollary, we can now restrict our attention to $\calL^1(\mu)$ without loss of generality.

Finally, we give a result that is helpful for verifying that two functions are equal $\muae$

\begin{cor}
	Let $\calB \subseteq \sa$ be a \siga.
	
	\begin{enumerate}
		\item If $u, w \in \calL^1(\calB)$ (so $u, w \in \calM(\calB)$) and $\int_B u d\mu = \int_B w d\mu$ for all $B \in \calB$, then $u = w\muae$
		
		\item If $u, w \in \calM^+(\calB)$ and $\int_B u d\mu = \int_B w d\mu$ for all $B \in \calB$ and $\mu\mid_\calB$ is $\sigma$-finite then $u = w \muae$
	\end{enumerate}
\end{cor}

\begin{proof}
	TODO
\end{proof}
	% !TeX root = ../mi-notes.tex

\chapter{Convergence theorems}

The purpose of this chapter is to generalise the results that were discussed in chapter 7, namely \autoref{thm:beppo-levi}, \autoref{thm:fatou-lemma} and the reverse Fatou lemma. By generalisation we mean that we will remove the restrictions about positive functions and/or increasing sequences.

\section{Convergence theorems}

First, we extend \autoref{thm:beppo-levi} to negative functions.
\begin{thm}[Monotone convergence]
	\label{thm:monotone-convergence}
	Let $(X, \sa, \mu)$ be a measure space and $(u_n)_{n\in\N} \subset \calL^1(\mu)$ be an increasing sequence of measurable functions such that
	\begin{align*}
		u_n \leq u_{n+1},\ \forall n \in \N\text{ and define } u = \sup_{n\in \N} u_n = \lim_{n \to \infty} u_n \in \calM_\Rb(\sa).
	\end{align*}
	Then,
	\begin{align}
		\label{eq:monotone-convergence-1}
		u \in \calL^1(\mu) \iff \sup_{n\in\N} \int u_n d\mu < \infty,
	\end{align}
	and in that case
	\begin{align}
		\label{eq:monotone-convergence-2}
		\int u d\mu = \int \lim_{n \to \infty} u_n d\mu = \int \sup_{n\in\N} u_n = \sup_{n\in\N} \int u_n d\mu = \lim_{n \to \infty} \int u_n d\mu.
	\end{align}
\end{thm}

Here we're writting both the $\lim$ and the $\sup$ versions of the equalities to be extra clear at the expense of being too verbose. This is not the case in \cite[p- 88]{schilling2017}.

\begin{proof}
	We want to use Beppo-Lévi but the functions may be negative, so we define a new sequence $w_n = u_n - u_1 \geq 0$ since $u_1 \leq u_n,\ \forall n \in \N$. Clearly, $w_n \in \calL^1(\mu) \subset \calM(\sa)$ and $0 \leq w_1 \leq w_2 \leq \dots$.
	
	Firstly, by \autoref{thm:beppo-levi},
	\begin{align*}
		0 \leq \int \sup_{n\in\N} w_n d\mu = \sup_{n\in\N} \int w_n d\mu. 
	\end{align*}
	Equivalently,
	\begin{align*}
		0 &\leq \int \sup(u_n - u_1)d\mu \\
		&= \sup_{n\in\N}\left(\int (u_n - u_1)d\mu \right)  \\
		&= \sup_{n\in\N}\left( \int u_n d\mu - \int u_1 d\mu \right) \\
		&= \sup_{n\in\N}\int u_n d\mu - \int u_1 d\mu.
	\end{align*}
	Note how we wait until $\sup$ is outside the integral to apply linearity, since we don't know if $\sup(u_n - u_1) \in \calL^1(\mu)$ yet.
	
	Assume $u = \sup_n u_n \in \calL^1(\mu)$. Then $\sup_n (u_n - u_1) = \sup_n u_n - u_1 \in \calL^1(\mu)$, and the above gives
	\begin{align*}
		\int \sup_n u_n d\mu - \int u_1 d\mu = \sup_n \int u_n d\mu - \int u_1 d\mu.
	\end{align*}
	Since $u_1 \in \calL^1(\mu)$, its integral is finite and we can subtract it from both sides to get
	\begin{align*}
		\sup_n \int u_n d\mu = \int \sup_n u_n d\mu < \infty.
	\end{align*}
	
	Conversely, assume $\sup_n \int u_n d\mu < \infty$. Then,
	\begin{align*}
		\int \sup_n (u_n - u_1)d\mu = \sup_n \int u_n d\mu - \int u_1 d\mu < \infty,
	\end{align*}
	which implies that $\sup(u_n - u_1) = \sup_n u_n - u_1 \in \calL^1(\mu)$. Since $u_1 \in \calL^1(\mu)$, we see that $\sup_n u_n \in \calL^1(\mu)$ as required. Also, if $\sup_n u_n \in \calL^1(\mu)$ the above shows \autoref{eq:monotone-convergence-2}.
\end{proof}

Of course, we can state the same for decreasing sequences and infima, as taking $u_n = - v_n$ for some decreasing sequence $(v_n)$ is enough to fulfill the assumptions of the \autoref{thm:monotone-convergence}. Anyway, we can state the result.

\begin{cor}
	Let $(u_n)_{n\in\N} \subset \calL^1(\mu)$ be a sequence of decreasing integrable functions such that
	\begin{align*}
	u_n \geq u_{n+1},\ \forall n \in \N\text{ and define } u = \inf_{n\in\N} u_n = \lim_{n \to \infty} u_n \in \calM_\Rb(\sa).
	\end{align*}
	Then,
	\begin{align}
	u \in \calL^1(\mu) \iff \inf_{n\in\N} \int u_n d\mu > -\infty,
	\end{align}
	in which case
	\begin{align}
	\int u d\mu 
	= \int \lim_{n \to \infty} u_n d\mu
	= \int \inf_{n\in\N} u_n d\mu
	= \inf_{n\in\N} \int u_n d\mu
	= \lim_{n \to \infty} \int u_n d\mu.
	\end{align}
\end{cor}

\begin{proof}
	Set $v_n = -u_n$ and apply the proof of \autoref{thm:monotone-convergence}.
\end{proof}

Now we move on to a very important result. This time, not only we drop non-negativity, but also, monotonicity. Of course, there is a price to pay for this.

\begin{thm}[Lebesgue Dominated Convergence]
	\label{thm:lebesgue-dominated-convergence}
	Let $(X, \sa, \mu)$ be a measure space and $(u_n)_{n\in\N} \subset \calL^1(\mu)$ be a sequence of functions such that
	\begin{enumerate}
		\item there exists $0 \leq w \in \calL^1(\mu)$ such that $\abs{u_n} \leq w \muae$ for all $n \in \N$, and
		\item $u = \lim_{n\to\infty} u_n$ exists in $\Rb\muae$.  
	\end{enumerate}
	
	Then $u \in \calL^1(\mu)$ and we have
	\begin{align}
		\lim_{n \to \infty} \int |u_n - u| d\mu = 0,
	\end{align}
	\begin{align}
		\lim_{n \to \infty} \int u_n d\mu = \int \lim_{n \to \infty} u_n d\mu = \int ud\mu.
	\end{align}
\end{thm}

\begin{proof}
	TODO
\end{proof}

\begin{remark}
	We have proved the theorem for the case in which $\abs{u_n} \leq w$ everywhere and $\lim_{n\to\infty} u_n \in \Rb$ everywhere, but they may both be relaxed to $\muae$.
\end{remark}

\section{Applications to parameter dependent-integrals}

\begin{thm}[Continuity lemma]
	\label{thm:continuity-lemma}
	
	Let $(X, \sa, \mu)$ be a measure space and $\emptyset \neq (a,b) \subset \R$ a non-degenerate open interval and $u: (a, b) \times X \to \R$ be a function satisfying
	\begin{enumerate}
		\item $x \mapsto u(t, x)$ is in $\calL^1(\mu)$ for every fixed $t \in (a, b)$;
		\item $t \mapsto u(t, x)$ is continuous for every fixed $x \in X$; and
		\item $|u(t, x)| \leq w(x)$ for all $(t, x) \in (a, b) \times X$ and some $w \in \calL^1_\plus(\mu)$.
	\end{enumerate}
	Then the function $v: (a, b) \to \R$ given by
	\begin{align}
		t \mapsto v(t) := \int u(t, x) \mu(dx)
	\end{align}
	is continuous.
\end{thm}

\begin{thm}[Differentiability lemma]
	\label{thm:differentiability-lemma}
	
	Let $(X, \sa, \mu)$ be a measure space, $\emptyset \neq (a, b) \subset \R$ a non-degenerate open interval and $u: (a, b) \times X \to \R$ be a function satisfying
	\begin{enumerate}
		\item $x \mapsto u(t, x)$ is in $\calL^1(\mu)$ for every fixed $t \in (a, b)$;
		\item $t \mapsto u(t, x)$ is differentiable for every fixed $x \in X$; and
		\item $|\partial_t u(t, x)| \leq w(x)$ for all $(t, x) \in (a, b) \times X$ and some $w \in \calL^1_\plus(\mu)$.
	\end{enumerate}
	Then the function $v: (a,b) \to \R$ given by
	\begin{align}
		t \mapsto v(t) := \int u(t, x) \mu(dx)
	\end{align}
	is differentiable and its derivative is
	\begin{align}
		\partial_t v(t) = \int \partial_t u(t, x) \mu(dx).
	\end{align}
\end{thm}

\section{Riemann integral vs. Lebesgue integral}

\begin{thm}
	Let $u: [a, b] \to \R$ be a measurable and Riemann-integrable function. Then $u \in \calL^1(\lambda)$ and the Riemann integral and Lebesgue integral agree.
	\begin{align}
		\int_a^b u dx = \int_{[a, b]} u d\lambda
	\end{align}
\end{thm}

\begin{remark}
	Note that the converse need not be true. For instance the Dirichlet function $\ind_{Q}$ (see \cite{dirichlet}) is Lebesgue integrable but not Riemann-integrable.
\end{remark}

\begin{proof}
	TODO
\end{proof}

\begin{thm}
	Let $u:[a, b] \to \R$ be a bounded and Riemann-integrable function, then
	\begin{align}
		\{x \in X \mid u \text{ is not continuous in } x\} \subseteq N \in \calN_\mu.
	\end{align}
\end{thm}

Note that the additional subset $N$ is not required if $u$ is measurable.

\subsection{Improper Riemann integrals}

\begin{cor}
	\label{cor:improper-riemann}
	Let $u:[a, b] \to \R$ be a measurable function that is Riemann integrable on $[0, N]$ for all $N\geq 1$. Then $u \in \calL^1([0,\infty])$ if, and only if,
	\begin{align}
		\lim_{N\to\infty} \int_0^N \abs{u}dx < \infty.
	\end{align}
	In this case,
	\begin{align*}
		\int_0^\infty udx = \int_{[0, \infty]} u d\lambda.
	\end{align*}
\end{cor}
	% !TeX root = ../mi-notes.tex

\chapter{\texorpdfstring{The function spaces $\calL^p$}{The function spaces Lp}}

\section{\texorpdfstring{A seminorm for $\Lp$}{A seminorm for Lp}}

We turn our attention to functions for which the $p$-th power of their absolute value is integrable. Throughout this chapter we will assume that $(X, \sa, \mu)$ is some measure space.

\begin{dfn}[$\calL^p$-space]
	Let $1 \leq p \in \R$. We define
	\begin{align}
		\calL^p(\mu) &= \{u : X \to \R \mid u \in \calM(\sa) \land \int \abs{u}^p d\mu < \infty  \},\ &p \in [1, \infty) \\
		\calL^\infty(\mu) &= \{u : X \to \R \mid u \in \calM(\sa) \land \exists c > 0,\ \mu(\{\abs{u} \geq c\}) = 0\},\ &p = \infty
	\end{align}
\end{dfn}

As usual we might also refer to these sets by $\calL^p = \calL^p(\sa) = \calL^p(X)$ if the choice of measure is clear or we want to stress the underlying \sigas or domains.

\begin{dfn}[$p$-seminorm]
	\label{dfn:p-seminorm}
	Let $u: X \to \R$ be a measurable function. We define
	\begin{align}
		\norm{u}_p &= \left(\int \abs{u}^p d \mu \right)^{\frac{1}{p}},\ &p \in [1,\infty), \\
		\norm{u}_\infty &= \inf \{ c > 0 \mid \mu\{\abs{u} \geq c \} = 0\},\ &p = \infty.
	\end{align}
\end{dfn}

\begin{remark}
	Of course $u \in \calL^p(\mu) \iff u \in \calM(\sa) \land \norm{u}_p < \infty$.
\end{remark}

It is not coincidental that we use the notation $\norm{\cdot}_p$ which clearly resembles a norm. Indeed, $\abs{\cdot}_p$ is a semi-norm, i.e. a function with the properties of a norm except for the property that identifies $\norm{v}_p = 0 \iff v = \vec{0}$. What happens here is that $\norm{u}_p = 0 \iff u = 0 \muae$ and thus we don't have a one-to-one correspondence between the 0 value of the norm and the vector space's zero element (we will eventually fix this though).

\begin{lem}
	\label{lem:seminorm}
	$\norm{\cdot}_p$ is indeed a seminorm \cite{wiki-norm}, i.e. it satisfies the following
	\begin{enumerate}
		\item (positive homogenous) $\norm{\alpha v}_p = \abs{\alpha}\norm{v}_p,\ \forall \alpha \in \R$
		\item (triangle inequality) $\forall u, v \in \calM(\sa),\ \norm{u + v}_p \leq \norm{u}_p + \norm{v}_p$.
	\end{enumerate}
\end{lem}

\begin{proof}
	\item Let $\alpha \in \R, u \in \calM(\sa)$. We have
	\begin{align*}
		\left(\int \abs{\alpha u}^p d \mu\right)^{\frac{1}{p}}
		= \left(\int \alpha^p \abs{u}^p d \mu\right)^{\frac{1}{p}}
		= \left(\abs{\alpha}^p \int \abs{u}^p d \mu\right)^{\frac{1}{p}}
		= \abs{\alpha} \left(\int \abs{u}^p d \mu\right)^{\frac{1}{p}}
	\end{align*}
	\item It turns out that proving the triangle inequality is not so easy. We will dedicate the rest of this section to it.
\end{proof}

\begin{dfn}[Conjugate numbers]
	Let $p, q \in \R$. We say that $p$ and $q$ are conjugate numbers if
	\begin{align}
		\frac{1}{p} + \frac{1}{q} = 1,
	\end{align}
	hence $q = \frac{p}{p - 1}$.
\end{dfn}

We will not bother giving the previous definition too much formality, for example, if $p = 1$ we can set $\frac{1}{\infty} = 0$ and hence $p = 1$ and $q = \infty$ are conjugate numbers. We will see that this spooky extension does not mess with the definition of $\norm{\cdot}_p$ (cf. \autoref{dfn:p-seminorm}). Also, note that $2$ is the only number which has itself as its conjugate number.

\begin{lem}[Young's inequality]
	\label{lem:young}
	Let $p, q \in (1, \infty)$ be conjugate numbers. Then
	\begin{align}
		AB \leq \frac{A^p}{p} + \frac{B^q}{q}
	\end{align}
	holds for all $A, B \geq 0$. Equality occurs if, and only if, $B = A^{p - 1}$.
\end{lem}

\begin{proof}
	TODO
\end{proof}

\begin{thm}[Hölder's inequality]
	\label{thm:holder}
	Let $u \in \Lp$ and $v \in \calL^q$ where $p, q \in [1,\infty]$ are conjugate numbers. Then $uv \in \calL^1(\mu)$ and
	\begin{align}
		\abs{\int uv d\mu } \leq \int \abs{uv}d\mu \leq \norm{u}_p \norm{v}_q.
	\end{align}
\end{thm}

\begin{proof}
	For the first inequality, see \autoref{thm:properties-mu-integral}.
	
	TODO
\end{proof}

\begin{cor}[Cauchy-Schwarz inequality]
	\label{cor:cauchy-schwarz}
	Hölder's inequality with $p = q = 2$ is called the Cauchy-Schwarz inequality:
	\begin{align}
		\int \abs{uv} d\mu \leq \norm{u}_2 \norm{v}_2.
	\end{align}
\end{cor}

\begin{thm}[Minkowski's inequality]
	\label{thm:minkowski}
	Let $u, v \in \Lp,\ p \in [1, \infty]$. Then the sum $u + v \in \Lp$ and
	\begin{align}
		\norm{u + v}_p \leq \norm{u}_p + \norm{v}_p.
	\end{align}
\end{thm}

Note how we stated that $u + v \in \Lp$. We already had this for $\calL^1$ (see \autoref{thm:properties-mu-integral}) but now we prove it for the more general case.

\begin{proof}
	content...
\end{proof}

We can further generalise this theorem to sequences of non-negative functions using Beppo-Lévi.

\begin{cor}
	\label{cor:minkowski-seq}
	
	Let $(u_n)_{n\in\N} \subset \Lp$ be any sequence of \textbf{non-negative} functions in $\Lp$ with $p \in [1,\infty)$. Then
	\begin{align}
		\norm{\sum_{n=1}^\infty u_n}_p
		\leq \sum_{n = 1}^\infty \norm{u_n}_p.
	\end{align}
\end{cor}

\begin{proof}
	By repeated applications of Minkowski's inequality we have, for any $N \in \N$,
	\begin{align*}
		\norm{\sum_{n=1}^N u_n}_p
		\leq \sum_{n = 1}^N \norm{u_n}_p
		\leq \sum_{n= 1}^\infty \norm{u_n}_p.
	\end{align*}
	Since the right hand side is independent of the choice of $N$, we can take the supremum of the left side without breaking the inequality:
	\begin{align*}
		\sup_{N\in\N} \norm{\sum_{n=1}^N u_n}_p
		\leq \sup_{N\in\N} \sum_{n = 1}^N \norm{u_n}_p
		\leq \sum_{n= 1}^\infty \norm{u_n}_p.
	\end{align*}
	We now wish to see that this holds for $N = \infty$, for which we use Beppo-Lévi. Observe that $(\sum_{n=1}^N u_n)^p$ is a sequence of increasing functions\footnote{$p \geq 1$ so it does not mess with us here}. Thus,
	\begin{multline*}
		\sup_{N\in\N} \norm{\sum_{n=1}^N u_n}_p^p
		= \sup_{N\in\N} \int \left(\sum_{n=1}^N u_n\right)^p d\mu
		= \int \left(\sup_{N \in \N} \sum_{n=1}^N u_n)\right)^p d\mu\\
		\overset{\ref{thm:beppo-levi}}{=} \int \left( \sum_{n=1}^\infty u_n \right)^p
		= \norm{\sum_{n=1}^\infty u_n}_p^p.
	\end{multline*}
	Taking the $p$-th root to get the norms and the proof follows.
\end{proof}

\begin{remark}
	We need the functions to be non-negative so that the function we are integrating, namely
	\begin{align*}
		\abs{\sum_{n=1}^N u_n}^p
	\end{align*}
	is an \textbf{increasing} sequence. At this point we don't really care if it is non-negative or whatever, i.e. using Monotone Convergence instead of Beppo-Lévi would not help as the absolute value in the $p$-norm reduces the question to the non-negative case.
	
	As a counterexample consider a sequence of functions that oscilates around 0: $-1, 0, 1, 0, -1, \dots$. The partial absolute value of the partial sums is $1, 1, 0, 0, 1, \dots$, which does not really converge. Another way of solving this problem would be to ask for a sequence of arbitrary but increasing functions and applying Monotone Convergence, but I feel its more useful to ask for any sequence of non-negative functions as we get the increasing part from taking the partial sums.
\end{remark}



\section{\texorpdfstring{A norm for $\Lp$}{A norm for Lp}}

\begin{remark}
	Note that from \autoref{thm:minkowski} and \autoref{lem:seminorm} we can deduce
	\begin{align}
		u, v \in \Lp \implies \alpha u  + \beta v \in \Lp,\ \forall \alpha, \beta \in \R.
	\end{align}
	which shows that $\Lp$ is an $\R$-vector space.
\end{remark}

In addition, we already saw that $\abs{\cdot}_p$ is a seminorm for $\Lp$ (cf. \autoref{lem:seminorm}). The thing that's missing for $\abs{\cdot}_p$ to be a norm is that $\abs{u}_p = 0 \implies u(x) = 0$ for every $x \in X$ (as opposed to for almost every $x$). There is an easy, but rather technical way to fix this.

\begin{itemize}
	\item We introduce the equivalence relation for $u, v\in \Lp$:
	\begin{align*}
		u \sim v \iff \{u \neq v \} \in \calN_\mu.
	\end{align*}
	It is not hard to verify that $\sim$ is indeed an equivalence relation (see \cite{wiki-equiv} for a definition).
	
	\item The space made up of the equivalence classes
	\begin{align*}
		[u]_p = \{ v \in \Lp \mid u \sim v\}
	\end{align*}
	is called the quotient space and denoted by $L^p = \Lp / \sim$.
	
	\item It is also not hard to see that $L^p$ is a vector space by proving
	\begin{align*}
		[\alpha u + \beta v]_p = \alpha[u]_p + \beta[v]_p.
	\end{align*}
	
	\item Moreover, it admits the norm\footnote{To be honest, I still have not figured out why we require this infimum as the norms are all the same inside of an equivalence class...}
	\begin{align*}
		\norm{[u]_p}_p := \inf \{ \norm{w}_p \mid w \in \Lp \land w \sim u\},
	\end{align*}
	and, fortunately, we have that $\norm{[u]_p}_p = \norm{u}_p$ so we will start abusing notation identify $[u]$ with $u$ and ditch the distinction $L_p$ vs $\Lp$.
\end{itemize}

All the results in this chapter so far are still valid if $u$ is $\muae$ real valued so there is no need to distinguish between the cases $L^p_\Rb$ and $L^p := L^p_R$.

\section{Convergence and completeness}

\begin{dfn}[$\Lp$-convergence]
	\label{dfn:lp-convergence}
	A sequence of functions $(u_n)_{n\in\N} \subset \Lp$ in $\Lp$ is said to be convergent in the space $\Lp$ with limit $\Lplim_{n\to\infty} u_n = u$ if, and only if,
	\begin{align}
		\lim_{n\to\infty} \norm{u_n - u}_p = 0.
	\end{align}
\end{dfn}

Remember, however that $\Lp$-limits are only almost everywhere unique. If $w_1, w_2$ are both $\Lp$-limits of the same sequence $(u_n)_{n\in\N}$ we have
\begin{align*}
	\norm{w_1 - w_2}_p
	\overset{\ref{thm:minkowski}}{\leq}\lim_{n\to\infty} \left( \norm{w_1 - u_n} + \norm{u_n - w_2}_p \right) = 0.
\end{align*}

\begin{remark}
	Pointwise convergence of a sequence $(u_n)_{n\in\N} \subset \Lp$ does not quarantee $\Lp$-convergence.
	
	Example needed...
\end{remark}

We can however, give a weaker result by using Lebesgue's dominated convergence theorem.

\begin{lem}
	Let $(u_n)_{n\in\N} \subset \Lp$ be a sequence of functions in $\Lp$ such that $\lim_{n\to\infty} u_n(x) =  u(x)$ for (almost) every $x$ and $\abs{u_n} \leq w$ for some function $w \in \Lp$. Then, $u \in \Lp$ and $\Lplim u_n = u$.
\end{lem}

\begin{proof}
	We want to show that $\Lplim u_n = u \iff \lim_{n\to\infty} \norm{u_n - u}_p = 0 \iff \lim_{n\to\infty} \norm{u_n - u}_p^p = 0$. To do that we show that the sequence $\abs{u_n - u}^p$ satisfies the hypotheses of \autoref{thm:lebesgue-dominated-convergence}. We have 
	\begin{align*}
	\abs{u_n - u}^p \leq (\abs{u_n} + \abs{u})^p \leq (2w)^p = 2^p w^p
	\end{align*}
	and therefore
	\begin{align*}
	\lim_{n\to\infty} \int \abs{u - u_n}^p d \mu
	= \int \lim_{n\to \infty} \abs{u_n - u} d \mu
	= \int 0d\mu = 0
	\end{align*}
	which in turn implies that $\Lplim u_n = u$.
\end{proof}

\begin{dfn}[$\Lp$-Cauchy sequence]
	\label{dfn:lp-cauchy}
	We call $(u_n)_{n\in\N}$ an $\Lp$-Cauchy sequence if
	\begin{align}
		\forall \varepsilon > 0,\ \exists N_\varepsilon : \forall n_1, n_2 \geq N_\varepsilon,\ \norm{u_{n_1} - u_{n_2}}_p < \varepsilon.
	\end{align}
\end{dfn}

\begin{remark}
	Any $\Lp$-convergent sequence is an $\Lp$-Cauchy sequence.
\end{remark}

\begin{proof}
	By $\Lp$-convergence we have that for any $\varepsilon / 2 > 0$, there exists an $N$ such that for any $n > N$.
	\begin{align*}
		\norm{u_n - u}_p < \frac{\varepsilon}{2}.
	\end{align*}
	
	We can rewrite that as
	\begin{align*}
		\norm{u_{n_1} - u_{n_2}}_p
		\leq \norm{u_{n_1} - u}_p + \norm{u_{n_2} - u}_p
		= \frac{\varepsilon}{2} + \frac{\varepsilon}{2}
		= \varepsilon.
	\end{align*}
\end{proof}

The converse is also true but much harder to prove.

\begin{thm}[Riesz-Fischer]
	The spaces $\Lp,\ p \in [1,\infty]$ are complete, i.e. every $\Lp$-Cauchy sequence $(u_n)_{n\in\N}\subset \Lp$ converges to some limit $u \in \Lp$.
\end{thm}

\begin{proof}
	TODO
\end{proof}

\begin{thm}[Riesz]
	Let $(u_n)_{n\in\N} \subset \Lp,\ p \in [1,\infty)$ be a sequence such that $\lim_{n\to\infty} u_n(x) = u(x)$ for almost every $x \in X$ and some $u \in \Lp$. Then
	\begin{align}
		\lim_{n\to\infty} \norm{u_n - u}_p = 0 \iff \lim_{n\to\infty} \norm{u_n}_p = \norm{u}_p.
	\end{align}
\end{thm}

\begin{proof}
	TODO
\end{proof}

\begin{remark}
	This theorem \textbf{does not hold} for $p = \infty$.
\end{remark}

	% !TeX root = ../mi-notes.tex


\chapter{Exercise sets}

\section{Exercise set 1}

Due September 20th, 2019.


\begin{ex}
	Let $X$ be a nonempty set and $\sa = \{A_1, A_2, \dots\}$ a collection of disjoint subsets of $X$ such that $X = \bigcup_{n = 1}^\infty A_n$. Show that each element $A \in \sigma(\sa)$ is a union of at most a countable subcollection of elements of $\sa$. (3 pts)
\end{ex}

The technique used here is similar to the good set principle (\autoref{rem:good-set-principle}), only that it is not necessary to show that the good set is inside the \siga.

\begin{proof}
	The idea is to prove that $\sigma(\sa)$ only contains countable unions of sets of $\sa$.
	
	Let us define
	\begin{align*}
		\calB = \left\{\bigcup_{i \in I} A_i \mid A_i \in \sa \land I \subset \N\right\}.
	\end{align*}
	
	We shall prove that $\calB$ is a \siga on $\sa$ and thus $\sigma (\sa) \subseteq \calB$, i.e. $\sigma(\sa)$ is made up of unions of at most a countable subcollection of $\sa$.
	
	\begin{enumerate}
		\item Firstly, $\emptyset \in \calB$ since $\emptyset = \bigcup_{i \in I} A_i$ by choosing $I = \emptyset \subset \N$.
		\item Secondly, for any set $B = \bigcup_{i \in I} A_i \in \calB$ we have that
		\begin{align*}
			B^\complement = \left(\bigcup_{i \in I} A_i\right)^\complement = \bigcap_{i \in I} A_i^\complement = \bigcap_{i \in I} \bigcup_{j \neq i} A_j = \bigcup_{j \not\in I} A_j = \bigcup_{j \in I^\complement} A_j \in \calB,
		\end{align*}
		since $I^\complement = \N \setminus I \subset \N$.
		
		\item Finally, for any countable collection $(B_n)_{n \in \N} \in \calB$ we need to show that the union is also in $\calB$.
		\begin{align*}
			\bigcup_{n\in\N} B_n = \bigcup_{n\in\N} \bigcup_{i \in I_n} A_i = \bigcup_{j \in J} A_j \in \calB,
		\end{align*}
		since $J = \bigcup_{n \in \N} I_n \subset \N$.
	\end{enumerate}
	Since we have shown that $\calB$ is a \siga on $\sa$ and thus $\sigma(\sa) \subset \calB$, we conclude that the elements of $\sigma(\sa)$ must be all unions of at most countable subcollections of elements in $\sa$, which are the only elements of $\calB$.
\end{proof}

\begin{ex}
	Let $(X, \calD, \mu)$ be a measure space, and let $\overline{\calD}^\mu$ be the completion of the \siga $\calD$ with respect to the measure $\mu$ (see exercise 4.15). We denote by $\overline{\mu}$ the extension of the measure $\mu$ to the \siga $\overline{\calD}^\mu$. Supose $f: X \to X$ is a function such that $\inv{f}(B) \in \calD$ and $\mu(\inf{f}(B)) = \mu(B)$ for each $B \in \calD$. Show that $\inf{f}(\overline{B}) \in \overline{\calD}^\mu$ and $\overline{\mu}(\inv{f}(\overline{B})) = \overline{\mu}(\overline{B})$ for all $\overline{B} \in \overline{\calD}^\mu$. (3 pts)
\end{ex}

\begin{proof}
	First we show that $\inv{f}(\overline{B}) \in \overline{\calD}^\mu$, for all $\overline{B} \in \overline{\calD}^\mu$. Recall from the definition of the completion $\overline{\calD}^\mu$ of the \siga $\calD$ that any set $\overline{B} \in \overline{\calD}^\mu$ can be written as $\overline{B} = B \cap M$ for some subset $M$ of a $\mu$-measurable null set $N$ in $\calD$. Therefore
	\begin{align*}
		\inv{f}(\overline{B}) = \inv{f}(B \cup M) = \inv{f}(B) \cup \inv{f}(M)
	\end{align*}
	Because $N \supset M$ we also have that $\inv{f}(N) \subset \inv{f}(M)$ and $\mu(\inv{f}(N)) = \mu(N) = 0$ by the definition of $f$. This means that $\inv{f}(M)$ is also a subset of a $\mu$-measurable null set in $\calD$ and since $\inv{f}(B) \in \calD$ by definition of $f$ we have
	\begin{align*}
		\inv{f}(\overline{B}) = \underbrace{\inv{f}(B)}_{\in \calD} \cup \underbrace{\inv{f}(M)}_{\subset \inv{f}(N),\ \mu(\inv{f}(N)) = 0} \in \overline{\calD}^\mu
	\end{align*}
	
	Now we need to verify that $\overline{\mu}(\inv{f}(\overline{B})) = \overline{\mu}(\overline{B})$ for all $\overline{B} \in \overline{\calD}^\mu$. Recall that the extension $\overline{\mu}$ is well-defined in $\overline{\calD}^\mu$ with $\overline{\mu}(\overline{B}) := \mu(B)$ for any $\overline{B} = B \cup M \in \overline{\calD}^\mu$. Hence,
	\begin{align*}
		\overline{\mu}(\inv{f}(\overline{B}))
		&= \overline{\mu}(\inv{f}(B \cup M)) \\
		&= \overline{\mu}(\inv{f}(B) \cup \inv{f}(M)) \\
		&= \mu(\inv{f}(B)) \\
		&= \mu(B) =: \overline{\mu}(\overline{B})
	\end{align*}
\end{proof}

\begin{ex}
	Let $(X, \sa)$ be a measurable space and $\mu : \sa \to [0, \infty)$ a function satisfying
	\begin{enumerate}
		\item $\mu$ is finitely additive
		\item $\mu$ is $\sigma$-subadditive
	\end{enumerate}
	Show that $\mu$ is $\sigma$-additive. (4 pts)
\end{ex}

\begin{proof}
	The plan for the proof is to sandwich $\mu(\bigcup_{n\in\N} A_n)$ between two sums that are the same when taking the limit.
		
	First of all, because of $\sigma$-subadditivity we have that, for any countable collection $(A_n)_{n\in\N} \subset \sa$,
	\begin{align}
		\label{eq:es1.3a}
		\mu(\bigcup_{n\in\N} A_n) \leq \sum_{n\in \N} \mu(A_n),
	\end{align}
	which in particular holds for pairwise disjoint unions, which we will assume from here on. We can rewrite the union as
	\begin{align*}
		\bigcupdot_{n\in \N} A_n = \bigcupdot_{n = 1}^N \cupdot \bigcupdot_{n = N+1}^\infty A_n.
	\end{align*}
	Because of finite additivity we can introduce the measure as
	\begin{align*}
		\mu\left(\bigcupdot_{n\in \N} A_n\right) = \mu\left(\bigcupdot_{n = 1}^N\right) + \mu\left(\bigcupdot_{n = N+1}^\infty A_n\right).
	\end{align*}
	And applying finite additivity again we get
	\begin{align*}
		\mu\left(\bigcupdot_{n\in \N} A_n\right) = \sum_{n = 1}^N \mu(A_n) + \mu\left(\bigcupdot_{n = N+1}^\infty A_n\right).
	\end{align*}
	Since $\mu \geq 0$ we can rearrange the previous expression to obtain
	\begin{align}
		\label{eq:es1.3b}
		\sum_{n = 1}^N \mu(A_n) \leq \sum_{n = 1}^N \mu(A_n) + \mu\left(\bigcupdot_{n = N+1}^\infty A_n\right) = 	\mu\left(\bigcupdot_{n\in \N} A_n\right)
	\end{align}
	By combining \ref{eq:es1.3a} and \ref{eq:es1.3b} we get
	\begin{align*}
		\sum_{i = 1}^N \mu(A_n) \leq \mu\left(\bigcupdot_{n\in \N} A_n\right) \leq \sum_{n \in \N} \mu(A_n)
	\end{align*}
	Taking the limit as $N \to \infty$, which we can do since $N$ is not inside the arguments to $\mu$ (in that case it would require that $\mu$ was already a measure, which it isn't, yet), we have
	\begin{align*}
		\sum_{n \in \N} \mu(A_n) \leq \mu\left(\bigcupdot_{n\in \N} A_n\right) \leq \sum_{n \in \N} \mu(A_n) \implies \mu\left(\bigcupdot_{n\in \N} A_n\right) = \sum_{n \in \N} \mu(A_n),
	\end{align*}
	or that $\mu$ is $\sigma$-additive.
\end{proof}


\section{Exercise set 2}

Due September 27th, 2019.

\begin{ex}
	Let $\Q$ be the set of all real rational numbers and let $\calI_\Q = \{[a, b)_\Q \mid a,b \in \Q\}$ where $[a,b)_\Q = \{q \in \Q \mid a \leq q < b\}$.
	\begin{enumerate}
		\item Prove that $\sigma(\calI_\Q) = \powerset(\Q)$ where $\powerset(\Q)$ is the collection of all subsets of $\Q$. (1.5 pts.)
		\item Let $\mu$ be the counting measure on $\powerset(\Q)$ and let $\nu = 2\mu$. Show that $\nu(A) = \mu(A)$ for all $A \in \calI_\Q$, but $\nu \neq \mu$ on $\sigma(\calI_\Q) = \powerset(\Q)$. Why doesn't this contradict Theorem 5.7 in your book? (1.5 pts.)
	\end{enumerate}
\end{ex}

\begin{proof}$ $\newline
	\begin{enumerate}
		\item We shall prove the double containment. First, recall that $\powerset(\Q)$ is a \siga on $\Q$. Also, by Remark 3.5 we have that $\calI_\Q \subseteq \powerset(\Q) \implies \sigma(\calI_\Q) \subseteq \sigma(\powerset(\Q)) \subseteq \powerset(\Q)$. The last inclusion comes from the fact that $\powerset(\Q)$ is also a \siga on $\Q$ so it must contain the smallest \siga on $\Q$ that contains information about $\powerset(\Q)$. For the reverse containment, we shall prove that any subset $A \in \powerset(\Q)$ is also in $\sigma(\calI_\Q)$. For any $A \subset \Q$ define $(q_n)_{n \in \N}$ to be an enumeration of the rationals in $A$. This is posible since $\# \Q = \# \N$. Therefore, we can write
		\begin{align*}
			A = \bigcup_{n \in \N} \{q_n\},\text{ where } \{q_n\} \in \sigma(\calI_\Q)
		\end{align*}
		Therefore, $A \in \sigma(\calI_\Q)$ because \sigas are closed under countable union.
		
		\item It is clear that any interval $A \in \calI_\Q$ contains infinitely many rationals, except if the interval is empty, i.e. $a = b \implies [a,b) = \emptyset$. Therefore,
		\begin{align*}
			\mu(A) = \nu(A) = \begin{cases}
			0       \tif A = \emptyset \\
			\infty &\text{ otherwise }
			\end{cases}
		\end{align*}
		But, if we consider $A \in \sigma(\calI_\Q) = \powerset(\Q)$ then we have some finite sets where $\mu(A) = \# A$, and clearly $\nu(A) = 2 \# A$. The equality between $\mu$ and $\nu$ only holds when the set is either empty or infinite, but not for finite sets such as $A = \{1, 2\}$ where $\mu(A) = 2$ but $\nu(A) = 4$.
		
		Why doesn't this contradict Theorem 5.7? Even though there is an exhausting sequence in the generator, namely $(A_n)_{n\in\N}$ where $A_n = [-n, n)_\Q$, the measure is not finite for any $A_n$. Moreover, there cannot be any exhausting sequence in $\calI_\Q$ with a finite measure because we already saw that $\mu(A) = \infty,\ \forall A \in \calI_\Q,\ A \neq \emptyset$.
	\end{enumerate}
\end{proof}

\begin{ex}
	Let $X$ be a set and $\mu, \nu:\powerset(X) \to [0, \infty)$ two outer measures on $X$. Define $\rho: \powerset(X) \to [0, \infty)$ by $\rho(A) = \max(\mu(A), \nu(A))$. Show that $\rho$ is another outer measure on $X$.
\end{ex}

\begin{proof}
	Firstly, from the definition of $\rho$ we can see that the domain and codomain are compatible with the definition of an outer measure. Next, we prove each of the properties of an outer measure.
	\begin{enumerate}
		\item $\rho(\emptyset) = \max(\mu(\emptyset), \nu(\emptyset)) = \max (0, 0) = 0$
		\item For any $A, B \in \powerset(X),\ A \subseteq B$ we have
		\begin{align*}
			\rho(A) =\max(\mu(A), \nu(A)) \leq \max (\mu(B), \nu(B)) = \rho(B)
		\end{align*}
		\item For any sequence $(A_n)_{n\in \N} \subset \powerset(X)$ we must verify that $\rho(\bigcup_{n \in \N} A_n) \leq \sum_{n\in \N} \rho(A_n)$.
		
		Let us prove the following first. For any function $f: X \times Y \to \R$ we have
		\begin{align*}
			\max_{x \in X} \sum_{y \in Y} f(x, y) \leq \sum_{y \in Y} \max_{x \in X} f(x, y)
		\end{align*}
		Choose any $x_0 \in X$ and any $y_0 \in Y$ and we have that $f(x_0, y_0) \leq \max f(x_0, y_0)$. Hence $\sum_{y \in Y} f(x_0, y) \leq \sum_{y \in Y} f(x, y)$. Because this is true for all $x_0 \in X$ we have $\max_{x \in X} \sum_{y \in Y} f(x, y) \leq \sum_{y \in Y} \max_{x \in X} f(x,y)$.
		
		Using this, i.e. choosing $Y = \N,\ X = \{\mu, \nu\}$ and defining $f(\mu, n) = \mu(A_n)$ and $f(\nu, n) = \nu(A_n)$ we have
		\begin{multline*}
			\rho(\bigcup_{n \in \N} A_n) = \max\left\{\mu\left(\bigcup_{n \in \N} A_n\right), \nu\left(\bigcup_{n \in \N} A_n\right) \right\} = \max \left\{\sum_{n\in \N} \mu(A_n), \sum_{n\in \N} \nu(A_n) \right\} \\
			\leq \sum_{n\in \N} \max\{\mu(A_n), \nu(A_n)\} = \sum_{n\in \N} \rho(A_n)
		\end{multline*}
	\end{enumerate}
\end{proof}


\begin{ex}
	Let $(X, \sa, \mu)$ be a measure space. For $A \in \sa$ let
	\begin{align*}
		S(A) := \{B \in \sa \mid B \subset A, \mu(B) < \infty\}.
	\end{align*}
	
	Define $\nu: \sa \to [0, \infty]$ by $\nu(A) := \sup\{ \mu(B) \mid B \in S(A)\}$.
	
	\begin{enumerate}
		\item Show that $\nu$ is monotone, i.e. if $A_1, A_2 \in \sa$ such that $A_1 \subseteq A_2$, then $\nu(A_1) \leq \nu(A_2)$. (0.5 pts).
		\item Show that if $A \in \sa$ with $\mu(A) < \infty$, then $\nu(A) = \mu(A)$. (1 pt.)
		\item Show that $\nu$ is a measure on $\sa$. (2.5 pts.)
		\item Show that if $\mu$ is $\sigma$-finite, the $\mu = \nu$. (1 pt.)
	\end{enumerate}
\end{ex}

\begin{proof}
	$ $\newline
	\begin{enumerate}
		\item We first show that if $A_1 \subseteq A_2$, then $S(A_1) \subseteq A_2$. Let $C \in S(A_1)$, then $C \in \sa \land C \subset A_1 \land \mu(C) < \infty$. Clearly $C \subseteq A_2$ since $A_1 \subseteq A_2$ so $C \in S(A_2)$. Finally,
		
		\begin{align*}
			\nu(A_1) = \sup \{\mu(B) \mid B \in S(A_1)\} \leq \sup \{\mu(B) \mid B \in S(A_2)\} = \nu(A_2).
		\end{align*}
		
		\item Clearly the domain and range of $\nu$ are compatible with the definition of a measure, i.e. $\nu: \sa \to [0, \infty]$. We need to show that the two properties from \autoref{dfn:measure} hold.
		\begin{enumerate}
			\item $\nu(\emptyset) = \sup \{\nu(B) \mid B \in S(\emptyset)\} = \mu(\emptyset) = 0$ since $S(\emptyset) = \{ \emptyset\}$.
			\item Let $(A_n)_{n\in\N} \subset \sa$ be a pairwise disjoint sequence of sets in $\sa$. Then,
			\begin{align*}
				\nu\left(\bigcupdot_{n \in \N} A_n\right) = 
			\end{align*}
		\end{enumerate}
	\end{enumerate}
\end{proof}


\section{Exercise set 3}

Due October 4th, 2019.

\begin{ex}
	Let $(X, \sa, \mu)$ be a measure space and $\calG=\{A_1, A_2, \dots\}$ a countable partition of $X$ with $A_k \in \sa,\ \forall k \in \N$. Define a function $u : X \to \R$ by $u(x) = \sum_{k = 1}^{\infty} k \cdot \ind_{A_k}$.
	
	\begin{enumerate}
		\item Show that $u$ is $\sa/\borel(\R)$-measurable. (1.5 pts)
		\item Show that $\sigma(\mu) = \sigma(\calG)$, where $\sigma(\mu)$ is the smallest \siga making $u$ measurable. (2.5 pts)
		\item Suppose that $0 < \mu(A_n) < \infty$ for all $n \in \N$. Define $\nu$ on $\sa$ by
		\begin{align*}
		\nu(B) = \sum_{n = 1}^{\infty} 3^{-n} \cdot \frac{\mu(B \cap A_n)}{\mu (A_n)}.
		\end{align*}
		Show that $\nu$ is a \textbf{finite} measure on $(X, \sa)$. (1.5 pts)
		
		\item Under the assumptions of part 3, prove that if $B \in \sa$, then $\mu(B) = 0$ if and only if $\nu(B) = 0$. (1 pt)
	\end{enumerate}
\end{ex}

\begin{proof}$ $\newline
	\begin{enumerate}
		\item Observe that the function $u(x)$ always evaluates to a natural number. In particular, $u(x) = k$ for the $k \in \N$ that satisfies $x \in A_k$. $u$ is well-defined since $\calG$ is a countable partition of $X$.
		
		We must check whether for any $B \in \borel(\R), \inv{u}(B) \in \sa$. Given a $B \in \borel(\R)$ we rewrite it as $B = B' \cup B_{nat}$ where $B_{nat}$ contains all the naturals in $B$ and $B' = B \setminus B_{nat}$. Then,
		\begin{align*}
		\inv{u}(B) = \inv{u}(B' \cup B_{nat}) = \inv{u}(B') \cup \inv{u}(B_{nat}) = \emptyset \cup \bigcup_{n \in B_{nat}} A_n \in \sa.
		\end{align*}
		
		\item By definition of $\sigma(u)$ we have
		\begin{align*}
		\sigma(u) := \sigma (\inv{u}(\borel(\R))) = \inv{u}(\borel(\R)),
		\end{align*}
		since the preimage of any \siga is a \siga.
		
		Also, recall from exercise set 1 that
		\begin{align*}
		\sigma(\calG) = \left\{ \bigcup_{i \in I} A_i \mid A_i \in \calG,\ I \subset \N \right\}.
		\end{align*}
		
		Now we prove the double containment. Take any $A \in \sigma(u)$. Then there is a $B \in \borel(\R)$ such that $A = \inv{u}(B)$. By part one we already now that
		\begin{align*}
		A = \emptyset \cup \bigcupdot_{n \in B_{nat}} A_n \in \sigma(\calG).
		\end{align*}
		
		For the reverse containment, let $G \in \sigma(\calG)$, therefore there is a set $I \subset \N$ such that $G = \bigcup_{i \in I} A_i$. Furthermore, $I \in \borel(\R)$ and thus, by part one,
		\begin{align*}
		\bigcup_{i \in I} A_i = \inv{u}(I) \in \sigma(u).
		\end{align*}
		
		\item First let us check that $\nu$ is well defined. $\nu(B)$ is well defined for any $B \in \sa$ since $B \cap A_n \in \sa$ as $\sa$ is \istable. Also, $\nu$ is non-negative since it is computed as a sum of products of non-negative numbers.
		
		Now we check the two properties in the definition of measure:
		\begin{enumerate}
			\item
			\begin{align*}
			\nu(\emptyset) = \sum_{n=1}^{\infty} 3^{-n} \cdot \frac{\mu(\emptyset \cap A_n)}{\mu(A_n)} = \sum_{n=1}^{\infty} 3^{-n} \cdot \frac{\mu(\emptyset)}{\mu(A_n)} = \sum_{n=1}^{\infty} 3^{-n} \cdot \frac{0}{\mu(A_n)} = 0
			\end{align*}
			\item For any pairwise disjoint collection $(B_n)_{n\in \N} \subset \sa$ we have
			\begin{align*}
			\nu\left(\bigcupdot_{j\in \N} B_j\right) &= \sum_{n = 1}^\infty 3^{-n} \cdot \frac{\mu\left(\left(\bigcupdot_{j\in \N} B_j\right) \cap A_n\right)}{\mu(A_n)}\\
			&= \sum_{n = 1}^\infty 3^{-n} \cdot \frac{\mu\left(\bigcupdot_{j\in \N} (B_j \cap A_n)\right)}{\mu(A_n)} \\
			&= \sum_{n = 1}^\infty 3^{-n} \cdot \frac{\sum_{j \in \N}\mu\left(B_j \cap A_n\right)}{\mu(A_n)} \\
			&= \sum_{j\in\N} \sum_{n=1}^{\infty} 3^{-n}\cdot \frac{\mu\left(B_j \cap A_n\right)}{\mu(A_n)} = \sum_{j \in \N} \nu(B_j)
			\end{align*}
		\end{enumerate}
		
		Additionally, we must check that $\nu$ is finite. Since $\mu$ is a measure, it is monotone so $\mu(B \cap A_n) \leq \mu(A_n)$ and thus
		\begin{align*}
		\nu(B) = \sum_{n = 1}^{\infty} 3^{-n} \cdot \frac{\mu(B \cap A_n)}{\mu (A_n)} \leq \sum_{n = 1}^{\infty} 3^{-n} < \infty
		\end{align*}
		
		
		\item It is clear that if $\mu(B) = 0$ then, by monotonicity, since $B \cap A_n \subset B$, we have that $0 \leq \mu(B \cap A_n) \leq \mu(B) = 0$ and therefore $\nu(B) = \sum_{n = 1}^\infty 3^{-n} \cdot 0 = 0$.
		
		For the reverse, we have that if $\nu(B) = 0$, then it must be because $\mu(B \cap A_n) = 0,\ \forall n \in \N$ since $3^{-n} > 0,\ \forall n \in \N$. We can rewrite $\mu(B)$ as
		\begin{align*}
		\mu(B) = \mu(B \cap X) = \mu\left(B \cap \bigcupdot_{j\in \N} A_j\right) = \mu\left(\bigcupdot_{j\in \N} B \cap A_j\right) \\
		= \sum_{j \in \N} \mu(B \cap A_j) = \sum_{j \in \N} 0 = 0.
		\end{align*}
		Therefore $\mu(B) = 0 \iff \nu(B) = 0,\ \forall B \in \sa$.
	\end{enumerate}
\end{proof}

\begin{ex}
	Consider the measure space $([0, 1], \borel, \lambda)$ where $\borel = \borel(\R) \cap [0, 1)$, i.e. the restriction of the Borel \siga to the interval $[0, 1)$, and $\lambda$ denotes the Lebesgue measure restricted to $\borel$. Define a map $T:[0, 1) \to [0, 1)$ by
	\begin{align*}
	T(x) = \begin{cases}
	4x \tif 0 \leq x \leq \frac{1}{4},\\
	\frac{4}{3}( x - \frac{1}{4}) \tif \frac{1}{4} \leq x < 1.
	\end{cases}
	\end{align*}
	
	\begin{enumerate}
		\item Show that $T$ is $\borel/\borel$-measurable (in short, Borel measurable). (1.5 pts)
		\item Consider the image measure $T(\lambda)$ defined by $T(\lambda)(B) = \lambda(\inv{T}(B))$, for all $B \in \borel$. Show that $T(\lambda) = \lambda$. (2 pts)
	\end{enumerate}
\end{ex}

\begin{proof}$ $\newline
	\begin{enumerate}
		\item It is enough to check if $\inv{T}([a,b)) \in \borel$, since $\calJ \cap [0, 1) = \{[a, b) \mid 0 \leq a \leq b \leq 1\}$ is a generator of $\borel$.
		
		\begin{align*}
		\inv{T}([a,b)) &= \{x \in [0, 1) \mid T(x) \in [a, b)\} \\
		&= \left\{x \in [0, \frac{1}{4}) \mid T(x) \in [a, b)\right\} \cup \left\{x \in [\frac{1}{4}, 1) \mid T(x) \in [a, b)\right\} \\
		&= \left\{x \mid 0 \leq a \leq x < b \leq \frac{1}{4}\right\} \cup \left\{x \mid \frac{1}{4} \leq a \leq \frac{4}{3}( x - \frac{1}{4}) < b \leq 1\right\} \\
		&= \left[\frac{a}{4}, \frac{b}{4}\right) \cup \left[\frac{3a}{4} + \frac{1}{4}, \frac{3b}{4} + \frac{1}{4}\right) \in \borel,
		\end{align*}
		since each of the intervals is itself in $\calJ \cap [0,1)$. (This is because $0 \leq \frac{a}{4}, \frac{b}{4}, \frac{3a}{4} + \frac{1}{4}, \frac{3b}{4} + \frac{1}{4} < 1$ since $a, b \in [0, 1)$.)
		
		\item First we will prove that $T(\lambda)([a, b)) = \lambda([a, b)), \forall [a, b) \in \calJ \cap [0,1)$. Using the same as in part one we have
		\begin{align*}
		T(\lambda)([a, b)) &= \lambda(\inv{T}[a,b))\\
		&= \lambda\left(\left[\frac{a}{4}, \frac{b}{4}\right) \cupdot \left[\frac{3a}{4} + \frac{1}{4}, \frac{3b}{4} + \frac{1}{4}\right)\right) \\
		&= \frac{b}{4} - \frac{a}{4} + \frac{3b}{4} + \frac{1}{4} - \frac{3a}{4} - \frac{1}{4} \\
		&= b - a = \lambda([a, b)).
		\end{align*} 
		Recall that $\borel = \sigma(\calJ \cap [0,1))$ and that there exists an exhausting sequence $B_n \uparrow [0,1)$ where $B_n = [0,1),\ \forall n \in \N$ and that $\lambda([0, 1)) < \infty$ and $T(\lambda)([0,1)) = \lambda(\inv{T}([0,1))) = \lambda([0,1)) < \infty$. Then, by uniqueness, we have that $T(\lambda) = \lambda$ on all $\borel$.
	\end{enumerate}
\end{proof}

\section{Exercise set 4}

Due October 25th, 2019.

\begin{ex}
	Let $(X, \sa, \mu)$ be a measure space and $u \in \calL^1_\Rb(\mu)$. Define $B_n = \{ x \in X \mid 2^{-n} \leq \abs{u(x)} < 2^n\}$, for $n \geq 1$. Set $B = \bigcup_{n = 1}^\infty B_n$.
	
	\begin{enumerate}
		\item Show that $\int_X u d\mu = \int_B u d\mu$. (1.5 pts)
		\item Prove that $\lim_{n \to \infty} \int_{B_n} u d\mu = \int_X u d\mu$. (1.5 pts)
		\item Show that for every $\varepsilon > 0$, there exists a positive integer $N$, such that $\mu(B_N) < \infty$ and $\abs{\int_{B_N^\complement} ud\mu} < \varepsilon$. (1.5 pts)
	\end{enumerate}
\end{ex}

\begin{ex}
	Consider the measure space $([0,1], \borel([0, 1]), \lambda)$ where $\borel([0,1])$ is the restriction of the Borel \siga to $[0,1]$, and $\lambda$ is the restriction of the one-dimensional Lebesgue measure to $[0,1]$.
	
	\begin{enumerate}
		\item Show that $\lim_{n \to \infty} \int_{[0,1]} \frac{x^n}{(1+ x)^2}d\lambda(x) = 0$. (1.5 pts)
		\item Show that $\lim_{n \to \infty} \int_{[0,1]} \frac{n x^{n-1}}{1 + x} d\lambda(x) = \frac{1}{2}$. (1 pt)
	\end{enumerate}
\end{ex}

\begin{ex}
	Let $(X, \sa, \mu)$ be a measure space, and $u \in \calL^p(\mu)$ for some $p \in [1, \infty)$. For $n \geq 1$, define $u_n = \min\{ \max(u, -n), n \}$.
	
	\begin{enumerate}
		\item Prove that $\lim_{n\to\infty} \norm{u_n}_p = \norm{u}_p$. (2 pts)
		\item Prove that for any $\varepsilon$, there exists an integer $n \geq 1$ such that $\int \abs{u - u_n}^p \leq \varepsilon$. (1 pt)
	\end{enumerate}
\end{ex}

\section{Practice Mid-Term, 2019-2020.}

\begin{ex}
	Let $(X, \sa)$ be a measure space such that $\sa = \sigma(\calG)$ where $\calG$ is a collection of subsets of $X$ such that $\emptyset \in \calG$. Show that for any $A \in \sa$ there exists a countable collection $\calG_A \subseteq \calG$ such that $A \in \sigma(\calG_A)$.
\end{ex}

\begin{proof}
	This exercise presents a good opportunity to apply the good set principle (cf. \autoref{rem:good-set-principle}).
	
	Let
	\begin{align*}
		\calB := \{ A \in \sa \mid \exists \calG_A \subset \calG \text{ countable, } A \in \sigma(\calG_A)\}.
	\end{align*}
	
	First, we will show that $\calB$ is a \siga.
	\begin{enumerate}
		\item Clearly $X \in \sa$. Let $G_X = \{\emptyset\}$ which is clearly finite and $G_X \subset \calG$ by hypothesis. Then $\sigma(\calG_X) = \{\emptyset, \emptyset^\complement\} = \{\emptyset, X\} \implies X \in \calB$.
		\item Let $A \in \calB$. Then $A \in \sa$ by definition of $\calB$ and there exists $\calG_A \subset \calG$ countable with $A \in \sigma(\calG_A)$. Since $\sigma(\calG_A)$ is a \siga it contains $A^\complement$ and clearly $A^\complement \in \sa$. Let $\calG_{A^\complement} = \calG_A$ and therefore $A^\complement \in \calB$.
		\item Finally, let $(A_n)_{n\in\N} \subset \calB$. Clearly, $(A_n)_{n\in\N} \subset \sa$ and, for each $n \in \N$, there exists $\calG_{A_n} \subset \calG$ countable such that $A_n \in \sigma(\calG_{A_n})$. Also, since $\sa$ is a \siga, we have that $\bigcup_{n\in\N}A_n \in \sa$. Let
		\begin{align*}
			\calG_{\bigcup_{n\in\N} A_n} = \bigcup_{n\in\N} \calG_{A_n}.
		\end{align*}
		$\calG_{\bigcup_{n\in\N} A_n} \subset \calG$ is countable since every $\calG_{A_n} \subset \calG$ is countable and $\calG_{\bigcup_{n\in\N} A_n}$ is a countable union of countable sets. Therefore $\bigcup_{n\in\N} A_n \in \calB$.
	\end{enumerate}

	Now we show that $\calB = \sa$. Clearly, from the definition of $\calB$ we see that $\calB \subseteq \sa$. For the reverse containment, we will show that $\calG \subset \calB$ and therefore $\sa = \sigma(\calG) \subseteq \sigma(\calB) = \calB$. Let $A \in \calG \subset \sa$. Define $\calG_A = \{A\} \subset \calG$ and clearly finite and $A \in \sigma(\calG_A)$. Therefore $A \in \calB,\ \forall A \in \calG \implies \calG\subset \calB$.
\end{proof}

\begin{ex}
	Let $X$ be a set and $\calF$ a collection of real-valued functions on $X$ satisfying the following properties:
	\begin{enumerate}
		\item $\calF$ contains the constant functions,
		\item if $f, g \in \calF$ and $c \in \R$, then $f + g,\ fg, cf \in \calF$.
		\item if $f_n \in \calF$, and $f = \lim_{n \to \infty} f_n$, then $f \in \calF$.
	\end{enumerate}

	For $A \subseteq X$, denote by $\ind_{A}$ the indicator function of $A$, i.e.
	
	\begin{align*}
		\ind_A(x) = \begin{cases}
		1 \tif x \in A,\\
		0 \tif x \not\in A.
		\end{cases}
	\end{align*}
	Show that the collection $\sa = \{ A \subseteq X \mid \ind_A \in \calF\}$ is a \siga.
\end{ex}

\begin{proof}
	We shall prove that the three properties of a \siga hold in $\sa$:
	\begin{enumerate}
		\item $X \in \sa$ since $\ind_X(x) = 1$ constantly, and therefore $\ind_X \in \calF$.
		\item if $A \in \sa$, then $\ind_A \in \calF$. We can write $\ind_{A^\complement} = 1 - \ind_A$. Since $1, 1_A \in \calF$ we have $\ind_{A^\complement} \in \calF \implies A^\complement \in \sa$.
		\item Let $(A_n)_{n\in \N} \subset \sa$. Then $(\ind_{A_n})_{n\in\N} \in \calF$. We can write
		\begin{align*}
			\ind_{\bigcup_{n\in\N} A_n} = \sum_{n=1}^{\infty} \ind_{A_i}
		\end{align*}
		To prove $\ind_{\bigcup_{n\in\N} A_i} \in \calF$ we construct the following sequence of functions $(f_n)_{n\in\N} \in \calF$ given by
		\begin{align*}
			f_n = \sum_{i=1}^{n} \ind_{A_i} = \ind_{A_n} + f_{n-1}
		\end{align*}
		Clearly, $f_n \in \calF$ for all $n$ because of property 2. Then,
		\begin{align*}
			f = \lim_{n \to \infty} f_n = \sum_{i = 1}^\infty \ind_{A_i} = \ind_{\bigcup_{n\in\N} A_n} \in \calF,
		\end{align*}
		and thus, $\bigcup_{n\in\N} A_i \in \sa$.
	\end{enumerate}
\end{proof}

\begin{ex}
	Consider the measure space $([0,1], \borel([0,1]), \lambda)$ where $\borel([0,1])$ is the restriction of the Borel \siga to to $[0,1]$, and $\lambda$ is the restriction of the Lebesgue measure to $[0,1]$. Let $E_1, \dots, E_m$ be a collection of Borel measurable subsets of $[0,1]$ such that every element $x \in [0,1]$ belongs to at least $n$ sets in the collection $\{E_j\}_{j=1}^m$, where $n \leq m$. Show that there exists a $j \in \{1, \dots, m\}$ such that $\lambda(E_j) \geq \frac{n}{m}$.
\end{ex}

\begin{proof}
	Observe that if every $x \in [0, 1]$ belongs to at least $n$ sets in $(E_j)$ then we have that
	\begin{align*}
		\sum_{j = 1}^m \ind_{E_j}(x) \geq n,\ \forall x \in [0,1].
	\end{align*}
	We proceed by contradiction. Suppose $\lambda(E_j) < \frac{n}{m},\ \forall j \in \{1, \dots, m\}$. Then
	\begin{align*}
		n = \int_{[0,1]}n d\lambda
		\overset{\ref{rem:mu-integrals-monotone}}{\leq} \int_{[0,1]} \sum_{j=1}^{m}\ind_{E_j}(x)d\lambda
		= \sum_{j = 1}^m \lambda(E_j) < m \cdot \frac{n}{m} = n. 
	\end{align*}
	This is a contradiction, so there must exists at least one $j \in \{1, \dots, m\}$ such that $\lambda(E_j) \geq \frac{n}{m}$.
\end{proof}

\begin{ex}
	Consider the measure space $(\R, \borel(\R), \lambda)$ where $\borel(\R)$ is the Borel \siga over $\R$, and $\lambda$ is the one-dimensional Lebesgue measure. Let $f_n : \R \to \R$ be defined by
	\begin{align*}
		f_n(x) = \sum_{k = 0}^{2^n - 1}\frac{3k + 2^n}{2^n} \ind_{[k/2^n, (k+1)/2^n]}(x),\ n \geq 1.
	\end{align*}
	\begin{enumerate}
		\item Show that $f_n$ is measurable, and $f_n(x) \leq f_{n+1}(x)$, for all $x \in \R$.
		\item Show that $\int \sup_{n \geq 1} f_n d\lambda = \frac{5}{2}$.
	\end{enumerate}
\end{ex}

\begin{proof}$ $\newline
	\begin{enumerate}
		\item For a given $n$ and $k$ we define
		\begin{align*}
			a^n_k = \frac{3k + 2^n}{2^n}\text{ and } A^n_k = \left[\frac{k}{2^n}, \frac{k+1}{2^n}\right).
		\end{align*}
		Clearly, $A^n_k \in \borel(\R)$ for any $n$ and $k$ and $a^n_k > 0$. Therefore, $f_n$ is a simple function and thus $f_n$ is $\sa/\borel(\R)$-measurable.
		
		Notice that
		\begin{multline*}
			A_k^n = \left[\frac{k}{2^n}, \frac{k+1}{2^n}\right)
			= \left[\frac{2k}{2\cdot 2^n}, \frac{2k+2}{2\cdot 2^n}\right)\\
			= \left[\frac{2k}{2^{n+1}}, \frac{2k+1}{2^{n+1}}\right) \cupdot \left[\frac{2k+1}{2^{n+1}}, \frac{2k+2}{2^{n+1}}\right) = A_{2k}^{n+1} \cupdot A_{2k+1}^{n+1},
		\end{multline*}
		and
		\begin{align*}
			a_k^n = \frac{3k + 2^n}{2^n} = \frac{6k + 2^{n+1}}{2^{n+1}} = a_{2k}^{n+1} \leq a_{2k}^{n+1} + a_{2k+1}^{n+1}.
		\end{align*}
		Thus,
		\begin{align*}
			f_n = \sum_{k = 0}^{2^n - 1} a_k^n \ind_{A_k^n} \leq \sum_{k = 0}^{2^n - 1} a_{2k}^{n+1} \ind_{A_{2k}^{n+1}} + a_{2k+1}^{n+1} \ind_{A_{2k+1}^{n+1}} = f_{n+1}.
		\end{align*}
		
		\item Since $(f_n)$ is an increasing sequence of simple functions we now that
		\begin{align*}
			\int \sup_{n \geq 1} f_n d\mu = \sup_{n \geq 1} \int f_n d\mu = \sup_{n \geq 1} I_\mu(f_n) = \lim_{n \to \infty} I_\mu(f_n),
		\end{align*}
		and
		\begin{align*}
			I_\mu(f_n) = \sum_{k=0}^{2^n - 1} a_k^n \mu(A_k^n)
			&= \sum_{k = 0}^{2^n - 1} \frac{3k + 2^n}{2^n} \cdot \left(\frac{k+1}{2^n} - \frac{k}{2^n}\right)\\
			&= \sum_{k = 0}^{2^n - 1} \frac{3k}{2^n \cdot 2^n} + \frac{1}{2^n} = 1 + \frac{3}{2} \frac{2^n - 1}{2^n}
		\end{align*}.
		Thus,
		\begin{align*}
			\int \sup_{n \geq 1} f_n d \mu = \lim_{n \to \infty} I_\mu(f_n) = 1 + \frac{3}{2} = \frac{5}{2}.
		\end{align*}
	\end{enumerate}
\end{proof}

\begin{ex}
	Let $\mu$ and $\nu$ be two measures on the measure space $(E, \borel)$ sucht that $\mu(A) \leq \nu(A)$ for all $A \in \borel$. Show that if $f$ is any non-negative function on $(E, \borel)$, then $\int_E f d\mu \leq \int_E f d\nu$.
\end{ex}

\begin{proof}
	\footnote{Now that the solutions to this practice midterm have been published, I noticed there is a much more cleaner (and modular!) way of doing this. First prove it for $f = \ind_A$ for some $A\in \sa$. Then, prove it for $f \in \calE^+(\sa)$, for some simple function $f$. Finally, prove it for the general case. Each of these builds on the previous and you're left with something much more readable and useful.}By \autoref{thm:sombrero} we now that there exists a sequence of non-negative simple functions $(f_n)_{n\in\N} \subset \calE^+(\borel)$ such that $f_n \uparrow f$. Moreover, each of this simple functions has a representation of the form
	\begin{align*}
		f_n = \sum_{j=0}^{M_n} a_j \ind_{A_j}
	\end{align*}
	Thus,
	\begin{multline*}
		\int_E fd\mu
		= \lim_{n \to \infty} \int_E f_n d\mu
		= \lim_{n \to \infty} I_\mu(f_n) \\
		= \lim_{n \to \infty} \sum_{i = 1}^n \sum_{j=0}^{M_n} a_j \mu(A_j)
		\leq \lim_{n \to \infty} \sum_{i = 1}^n \sum_{j = 0}^{M_n} a_j \nu(A_j) \\
		= \lim_{n \to \infty} I_\nu(f_n) 
		= \lim_{n \to \infty} \int_E f_n d\nu
		= \int fd\nu
	\end{multline*}
\end{proof}
	
	
	\listoftheorems[ignore={dfn,ex,eg,remark,lem,cor}]
	
	\bibliographystyle{IEEEtran}
	\bibliography{IEEEabrv,mi-notes}
\end{document}