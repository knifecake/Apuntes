\chapter{Basic tools for mathematical modelling}

Teaching for this chapter started on Monday, 2019.11.11 (week 46a) and ended on
Monday, 2019.11.18. This chapter corresponds to part of chapter 1 in
\cite{eck2017mathematical}.

In this introductory chapter we will introduce the mindset that we should have
when trying to \textbf{translate a specific problem} from the natural sciences,
the social sciences or technology into a \textbf{well-defined mathematical
problem}\footnote{This is the definition of \textit{mathematical modelling}
given in \cite[p.  1]{eck2017mathematical} with Kreisbeck's emphasis.}.

TODO: some context and general pointers would probably look good here.

\section{Case study: population dynamics}

Suppose we want to model the change in population (i.e. number of individuals)
in an environment over a period of time. First thing we need is to make some
assumptions about what's really happening here. We might, for example, make the
following assumptions\footnote{Stolen from \cite{kreisbeck2019slides}.}

\begin{enumerate}
  \item growth rate independent of population size (unlimited growth possible,
    neglecting e.g. limited resources)

  \item growth rate independent of time (neglecting time-dependence due to e.g.
    influence of enemies, economical or cultural changes)

  \item population within closed systems (neglecting e.g. migration)

  \item assuming an equal distribution of male and female, age distribution not
    considered

  \item continuous model with non-integer solutions (idealization reasonable
    for very large populations, for small populations stochastic effects have
    to be taken into account)
\end{enumerate}

After this, we name the quantities that intervene in our problem. We will use
$t$ for time, $x(t)$ for the number of individuals (population) at time $t$ and
$\frac{dx}{dt}(t)$ or $x'(t)$ for the rate of change in population. To model
the change we introduce the quantities
\begin{itemize}
  \item $b(t, \Delta t)$ for the increase of population during the time
    interval $(t, \Delta t)$, and
  \item $d(t, \Delta t)$ for the decrease of population during the time
    interval $(t, \Delta t)$.
\end{itemize}

Therefore the population at time $t + \Delta t$ is given by
\[
  x(t + \Delta t) = x(t) + b(t, \Delta t) - d(t, \Delta t).
\]
That $\Delta t$ desperately wants us to take the limit as $\Delta t \to 0$ and
so we do
\[
  \lim_{\Delta t \to 0} \frac{x(t + \Delta t) - x(\Delta t)}{\Delta t}
  = \lim_{\Delta t \to 0} \frac{b(t, \Delta t)}{\Delta t} - \lim_{\Delta t \to 0} \frac{d(t, \Delta t)}{\Delta t}.
\]

Note that here we're assuming the limit really does exist, which is quite a big
assumption\ldots\ Rename,
\[
  B(t) = \lim_{\Delta t \to 0} \frac{b(t, \Delta t)}{\Delta t} \text{ and }
  D(t) = \lim_{\Delta t \to 0} \frac{d(t, \Delta t)}{\Delta t}
\]
and use the definition of derivative to get
\[
  \frac{dx}{dt}(t) = x'(t) = B(t) - D(t)
\]
where $B(t)$ and $D(t)$ being the rates at which the population increasing,
resp. decreases at time $t$. Recall that we assumed that the rates of change in
the population were independent of time and population size. This is equivalent
to saying that $B(t)$ and $D(t)$ are really constants which gives us the final
model
\[
  x'(t) = \beta - \delta \implies x(t) = (\beta - \delta)t.
\]
The previous is a not-particularly-interesting ODE with solution\footnote{For
help with solving differential equations see \cite{math24}.}
\[
  x(t) = (\beta - \delta)x + C.
\]

This model has a lot of shortcomings, first of all, it does not account for the
size of the population in the rates of change. But, one might argue that the
more individuals there are in a population the greater the rates of change are.
We can go back and restate assumption one as \say{population increase, resp.
decrease in the time interval $(t, \Delta t)$ is directly proportional to the
population at time $t$ and the time passed}.  This in turn gives us
\[
  b(t, \Delta t) = \beta x(t) \Delta t \text{ and }
  d(t, \Delta t) = \delta x(t) \Delta t.
\]
Taking the limit as before leads us to the model
\[
  x'(t) = (\beta - \delta)x(t) = px(t).
\]
From now on, we shall let $p = \beta - \delta$ since we don't really need to
distinguish between changes in the population because of births and deaths---we
just care about the overall evolution of the population. This is another ODE,
this time a bit more interesting, with solution
\[
  x(t) = C e^{p t}.
\]

Although a bit better, you can probably see that this model explodes as time
passes since it does not include any provisions for when the population turns
stupidly large. Anyhow, it is common enough that it deserves its own name: the
\textbf{exponential growth model}.

A small step in the right direction would be to account for a population limit
in the system, i.e. number of individuals that flips the rate of growth. More
precisely, let's change assumption one to \say{there is a number $x_M$ that is
  the maximum population in the system (sometimes called the \textit{carrying
capacity} and that the rate of change in population $p(x(t))$ is positive if
$x(t) > x_M$ and negative if $x(t) > x_M$}. The easiest way to model this is
with a linear ansatz for $p(x)$, i.e. something of the form $p(x) = q(x_M - x)$
with a parameter\footnote{The meaning of $q$ is a bit more complex to explain,
but at heart it is just a proportionality constant.} $q > 0$. Notice how
\[
  \begin{cases}
    p(x) > 0 \tif x < x_M,\text{ and } \\
    p(x) < 0 \tif x > x_M
  \end{cases}
  .
\]
Plugging this into our previous model to get
\[
  x'(t) = q(x_M - x(t))x(t) = q x_M x(t) - q x^2(t),
\]
which is our final model for now and is called the
\textbf{logistic growth model}

This ODE can be solved exactly and the solution is
\[
  x(t) = \frac{x_M x_0}{x_0 + (x_M - x_0) e^{- x_M q (t - t_0)}},
\]
where $t_0$ is the initial time and $x_0 = x(t_0)$ is the initial population.


\begin{figure}
  [h]
  \incfig{logistic-growth-model}

  \caption{Solutions for the three iterations of the model, for different
  values for the parameters on each version.}
\end{figure}

We'll stop here for now, but keep in mind that we're missing the second half of
the solution---we still need to apply this models to the real world. This means
\textit{fitting} those curves to the specific problem at hand, in this case,
getting some data (at least two data points) to calculate the constants that
are present in our solutions. Also, recall that we made a lot of assumptions,
there are more population dynamics models that account for changes in the
environment, migrations, etc.



\section{Dimensional analysis and non-dimensionalisation}

The previous models had two or three parameters each, but as we work our way to
more complex examples the number of parameters will increase. Moving around all
those constants is cumbersome and draws our attention away from really
understanding the problem at hand. In addition, as we apply our models to
specific problems we will need to take into account the units of the quantities
we are dealing with.

\subsection{Dimensional analysis}

In addition to being a prerequisite to doing non-dimensionalisation,
dimensional analysis provides a sanity check for us to \textit{make sure we're
not adding apples to oranges}. When coming up with a model, we generally need
to specify the \textit{physical dimension} of the quantities involved, but not
necessarily want to specify the particular units that quantity is expressed in.
For this we will denote by $[c]$ the physical dimension of a quantity $c$. For
example if $t$ denotes time, when we write $[t]$ we mean the \textit{dimension
of time} or \textit{some units of time} but do not specify which.

Revisiting our population model we can define the characteristic units
\[
  T := \text{ time },\qquad \text{ and } \qquad N := \# \text{ of individuals }
\]
so that we get the following dimensions for the involved quantities
\[
  [t] = T,\ [x(t)] = N,\ [x'(t)] = \frac{N}{T}.
\]

We can also do this for the parameters by solving for them in the equation for
the model\footnote{Notice how $[x_M - x] = N$ and not something weird like $N -
N = 0$, since when you subtract apples from apples you still get apples.}
\[
  [x_M] = N,\ [t_0] = T,\ [x_0] = N,\ 
  [q] = \frac{[x']}{[x_M - x][x]} = \frac{N / T}{N \cdot N} = \frac{1}{T}. 
\]

\subsection{Non-dimensionalisation}

Once we know the physical units of all involved quantities in our model we are
ready to choose actual units for our model. For instance, for time we might
choose years, days or hours but most of the time it is better to choose
appropriate units for our problem. Non-dimensionalisation\footnote{Yes, this is
an accepted spelling although not very common in the literature.} is a recipe
for choosing the most appropriate units.

From the dimensional analysis of our population dynamics examples we now that
there are two physical dimensions and therefore we will choose two
characteristic quantities $\overline{t}$ and $\overline{x}$.

\subsection{Non-dimensionalisation when there are several options. The
projectile problem.}

\section{Asymptotic expansion method}

\subsection{Error estimation}
