\chapter{Partial differential equations}

Unlike with ODEs, there is no general theorem for existence, uniqueness or
structure of solutions for PDEs. In this section we will classify some of the
most common PDEs and try to give some results for each category that help solve
them or gain some insight on how the solutions behave.

\begin{dfn}
  A partial differential equation or PDE is an identity of the form
  \begin{equation}
    \label{eq:pde}
    F(D^k, \dots, D^2, \nabla u, u, x) = 0
  \end{equation}
  where
  \begin{itemize}
    \item $x \in \Omega \subset \R^n$ is the independent variable,
    \item $u : \Omega \subset \R^n \to \R$ is the function we want to solve
      \eqref{eq:pde} for,
    \item TODO
  \end{itemize}

  TODO: comment about the order
\end{dfn}


\section{First-order partial differential equations}

In general, first-order PDEs can be linear, semilinear, quasilinear or
non-linear. An example of a non-linear equation is the TODO eq
\[
  \norm{\nabla u(x)} = 1.
\]
An example of a linear equation is the generic transport equation
\[
  \partial_t u(t, x) + \partial_x u(t, x) = 0.
\]

In general all but non-linear equations can be solved using the so called
\textbf{method of characteristics} which exploits certain geometric properties
to reduce the PDE to a system of ODEs.

\subsection{Method of characteristics for linear partial differential equations}

\section{Second-order partial differential equations}

In this harder case, we will restrict ourselves to just linear equations.

\begin{dfn}
  [Second-order, linear PDE]

  We say the identity
  \begin{equation}
    \label{eq:semilinear-2o}
    \sum_{j, k = 1}^n a_{jk}(x) \partial_{x_j} \partial_{x_k} u(x)
    + F(\nabla u(x), u(x), x) = 0
  \end{equation}
  with
  \[
    A(x) = \left( a_{jk}(x) \right)_{j, k = 1}^n \in \R^{n \times n} \text{ symmetric}
  \]
  is a linear second-order partial differential equation.
\end{dfn}

Note that the symmetricness of the matrix $A(x)$ comes from Schwarz's theorem
for mixed derivatives.

\begin{dfn}
  Consider a semilinear second-order PDE of the form of
  \eqref{eq:semilinear-2o}. For any $x \in \Omega$ we say
  \eqref{eq:semilinear-2o} is
  \begin{itemize}
    \item \textbf{elliptic} at $x$ if all the eigenvalues of $A(x)$ are
      non-zero and have the same sign,
    \item \textbf{parabolic} at $x$ if $A(x)$ has the eigenvalue 0 with
      multiplicity one and all other eigenvalues have the same sign, and
    \item \textbf{hyperbolic} at $x$ if 0 is not an eigenvalue of $A(x)$ and
      exactly $n - 1$ eigenvalues have the same sign.
  \end{itemize}

  If any of the previous contitions holds for every $x \in \Omega$ we say
  \eqref{eq:semilinear-2o} is \textbf{globally} elliptic, resp. parabolic,
  hyperbolic.
\end{dfn}

\begin{eg}
  The Laplace equation $\Delta u(x) = 0$ is globally elliptic since
  \[
    \Delta u(x) = \sum_{i = 1}^n \partial^2_{x_i} u(x)
    = \sum_{i = 1}^n 1 \partial_{x_i} \partial_{x_i} u(x)
    \implies A(x) = \Id.
  \]
\end{eg}

\begin{eg}
  The heat equation $\partial_t u(t, x) - \Delta u(t, x) = 0$ is globally
  parabolic since
  \[
    A(x) =
    \left(
      \begin{array}{c|ccc}
        0 & 0 & \dots & 0 \\ \hline
        0 & -1 & \dots & 0 \\
        0 & 0 & \ddots & 0 \\
        0 & 0 & \dots & -1
      \end{array}
    \right).
  \]
\end{eg}

\begin{eg}
  The wave equation $\partial^2_t u(t, x) - \Delta u(t, x) = 0$ is globally
  hyperbolic since
  \[
    A(x) =
    \left(
      \begin{array}{c|ccc}
        1 & 0 & \dots & 0 \\ \hline
        0 & -1 & \dots & 0 \\
        0 & 0 & \ddots & 0 \\
        0 & 0 & \dots & -1
      \end{array}
    \right).
  \]
\end{eg}

\subsection{Results for elliptic second-order partial differential equations}

Since the elliptic is the simplest of the three categories, we will use the
remaining time to give some results about two very important examples of these
equations: the Laplace equation and the Poisson equation.

\subsubsection{Fundamental solution to the Laplace equation}

Recall that the Laplace equation is of the form
\begin{equation}
  \label{eq:pde-laplace}
  \Delta u(x) = 0
\end{equation}
for some $u$ TODO.

We make the ansatz $u(x) = w(\norm{x}) = w(r)$, i.e. we assume rotational
symmetry of the solution. If this were the case then \eqref{eq:pde-laplace} becomes
\[
\Delta u
\]

\begin{dfn}
  [Convolution product]

  Given two functions $f, g : \Omega \subset \R^n \to \R$ we define the
  convolution product of $f$ and $g$ as
  \[
    (f \ast g)(x) = \int_\Omega f(x - y) g(y) dy.
  \]
\end{dfn}

That we name this weird thing product is not coincidental, as it shares some
properties with the standard number product.

\begin{thm}
  [Properties of the convolution product]
   Let $f, g : \Omega \subset \R^n \to \R$ be two functions. Then
   \begin{enumerate}
     \item (Conmutativity) $f \ast g = g \ast f$, and
     \item (Linearity with respect to differentiation)
       \[
         \partial_{x_j}^k (f \ast g)
         = (\partial_{x_j}^k f) \ast g
         = f \ast ( \partial_{x_j}^k g).
       \]
   \end{enumerate}
\end{thm}

We will finish the course with two results about vector valued functions.

\begin{thm}
  Let $\Omega \subset \R^n$ and let $B_R(x) \subset \Omega$ be a ball with
  radius $R > 0$ centered in $x \in \Omega$. If $u \in C^2(\Omega)$ is harmonic
  then
  \[
    u(x)
    = \frac{1}{\norm{\partial B_R(x)}} \int_{\partial B_R(x)} u(y) d s_y
    = \frac{1}{\norm{B_R(x)}} \int_{B_R(x)} u(y) dy,
  \]
  where $\norm{\partial B_R(x)}$ denotes the hypersurface area of the boundary
  of $B_R(x)$ and $\norm{B_R(x)}$ denotes the hypervolume of the ball $B_R(x)$.
\end{thm}

Note that in the previous theorem we have not inlined the well known formulas
for the area and volume of a sphere as these only hold for $R^3$.


TODO: other theorem or maximum principle.




