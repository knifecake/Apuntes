\chapter{Continuum mechanics}

This chapter deals with the physics of a very general body that we describe
using calculus, hence the name continuum. The main assumption that we make to
be able to do this is that the behaviour of a system can be described by
averaging some characteristic quantities over the space it occupies, instead of
computing every one of them for every atom or molecule, i.e. that the involved
quantities in a problem are defined in a continuum. It turns out this
assumption is pretty good and the models we will get closely describe what is
happening in reality.

\section{Frames of reference and coordinate systems}

Before we begin we need a way to describe points in space. We will do this
using the standard definitions and results of affine spaces that we get from
linear algebra.

\begin{dfn}
  Let $\{e_1, \dots, e_n\}$ be an orthonormal basis for $R^n$ and let $O \in
  \R^n$ be a point which we will call the origin. In the equation
  \[
    X = O + \sum_{i = 1}^n x_i e_i,
  \]
  we define $(x_1, \dots, x_n)$ to be the coordinates of the point $X$ with
  respect to the origin $O$.
\end{dfn}

Notice how in the above definition we talked about points and not vectors when
dealing with $X$ and $O$. The distinction is rather technical and can almost
always be ignored but in short the deal is that points live in the affine space
$\mathbb{A}^n_\R = (A, \R^n, f)$ while vectors live in $\R^n$. An affine space
always comes with a function that maps points to vectors and is generally of
the form
\[
  f : A \times A \to \R^n,\qquad (P, Q) \mapsto \overrightarrow{PQ},
\]
where $\overrightarrow{PQ}$ denotes the vector which starts at $P$ and ends at
$Q$ that can be computed by treating $P$ and $Q$ as vectors in $\R^n$ and
taking the difference. More details can be found in \cite{wikiaffine}.


One important takeaway is the following lemma.

\begin{lem}
  Let $(O; e_1, \dots, e_n)$ and $(O^*, e_1^*, \dots, e_n^*)$ be two affine
  coordinate systems and let $X$ be a point in $\mathbb{A}^n_\R$ defined in the
  two coordinate systems, i.e.
  \[
    X = O + \sum_{i = 1}^n x_i e_i = O^* + \sum_{i = 1}^n x_i^* e_i^*,
  \]
  then there exists $a \in \R^n$ and $Q \in \R^{n\times n}$ orthogonal such that
  \[
    x^* = a  + Q x.
  \]
\end{lem}


\section{Introduction and setup of continuum mechanics}

\subsection{Lagrangian and Eulerian coordinate systems}

Some of the problems we will focus on require us to describe the evolution of
the shape of a body. We might do this by following the position of a point
inside the body as time passes. To be precise, we choose an initial state or
\textbf{reference configuration} $\Omega \subset \R^n$, which is normally the
set of all points in space which are part of the body under consideration. Lets
say that \textbf{material point} is called $X \in \Omega$. We can think of $X$
as a way to pinpoint an atom in the body and lets assume that we can track it
as time passes. We might describe its position with the function
\[
  x : \R \times \R^n \to \R^n,\qquad (t, X) \mapsto x(t, X).
\]

On the other hand, we may be interested in some other quantities, for instance
the temperature at a point in the body. While it could be the case that we are
interested in the temperature of a particular atom (even if that made sense), it
is most likely that we are looking for the evolution of temperature at a fixed
point in space. It could even be that we are dealing with a fluid and looking
at individual atoms or molecules does not make sense. In both of this cases we
are interested in a quantity $\varphi(t, x)$ where $x \in \R^n$. The choice of
the letter $x$ for the physical point is not a coincidence since for a material
point $X$ in the reference configuration we can get its position at time  $t$
with $x(t, X)$ and therefore $\varphi(t, x) = \varphi(t, x(t, X))$.

Let us formalise these concepts. We start by making suitable assumptions
\begin{enumerate}
  \item The material point $X$ is described by its position at some initial
    time $t = t_0$ by the function $X = x(t_0, X)$.
  \item The mapping $(t, X) \mapsto x(t, X)$ is continuously differentiable.
  \item For all $t \geq t_0$, the mapping $x$ is invertible.
  \item $x$ is orientation preserving, i.e. the matrix\footnote{The book
      \cite[p. 207]{eck2017mathematical} calls the determinant of this matrix
      the Jacobi determinant. In reality, $Jx$ is just the the differential
      matrix $Dx$ only we omit the first row since we are interested in the
      derivatives with respect to the coordinates of the material point $X$.
      Notice how we say differential matrix since the Jacobian matrix is
      guaranteed to exist and be the derivative because we asked for $x$ to be
      continuously differentiable.}
    \[
      Jx(t, X)_{ij} = \left( \frac{\partial x_i}{\partial X_j} (t, X) \right)
    \]
    has a positive determinant for every $t \geq t_0$ and every $X \in \Omega$.
\end{enumerate}

The terms $x$ and $X$ represent two different types of coordinates\footnote{I
tried to phrase this first part of the section as a standard
definition-properties introductory section. However, these constructs are
sometimes just notational conventions used by people who work with different
branches of continuum mechanics and hence are not normally that clearly
defined. One may take this section with a grain of salt and just make sure the
math makes sense when dealing with equations on each coordinate system.}
corresponding to the two ways of looking at a problem described before:
\begin{itemize}
  \item \textbf{Lagrangian coordinates} $X$ allow us to choose one material point and
    follow its evolution, while
  \item \textbf{Eulerian coordinates} $x$ relate to a fixed point in space.
\end{itemize}

In general, we will observe different material points at different times at
point $x$ specified in Eulerian coordinates.

\subsection{Analysis of relevant quantities in Lagrangian and Eulerian coordinate systems}

Before we move on, let's take a moment to really formalise how the function $x$
is defined and to look at how we might take the derivatives. In the book
(\cite[p. 208]{eck2017mathematical}) we can find the following definition

\begin{dfn}
  [Material time derivative]

  Let $\varphi : \R \times \R^3 \to \R$. We define the material time derivative
  of $\varphi$ with respect to $t$ as
  \[
    D_t\varphi(t, x) = \partial_t \varphi(t, x) + \nabla \varphi(t, x) \cdot v(t, x),
  \]
  where $v(t, x) = \partial_t x(t, X(t, x))$.
\end{dfn}

To me, this did not make sense at all on the first reading. It also didn't make
sense on the second, third, and $n$-th readings, for a substantially large $n$.
What helped me understand it was to not take it as a definition, but to try to
arrive at the same result with standard vector calculus results, i.e. the chain
rule. In what follows I try to describe this to the best of my ability.

Take any $X_0 \in \Omega$ and define $x_0(t) = x(t, X_0)$, i.e. fix the second
parameter of the function $x(t, X)$ so that it describes the position of only
one particular material point $X_0$. In this case,
\[
  \ddt x_0(t) = \left. \partial_t x(t, X)\right|_{X = X_0}.
\]
Now going back to our relevant quantity $\varphi$ we do the same
thing---restrict it to work for just one material point $X_0$. In this case
$\varphi$ becomes $\varphi(t, x_0(t))$.







We will use the convention that quantities expressed as a function of
Lagrangian coordinates will be denoted by [greek] capital letters like $\Phi(t,
X)$ while quantities defined on Eulerian coordinates will be identified by a
lowercase letter $\varphi(t, x(t, X))$. It is important to observe that here
\[
  \Phi(t, X) = \varphi(t, x(t, X)),
\]
that is, they denote the same quantity and hence evaluate to the same
mathematical object (in some cases we use vectors or matrices instead of just
numbers to represent relevant quantities).

One instance where this convention is used is when dealing with the velocity
fields, which are
\[
  V(t, X) = \frac{\partial x}{\partial t}(t, X)
\]
in Lagrangian coordinates, and
\[
  v(t, x) = V(t, X(t, x))
\]
in Eulerian coordinates. Here $X$ is the inverse of the function $x$ which we
assumed to exist for $t \geq t_0$.

More generally, we may wish to calculate the derivative of a quantity. In
Eulerian coordinates, when we denote a quantity by $\varphi(t, x)$ we must not
forget the fact that $x$ is itself a function of $t$ and a material point $X$.
More explicitly, we may write $\varphi(t, x(t, X))$ as before to make it easier
to compute the derivative with respect to time using the chain rule.


We conclude this section by defining two sets of important curves, namely
pathlines and streamlines.

\begin{dfn}
  [Pathlines]

  Pathlines are solutions of the equation
  \[
    y'(t) = v(t, y(t)).
  \]
\end{dfn}

They describe the paths followed by a material point during the evolution of
the system.

\begin{dfn}
  [Streamlines]

  Streamlines are solutions to the equation
  \[
    z'(s) = v(t, z(s)).
  \]
\end{dfn}

They are curves tangent to the velocity field at every point, i.e. they give a
\textit{snapshot} of the velocity field at time $t$.

Observe that in streamlines we treat $t$ as a parameter whereas the equation
for pathlines is more involved. See the solution TODO add reference to extra problem.


Throughout the next sections, the structure will be the following. We will
first focus on fluids and for that we will use Eulerian coordinates.  This will
take the greater part of the chapter. Finally, we will conclude with elastic
solids, for which we will use Lagrangian coordinates.

\section{Reynold's transport theorem}

We start with some basic results from calculus / algebra.

\begin{lem}
  [Jacobi's formula]
  Let $t \mapsto A(t) \in \R^{n \times n}$ be a differentiable map such that
  $A(t)$ is invertible. Then,
  \[
    \ddt \det A(t) = \tr \left( \inv{A}(t) \ddt A(t) \right) \det A(t).
  \]
\end{lem}

This result can be proved by setting $\det A = F(a_{11}, \dots, a_{nn})$ as a
function of the entries of the matrix $A$ and appyling the cain rule with a lot
of care. See TODO for details.

\begin{thm}
  Let $(t, X) \mapsto x(t, X)$ be a continuously differentiable, invertible and
  orientation preserving mapping, and that $(t, X) \mapsto \partial_t x(t, X)$
  is also continuously differentiable. Then,
  \[
    \partial_t J(t, X) = \left. \nabla \cdot v(t, x)\right|_{x = x(t, X)} J(t, X),
  \]
  where $J$ is the matrix introduced in the previous section.
\end{thm}
